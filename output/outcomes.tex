\newpage

\hypertarget{ux30d7ux30edux30d5ux30a7ux30c3ux30b7ux30e7ux30caux30eaux30baux30e0}{%
\section{プロフェッショナリズム}\label{ux30d7ux30edux30d5ux30a7ux30c3ux30b7ux30e7ux30caux30eaux30baux30e0}}

人の命に深く関わり健康を守るという医師の職責を十分に自覚し、多様性・人間性を尊重し、利他的な態度で診療にあたりながら、医師としての道を究めていく。

\hypertarget{ux4fe1ux983c}{%
\subsection{信頼}\label{ux4fe1ux983c}}

社会から信頼を得る上で必要なことを常に考え行動する。

\hypertarget{ux8aa0ux5b9fux3055}{%
\subsubsection{誠実さ}\label{ux8aa0ux5b9fux3055}}

\begin{enumerate}
\def\labelenumi{\arabic{enumi}.}
\tightlist
\item
  患者や社会に対して誠実である行動とはどのようなものかを考え、そのように行動する。
\item
  社会から信頼される専門職集団の一員であるためにはどのように行動すべきかを考え、そのように行動することができる。
\end{enumerate}

\hypertarget{ux7701ux5bdf}{%
\subsubsection{省察}\label{ux7701ux5bdf}}

\begin{enumerate}
\def\labelenumi{\arabic{enumi}.}
\tightlist
\item
  自分自身の限界を適切に認識し行動する。
\item
  他者からのフィードバックを適切に受け入れる。
\end{enumerate}

\hypertarget{ux601dux3044ux3084ux308a}{%
\subsection{思いやり}\label{ux601dux3044ux3084ux308a}}

品格と礼儀を持って、他者を適切に理解し、思いやりを持って接する。

\hypertarget{ux601dux3044ux3084ux308a-1}{%
\subsubsection{思いやり}\label{ux601dux3044ux3084ux308a-1}}

\begin{enumerate}
\def\labelenumi{\arabic{enumi}.}
\tightlist
\item
  患者を含めた他者に思いやりをもって接する。
\item
  他者に思いやりをもって接することができない場合の原因・背景を考える。
\end{enumerate}

\hypertarget{ux4ed6ux8005ux7406ux89e3ux3068ux81eaux5df1ux7406ux89e3}{%
\subsubsection{他者理解と自己理解}\label{ux4ed6ux8005ux7406ux89e3ux3068ux81eaux5df1ux7406ux89e3}}

\begin{enumerate}
\def\labelenumi{\arabic{enumi}.}
\tightlist
\item
  自身の想像力の限界を認識した上で、他者を理解することに努める。
\item
  他者を適切に理解するための妨げとなる自分や自集団の偏見とはどのようなものか考え、それらを意識して行動する。
\end{enumerate}

\hypertarget{ux54c1ux683cux793cux5100}{%
\subsubsection{品格・礼儀}\label{ux54c1ux683cux793cux5100}}

\begin{enumerate}
\def\labelenumi{\arabic{enumi}.}
\tightlist
\item
  医師に求められる品格とはどのようなものかを考え、それを備えるように努める。
\item
  礼儀正しく振る舞う。
\end{enumerate}

\hypertarget{ux793eux4f1aux6b63ux7fa9}{%
\subsection{社会正義}\label{ux793eux4f1aux6b63ux7fa9}}

社会的公正を実現する。

\hypertarget{ux533bux7642ux8cc7ux6e90ux306eux516cux5e73ux306aux5206ux914d}{%
\subsubsection{医療資源の公平な分配}\label{ux533bux7642ux8cc7ux6e90ux306eux516cux5e73ux306aux5206ux914d}}

\begin{enumerate}
\def\labelenumi{\arabic{enumi}.}
\tightlist
\item
  医療資源を公平に分配するとはどういうことか考え、自らの意見を述べる。
\end{enumerate}

\hypertarget{ux6559ux990a}{%
\subsection{教養}\label{ux6559ux990a}}

医師に相応しい教養を身につける。

\hypertarget{ux6559ux990a-1}{%
\subsubsection{教養}\label{ux6559ux990a-1}}

\begin{enumerate}
\def\labelenumi{\arabic{enumi}.}
\tightlist
\item
  人の生命に深く関わる医師に相応しい教養を身につける。
\item
  答えのない問いについて考え続ける。
\end{enumerate}

\hypertarget{ux533bux306eux502bux7406}{%
\subsection{医の倫理}\label{ux533bux306eux502bux7406}}

医療における倫理の重要性を学ぶ。

\hypertarget{ux533bux306eux502bux7406-1}{%
\subsubsection{医の倫理}\label{ux533bux306eux502bux7406-1}}

\begin{enumerate}
\def\labelenumi{\arabic{enumi}.}
\tightlist
\item
  生と死に関わる倫理的問題を概説できる。
\item
  医師の職業倫理指針に関する規範を概説できる。
\end{enumerate}

\newpage

\hypertarget{ux7dcfux5408ux7684ux306bux60a3ux8005ux751fux6d3bux8005ux3092ux307fux308bux59ffux52e2}{%
\section{総合的に患者・生活者をみる姿勢}\label{ux7dcfux5408ux7684ux306bux60a3ux8005ux751fux6d3bux8005ux3092ux307fux308bux59ffux52e2}}

患者の抱える問題を臓器横断的に捉えた上で、心理社会的背景も踏まえ、ニーズに応じて柔軟に自身の専門領域にとどまらずに診療を行い、個人と社会のウェルビーイングを実現する。

\hypertarget{ux5168ux4ebaux7684ux306aux8996ux70b9ux3068ux30a2ux30d7ux30edux30fcux30c1}{%
\subsection{全人的な視点とアプローチ}\label{ux5168ux4ebaux7684ux306aux8996ux70b9ux3068ux30a2ux30d7ux30edux30fcux30c1}}

患者の抱える問題を臓器横断的だけでなく心理・社会的視点で捉え、専門領域にとどまらない姿勢で責任をもって診療に関わり、最善の意思決定や行動科学に基づく臨床実践に関与できる。

\hypertarget{ux81d3ux5668ux6a2aux65adux7684ux306aux8a3aux7642}{%
\subsubsection{臓器横断的な診療}\label{ux81d3ux5668ux6a2aux65adux7684ux306aux8a3aux7642}}

\begin{enumerate}
\def\labelenumi{\arabic{enumi}.}
\tightlist
\item
  臓器横断的に医学的課題を捉えることができる。
\item
  適切な医療機関や診療科につなぐ重要性を理解している。
\item
  基本的なフレームワーク(頻度・重症度・緊急度、解剖学的アプローチ、病態生理学的アプローチ、二重過程理論、事前確率等)を用いて臨床推論を行うことができる。
\item
  主訴に応じて必要な医療面接・身体診察・検査を実施できる。
\item
  診断がつかない健康問題やその介入方法を理解している。
\item
  多疾患が併存した状態及び複数臓器にまたがる疾患について、その介入方法を理解している。
\item
  ポリファーマシーとその介入方法を理解している。
\end{enumerate}

\hypertarget{ux751fux7269ux5fc3ux7406ux793eux4f1aux7684ux306aux554fux984cux3078ux306eux5305ux62ecux7684ux306aux8996ux70b9}{%
\subsubsection{生物・心理・社会的な問題への包括的な視点}\label{ux751fux7269ux5fc3ux7406ux793eux4f1aux7684ux306aux554fux984cux3078ux306eux5305ux62ecux7684ux306aux8996ux70b9}}

\begin{enumerate}
\def\labelenumi{\arabic{enumi}.}
\tightlist
\item
  身体・心理・社会の問題を統合したアプローチを理解している。
\item
  個人・家族の双方への影響を踏まえたアプローチを理解している。
\item
  削除:トラウマインフォームドケアの対応について概説できる。:コンテクストを確認(春田先生)
\end{enumerate}

\hypertarget{ux60a3ux8005ux4e2dux5fc3ux306eux533bux7642}{%
\subsubsection{患者中心の医療}\label{ux60a3ux8005ux4e2dux5fc3ux306eux533bux7642}}

\begin{enumerate}
\def\labelenumi{\arabic{enumi}.}
\tightlist
\item
  個々の患者の医療への期待、解釈モデル、健康観を聞き出すことができる。
\item
  患者の社会的背景(経済的・制度的側面等)が病いに及ぼす影響を理解している。
\item
  医療の継続性(時間・情報・関係等)がもたらす影響を理解している。
\end{enumerate}

\hypertarget{ux6839ux62e0ux306bux57faux3065ux3044ux305fux533bux7642}{%
\subsubsection{根拠に基づいた医療}\label{ux6839ux62e0ux306bux57faux3065ux3044ux305fux533bux7642}}

\begin{enumerate}
\def\labelenumi{\arabic{enumi}.}
\tightlist
\item
  根拠に基づいた医療(EBM)の5つのステップを列挙できる。
\item
  PICO(PECO)を用いた問題の定式化ができる。
\item
  データベースや二次文献からのエビデンス、診療ガイドラインを検索することができる。
\item
  得られたエビデンスの批判的吟味ができる。
\item
  診療ガイドラインの種類、推奨の強さ、使用上の注意を理解している。
\item
  患者の個別性や状況を考慮してエビデンスの適用について考えることができる
\end{enumerate}

\hypertarget{ux884cux52d5ux79d1ux5b66}{%
\subsubsection{行動科学}\label{ux884cux52d5ux79d1ux5b66}}

\begin{enumerate}
\def\labelenumi{\arabic{enumi}.}
\tightlist
\item
  行動科学に関する知識・理論・面接法を予防医療、診断、治療、ケアに適用できる。
\item
  適切な環境調整や認知行動療法を提案できる。
\item
  健康に関する行動経済学の知識を活用できる。
\end{enumerate}

\hypertarget{ux7de9ux548cux30b1ux30a2}{%
\subsubsection{緩和ケア}\label{ux7de9ux548cux30b1ux30a2}}

\begin{enumerate}
\def\labelenumi{\arabic{enumi}.}
\tightlist
\item
  緩和ケアの概念を理解した上で、全人的苦痛(身体的苦痛、心理社会的苦痛、スピリチュアルペイン)を評価できる。
\item
  がん・非がんの疼痛緩和の薬物療法や非薬物療法について理解している。
\item
  救急・急性期医療における治療・ケアに関して、人生の最終段階における医療(エンド・オブ・ライフ・ケア)を踏まえた患者・家族とのコミュニケーションの意義を理解し、頻度の高い苦痛とその対処法・ケアを計画できる。
\item
  慢性疾患や慢性疼痛の病態、経過、治療を理解した上で、その対処法・ケアを計画できる。
\item
  患者の苦痛や不安感に配慮しながら、就学・就労、育児・介護等との両立支援を含め患者と家族に対して誠実で適切な支援を計画できる。
\end{enumerate}

\hypertarget{ux5730ux57dfux306eux8996ux70b9ux3068ux30a2ux30d7ux30edux30fcux30c1}{%
\subsection{地域の視点とアプローチ}\label{ux5730ux57dfux306eux8996ux70b9ux3068ux30a2ux30d7ux30edux30fcux30c1}}

地域の実情に応じた医療・介護・保健・福祉の現状及び課題を理解し、医療の基本としてのプライマリ・ケアの実践、ヘルスケアシステムの質の向上に貢献するための能力を獲得する。

\hypertarget{ux30d7ux30e9ux30a4ux30deux30eaux30b1ux30a2ux306bux304aux3051ux308bux57faux672cux6982ux5ff5}{%
\subsubsection{プライマリ・ケアにおける基本概念}\label{ux30d7ux30e9ux30a4ux30deux30eaux30b1ux30a2ux306bux304aux3051ux308bux57faux672cux6982ux5ff5}}

\begin{enumerate}
\def\labelenumi{\arabic{enumi}.}
\tightlist
\item
  地域の健康格差を理解し、医療へのアクセス障害等のヘルスケアシステム上の課題を適切に判断できる。
\item
  患者の所属する地域や文化的な背景が健康に関連することを理解している。
\end{enumerate}

\hypertarget{ux5730ux57dfux306bux304aux3051ux308bux30d7ux30e9ux30a4ux30deux30eaux30b1ux30a2}{%
\subsubsection{地域におけるプライマリ・ケア}\label{ux5730ux57dfux306bux304aux3051ux308bux30d7ux30e9ux30a4ux30deux30eaux30b1ux30a2}}

\begin{enumerate}
\def\labelenumi{\arabic{enumi}.}
\tightlist
\item
  地域(都会・郊外・へき地・離島を含む)の実情に応じた医療と医師の偏在(地域、診療科及び臨床・非臨床)の現状を理解している。
\item
  地域の医療体制や診療機関の規模・役割に応じて、医療者として柔軟に対応できる。
\item
  患者の居住する地域における各疾患の罹患率、有病率等の指標を用い、臨床推論で活用できる。
\item
  地域の量的指標(人口構成等)や質的情報(地理的・歴史的・経済的・文化的背景)を収集し、地域の健康課題を説明できる。
\item
  地域の住民や専門職と協働した地域の健康増進活動の意義を理解している。
\end{enumerate}

\hypertarget{ux533bux7642ux8cc7ux6e90ux306bux5fdcux3058ux305fux30d7ux30e9ux30a4ux30deux30eaux30b1ux30a2}{%
\subsubsection{医療資源に応じたプライマリ・ケア}\label{ux533bux7642ux8cc7ux6e90ux306bux5fdcux3058ux305fux30d7ux30e9ux30a4ux30deux30eaux30b1ux30a2}}

\begin{enumerate}
\def\labelenumi{\arabic{enumi}.}
\tightlist
\item
  地域の人的・物的資源に応じた医療・サービスを提案できる。
\item
  離島・へき地や医師不足地域等の医療資源が限られた状況での医療提供体制及び介護・保健・福祉の体制を理解している。
\end{enumerate}

\hypertarget{ux5728ux5b85ux306bux304aux3051ux308bux30d7ux30e9ux30a4ux30deux30eaux30b1ux30a2}{%
\subsubsection{在宅におけるプライマリ・ケア}\label{ux5728ux5b85ux306bux304aux3051ux308bux30d7ux30e9ux30a4ux30deux30eaux30b1ux30a2}}

\begin{enumerate}
\def\labelenumi{\arabic{enumi}.}
\tightlist
\item
  在宅医療の現状と適応を踏まえて、その必要性や課題を理解している。
\item
  在宅における人生の最終段階における医療、看取りの在り方と課題を理解している。
\end{enumerate}

\hypertarget{ux4ebaux751fux306eux8996ux70b9ux3068ux30a2ux30d7ux30edux30fcux30c1}{%
\subsection{人生の視点とアプローチ}\label{ux4ebaux751fux306eux8996ux70b9ux3068ux30a2ux30d7ux30edux30fcux30c1}}

患者・生活者の成長、発達、老化、死のプロセスを踏まえ、経時的に患者・家族・生活者に起こり得る精神・社会・医学的な問題に関与できる。

\hypertarget{ux4ebaux751fux306eux30d7ux30edux30bbux30b9}{%
\subsubsection{人生のプロセス}\label{ux4ebaux751fux306eux30d7ux30edux30bbux30b9}}

\begin{enumerate}
\def\labelenumi{\arabic{enumi}.}
\tightlist
\item
  ライフサイクル(胎児期・新生児期・乳幼児期・学童期・思春期・青年期・成人期・壮年期・老年期・終末期)の視点で、患者の課題を検討できる。
\item
  ライフステージやライフイベントの視点で、健康管理と環境・生活習慣改善を検討できる。
\item
  家族ライフサイクル・家族成員間関係・家族システムの視点で、患者・家族間の問題(虐待・ネグレクト等)を指摘できる。
\end{enumerate}

\hypertarget{ux5c0fux5150ux671fux5168ux822c}{%
\subsubsection{小児期全般}\label{ux5c0fux5150ux671fux5168ux822c}}

\begin{enumerate}
\def\labelenumi{\arabic{enumi}.}
\tightlist
\item
  小児期の生理機能の発達について理解している。
\item
  小児期の正常な精神運動発達について理解している。
\item
  小児期の愛着形成や保育法・栄養法について理解している。
\item
  小児期の栄養面での特性や食育について理解している。
\item
  小児期の免疫発達と感染症の関係について理解している。
\item
  小児期から成人期への医療の移行について、現状と課題を理解している。
\end{enumerate}

\hypertarget{ux80ceux5150ux671fux65b0ux751fux5150ux671fux4e73ux5e7cux5150ux671f}{%
\subsubsection{胎児期・新生児期・乳幼児期}\label{ux80ceux5150ux671fux65b0ux751fux5150ux671fux4e73ux5e7cux5150ux671f}}

\begin{enumerate}
\def\labelenumi{\arabic{enumi}.}
\tightlist
\item
  胎児の循環・呼吸の生理的特徴と出生時の変化について理解している。
\item
  新生児・乳幼児の生理的特徴について理解している。
\end{enumerate}

\hypertarget{ux5b66ux7ae5ux671fux601dux6625ux671fux9752ux5e74ux671fux6210ux4ebaux671f}{%
\subsubsection{学童期・思春期・青年期・成人期}\label{ux5b66ux7ae5ux671fux601dux6625ux671fux9752ux5e74ux671fux6210ux4ebaux671f}}

\begin{enumerate}
\def\labelenumi{\arabic{enumi}.}
\tightlist
\item
  思春期発現の機序と性徴について理解している。
\item
  学童期・思春期と関連する課題(学業、友達等に関わる課題)について理解している。
\item
  思春期・青年期と関連する課題(生殖、いのち等に関わる課題)について理解している。
\item
  成人期と関連する課題(メンタルヘルス、仕事、不妊等に関わる課題)について理解している。
\end{enumerate}

\hypertarget{ux58eeux5e74ux671fux8001ux5e74ux671f}{%
\subsubsection{壮年期・老年期}\label{ux58eeux5e74ux671fux8001ux5e74ux671f}}

\begin{enumerate}
\def\labelenumi{\arabic{enumi}.}
\tightlist
\item
  加齢に伴う臓器や身体機能の変化、それに伴う生理的変化について理解している。
\item
  高齢者総合機能評価(CGA)を実施できる。
\item
  老年症候群(歩行障害・転倒、認知機能障害、排泄障害、栄養障害、摂食・嚥下障害等)について理解している。
\item
  フレイル、サルコペニア、ロコモティブ・シンドロームの概念、その対処法、予防について理解している。
\item
  国際生活機能分類(ICF)について理解している。
\item
  高齢者の栄養マネジメントについて理解している。
\item
  日常生活動作(ADL)※に応じた介護と環境整備について理解している。
\end{enumerate}

\hypertarget{ux7d42ux672bux671f}{%
\subsubsection{終末期}\label{ux7d42ux672bux671f}}

\begin{enumerate}
\def\labelenumi{\arabic{enumi}.}
\tightlist
\item
  死の概念と定義や生物学的な個体の死について理解している。
\item
  死に至る身体と心の過程の知識を活用して、患者や家族がもつ死生観を配慮できる。
\item
  人生の最終段階における医療(エンド・オブ・ライフ・ケア)について理解している。
\item
  小児の終末期の特殊性について理解している。
\item
  意思決定(ACP)、事前指示書遵守(AD)、延命治療、DNAR、尊厳死と安楽死、治療の中止と差し控え等について理解している。
\item
  悲嘆のケア(グリーフケア)について理解している。
\end{enumerate}

\hypertarget{ux793eux4f1aux306eux8996ux70b9ux3068ux30a2ux30d7ux30edux30fcux30c1}{%
\subsection{社会の視点とアプローチ}\label{ux793eux4f1aux306eux8996ux70b9ux3068ux30a2ux30d7ux30edux30fcux30c1}}

文化的・社会的文脈のなかで生成される健康観や人々の言動・関係性を理解し、文化人類学・社会学(主に医療人類学・医療社会学)の視点から、それを臨床実践に活用することができる。

\hypertarget{ux533bux5b66ux7684ux6587ux5316ux7684ux793eux4f1aux7684ux6587ux8108ux306bux304aux3051ux308bux5065ux5eb7}{%
\subsubsection{医学的・文化的・社会的文脈における健康}\label{ux533bux5b66ux7684ux6587ux5316ux7684ux793eux4f1aux7684ux6587ux8108ux306bux304aux3051ux308bux5065ux5eb7}}

\begin{enumerate}
\def\labelenumi{\arabic{enumi}.}
\tightlist
\item
  患者の健康観や病いに対する価値観を理解したうえで、健康に関する知識※を活用し、健康問題に対する包括的アプローチを実践できる。
\item
  患者が受療に至るまでにどのような過程があるかを生活者の視点から説明できる。
\item
  栄養やエネルギー代謝に関する知識や統計情報をもとに個人の栄養状態を評価でき、本人や家族の生活や価値観もふまえたうえで食生活の支援を計画できる。
\item
  身体活動・運動の知識や統計情報をもとに個人の生活活動を評価でき、本人や家族の生活や価値観も踏まえたうえで活動や運動の支援を計画できる。
\item
  休養や心の健康について知識や統計情報をもとに評価し、本人や家族の生活や価値観も踏まえたうえで支援を計画できる。
\item
  喫煙や飲酒に関して、喫煙や飲酒による健康影響の知識や統計情報をもとに、本人や家族の生活や価値観を踏まえた評価や支援を計画できる。
\item
  健康の社会的決定要因とアドボカシーについて理解している。
\end{enumerate}

\hypertarget{ux793eux4f1aux79d1ux5b66}{%
\subsubsection{社会科学}\label{ux793eux4f1aux79d1ux5b66}}

\begin{enumerate}
\def\labelenumi{\arabic{enumi}.}
\tightlist
\item
  人の言動の意味をその人の人生史・生活史や社会関係の文脈の中において検討できる。
\item
  文化人類学・社会学(主に医療人類学・医療社会学)の視点で、患者やその家族と生活環境・地域社会・医療機関等との関係について説明できる。
\item
  文化人類学・社会学(主に医療人類学・医療社会学)の理論や概念を用いて、患者の判断や行動に関わる諸事象を説明できる。
\end{enumerate}

\newpage

\hypertarget{ux751fux6dafux306bux308fux305fux3063ux3066ux5171ux306bux5b66ux3076ux59ffux52e2}{%
\section{生涯にわたって共に学ぶ姿勢}\label{ux751fux6dafux306bux308fux305fux3063ux3066ux5171ux306bux5b66ux3076ux59ffux52e2}}

絶えず省察し、他の医師・医療者と共に研鑽しながら、安全で質の高い医療を実践するために生涯にわたって自律的に学び続け、また積極的に教育に関わっていく。

\hypertarget{ux533bux7642ux8005ux6559ux80b2}{%
\subsection{医療者教育}\label{ux533bux7642ux8005ux6559ux80b2}}

医師・医学生に限らず同僚や後輩を含む医療者への教育に貢献する。

\hypertarget{ux533bux7642ux8005ux6559ux80b2ux306eux5b9fux8df5}{%
\subsubsection{医療者教育の実践}\label{ux533bux7642ux8005ux6559ux80b2ux306eux5b9fux8df5}}

\begin{enumerate}
\def\labelenumi{\arabic{enumi}.}
\tightlist
\item
  後輩や同僚等と協働して学修できる。
\item
  後輩や同僚等に対して、適切にフィードバックできる。
\item
  成人学習理論を活用し、後輩や同僚等に対して教育を実践できる。
\end{enumerate}

\hypertarget{ux751fux6dafux5b66ux7fd2}{%
\subsection{生涯学習}\label{ux751fux6dafux5b66ux7fd2}}

生涯学び続ける価値観を形成する。

\hypertarget{ux751fux6dafux5b66ux7fd2ux306eux5b9fux8df5}{%
\subsubsection{生涯学習の実践}\label{ux751fux6dafux5b66ux7fd2ux306eux5b9fux8df5}}

\begin{enumerate}
\def\labelenumi{\arabic{enumi}.}
\tightlist
\item
  医学知識が常に変わりゆくことを認識し、現時点での最善の医学情報にアクセスできる。
\item
  学修・経験したことを省察し、自己の課題を明確にすることができる。
\end{enumerate}

\hypertarget{ux30adux30e3ux30eaux30a2ux958bux767a}{%
\subsubsection{キャリア開発}\label{ux30adux30e3ux30eaux30a2ux958bux767a}}

\begin{enumerate}
\def\labelenumi{\arabic{enumi}.}
\tightlist
\item
  自身の職業観を涵養しながら、主体的にキャリアを構築していくことができる。
\end{enumerate}

\newpage

\hypertarget{ux79d1ux5b66ux7684ux63a2ux7a76}{%
\section{科学的探究}\label{ux79d1ux5b66ux7684ux63a2ux7a76}}

医学・医療の発展のための医学研究の重要性を理解し、科学的思考を身に付けながら、学術・研究活動に関与して医学を創造する。

\hypertarget{ux30eaux30b5ux30fcux30c1ux30deux30a4ux30f3ux30c9}{%
\subsection{リサーチマインド}\label{ux30eaux30b5ux30fcux30c1ux30deux30a4ux30f3ux30c9}}

\hypertarget{ux80fdux52d5ux7684ux59ffux52e2}{%
\subsubsection{能動的姿勢}\label{ux80fdux52d5ux7684ux59ffux52e2}}

\begin{enumerate}
\def\labelenumi{\arabic{enumi}.}
\tightlist
\item
  常識を疑う。
\end{enumerate}

\hypertarget{ux80fdux52d5ux7684ux59ffux52e2-1}{%
\subsubsection{能動的姿勢}\label{ux80fdux52d5ux7684ux59ffux52e2-1}}

\begin{enumerate}
\def\labelenumi{\arabic{enumi}.}
\tightlist
\item
  何事にも知的好奇心を持って取り組む。
\end{enumerate}

\hypertarget{ux63a2ux7a76ux5fc3}{%
\subsubsection{探究心}\label{ux63a2ux7a76ux5fc3}}

\begin{enumerate}
\def\labelenumi{\arabic{enumi}.}
\tightlist
\item
  最先端の研究に刺激を受ける。
\item
  ロールモデルとしての研究者の生き方に触れる。
\end{enumerate}

\hypertarget{ux65e2ux77e5ux306eux77e5}{%
\subsection{既知の知}\label{ux65e2ux77e5ux306eux77e5}}

\hypertarget{ux533bux5b66ux3068ux533bux7642}{%
\subsubsection{医学と医療}\label{ux533bux5b66ux3068ux533bux7642}}

\begin{enumerate}
\def\labelenumi{\arabic{enumi}.}
\tightlist
\item
  医療の実践が基礎医学・臨床医学・社会医学の研究に基づいていることを理解する。
\end{enumerate}

\hypertarget{ux8ad6ux6587ux8aadux89e3}{%
\subsubsection{論文読解}\label{ux8ad6ux6587ux8aadux89e3}}

\begin{enumerate}
\def\labelenumi{\arabic{enumi}.}
\tightlist
\item
  医学論文(英語)を読んで内容を理解する。
\end{enumerate}

\hypertarget{ux7814ux7a76ux306eux5b9fux65bd}{%
\subsection{研究の実施}\label{ux7814ux7a76ux306eux5b9fux65bd}}

\hypertarget{ux554fux3044}{%
\subsubsection{問い}\label{ux554fux3044}}

\begin{enumerate}
\def\labelenumi{\arabic{enumi}.}
\tightlist
\item
  自身の関心を問いにすることができる。
\end{enumerate}

\hypertarget{ux7814ux7a76ux8a08ux753b}{%
\subsubsection{研究計画}\label{ux7814ux7a76ux8a08ux753b}}

\begin{enumerate}
\def\labelenumi{\arabic{enumi}.}
\tightlist
\item
  研究計画の素案を作ることができる。
\end{enumerate}

\hypertarget{ux7814ux7a76ux624bux6cd5}{%
\subsubsection{研究手法}\label{ux7814ux7a76ux624bux6cd5}}

\begin{enumerate}
\def\labelenumi{\arabic{enumi}.}
\tightlist
\item
  基礎医学の実習から基本的な実験手技を体得する。
\item
  社会医学(行動科学を含む)の実習から基本的な研究方法論を体得する。
\item
  研究室配属等で医学研究の基本的な研究手法を習得する。
\end{enumerate}

\hypertarget{ux7814ux7a76ux7d50ux679c}{%
\subsubsection{研究結果}\label{ux7814ux7a76ux7d50ux679c}}

\begin{enumerate}
\def\labelenumi{\arabic{enumi}.}
\tightlist
\item
  研究データを適切に記録、管理することができる。
\end{enumerate}

\hypertarget{ux7814ux7a76ux306eux767aux4fe1}{%
\subsection{研究の発信}\label{ux7814ux7a76ux306eux767aux4fe1}}

\hypertarget{ux7814ux7a76ux767aux8868}{%
\subsubsection{研究発表}\label{ux7814ux7a76ux767aux8868}}

\begin{enumerate}
\def\labelenumi{\arabic{enumi}.}
\tightlist
\item
  自身の行った研究内容を論文や報告書・学会発表などの形にまとめる。
\item
  発表の場に応じて読者・聴衆にわかりやすく研究内容をプレゼンテーションできる。
\item
  他の研究者の発表に対して質問や意見を述べることができる。
\end{enumerate}

\hypertarget{ux7814ux7a76ux502bux7406}{%
\subsection{研究倫理}\label{ux7814ux7a76ux502bux7406}}

\hypertarget{ux9069ux5207ux306aux7814ux7a76ux9042ux884c}{%
\subsubsection{適切な研究遂行}\label{ux9069ux5207ux306aux7814ux7a76ux9042ux884c}}

\begin{enumerate}
\def\labelenumi{\arabic{enumi}.}
\tightlist
\item
  捏造、改ざん、盗用などを含め研究不正の類型を説明することができ、研究不正を行わない。
\end{enumerate}

\hypertarget{ux5bfeux8c61ux8005ux306eux4fddux8b77}{%
\subsubsection{対象者の保護}\label{ux5bfeux8c61ux8005ux306eux4fddux8b77}}

\begin{enumerate}
\def\labelenumi{\arabic{enumi}.}
\tightlist
\item
  人を対象とした研究(治験、特定臨床研究を含む)に関するルールの概要を説明でき、遵守する。
\item
  利益相反や動物・遺伝子組み換え実験に関するルールを概説でき、遵守する。
\end{enumerate}

\newpage

\hypertarget{ux5c02ux9580ux77e5ux8b58ux306bux57faux3065ux3044ux305fux554fux984cux89e3ux6c7aux80fdux529b}{%
\section{専門知識に基づいた問題解決能力}\label{ux5c02ux9580ux77e5ux8b58ux306bux57faux3065ux3044ux305fux554fux984cux89e3ux6c7aux80fdux529b}}

医学および関連する学問分野の知識を身に付け、根拠に基づいた医療を基盤に、経験も踏まえながら、患者の抱える問題を解決する。

\hypertarget{ux57faux790eux533bux5b66}{%
\subsection{基礎医学}\label{ux57faux790eux533bux5b66}}

\hypertarget{ux751fux547dux73feux8c61ux306eux79d1ux5b66}{%
\subsubsection{生命現象の科学}\label{ux751fux547dux73feux8c61ux306eux79d1ux5b66}}

\begin{enumerate}
\def\labelenumi{\arabic{enumi}.}
\tightlist
\item
  細胞の観察法について理解している。
\item
  細胞の全体像を図示できる。
\item
  核とリボソーム、小胞体、ゴルジ体、リソソーム等の細胞内膜系、ミトコンドリア、葉緑体、細胞骨格の種類とその構造と機能について理解している。
\item
  細胞膜の構造と機能、細胞同士の接着と結合様式について理解している。
\item
  原核細胞と真核細胞の特徴について理解している。
\item
  メンデルの法則、ミトコンドリア遺伝、インプリンティング及び多因子遺伝について理解している。
\item
  遺伝型と表現型の関係について理解している。
\item
  染色体の構造を理解し、ゲノムと染色体及び遺伝子の構造と関係性、体細胞分裂及び減数分裂における染色体の挙動について理解している。
\item
  DNAの複製と修復、DNAからRNAへの転写、タンパク質合成に至る翻訳を含む遺伝情報の発現及び調節(セントラルドグマ)について理解している。
\item
  染色体分析・DNA配列決定を含むゲノム解析技術について理解している。
\item
  ゲノム編集技術とその応用について理解している。
\item
  進化の基本的な考え方について理解している。
\item
  生物種とその系統関係について理解している。
\item
  アミノ酸配列や塩基配列の比較による分子系統樹について理解している。
\end{enumerate}

\hypertarget{ux500bux4f53ux306eux69cbux6210ux3068ux6a5fux80fd}{%
\subsubsection{個体の構成と機能}\label{ux500bux4f53ux306eux69cbux6210ux3068ux6a5fux80fd}}

\begin{enumerate}
\def\labelenumi{\arabic{enumi}.}
\tightlist
\item
  細胞内液・外液のイオン組成、および浸透圧と(静止)膜電位の形成機構について理解している。
\item
  細胞膜のイオンチャネル、ポンプ、および膜を介する物質の能動・受動輸送過程について理解している。
\item
  活動電位の発生機構と伝導、シナプス(神経筋接合部を含む)の形態とシナプス伝達の機能(興奮性、抑制性)と可塑性について理解している。
\item
  情報伝達の種類と機能について理解している。
\item
  受容体の種類・細胞内局在・機能、受容体による細胞内シグナル伝達過程について理解している。
\item
  液性因子による細胞間情報伝達(自己分泌、傍分泌、内分泌)について理解している。
\item
  細胞骨格を構成するタンパク質とその機能、アクチンフィラメント系による細胞運動について理解している。
\item
  細胞膜を介する分泌と吸収の過程と細胞内輸送システム、微小管の役割や機能について理解している。
\item
  軸索輸送、軸索の変性と再生について理解している。
\item
  上皮組織と腺の構造と機能について理解している。
\item
  支持組織を構成する細胞と細胞間質(線維成分と基質)について理解している。
\item
  血管とリンパ管の微細構造と機能について理解している。
\item
  神経組織の微細構造について理解している。
\item
  筋組織について、骨格筋、心筋、平滑筋の構造と機能を対比して理解している。
\item
  組織の再生の機序について理解している。
\item
  位置関係を方向用語(上下、前後、内・外側、浅深、頭・尾側、背・腹側、近位・遠位、内転・外転)で理解している。
\item
  刺激に対する感覚受容の種類と機序について理解している。
\item
  反射について理解している。
\item
  生体の恒常性維持と適応、恒常性維持のための調節機構(フィードバック調節)について理解している。
\item
  生体機能や体内環境のリズム性変化について理解している。
\item
  生体の恒常性維持における常在菌・腸内細菌と宿主との相互作用の重要性について理解している。
\item
  配偶子の形成から出生に至る一連の経過と胚形成の全体像、胚内体腔の形成過程について理解している。
\item
  体節の形成と分化、鰓弓・鰓嚢の分化と頭・頸部と顔面・口腔の形成過程について理解している。
\item
  体幹と四肢の骨格と筋、心血管系、泌尿生殖器系各器官の形成過程について理解している。
\item
  消化・呼吸器系各器官の形成過程について理解している。
\item
  神経管の分化と脳、脊髄、視覚器、平衡聴覚器と自律神経系、皮膚の形成過程について理解している。
\item
  酵素の機能と調節について理解している。
\item
  糖質の構造、代謝と調節(解糖、クエン酸回路、電子伝達系と酸化的リン酸化、グリコーゲン代謝、糖新生、五炭糖リン酸回路)、生理的意義について理解している。
\item
  タンパク質の構造、代謝と調節、生理的意義、ならびに主要なアミノ酸の代謝と尿素回路について理解している。
\item
  脂質の構造、代謝と調節、生理的意義、ならびに脂質の輸送(リポタンパク質)について理解している。
\item
  ヘム・ポルフィリンの代謝について理解している。
\item
  ヌクレオチドの合成・異化・再利用経路について理解している。
\item
  酸化ストレス(フリーラジカル、活性酸素)について理解している。
\item
  ビタミン、微量元素の種類と作用について理解している。
\item
  栄養素の相互変換とエネルギー代謝(エネルギーの定義、食品中のエネルギー値、エネルギー消費量、推定エネルギー必要量)について理解している。
\item
  空腹(飢餓)時、食後(過食時)と運動時における代謝について理解している。
\item
  複合糖質、複合脂質について理解している。
\end{enumerate}

\hypertarget{ux500bux4f53ux306eux53cdux5fdc}{%
\subsubsection{個体の反応}\label{ux500bux4f53ux306eux53cdux5fdc}}

\begin{enumerate}
\def\labelenumi{\arabic{enumi}.}
\tightlist
\item
  原核生物としての細菌の構造と機能の違いについて真核生物と比較して理解している。
\item
  細菌の感染経路を分類し、細菌が疾病を引き起こす機序について理解している。
\item
  細菌の産生するタンパク質性毒素、非タンパク質性毒素の作用機序について理解している。
\item
  主なグラム陽性球菌、グラム陽性桿菌、グラム陰性球菌、グラム陰性桿菌の細菌学的特徴、リスク因子、感染経路と病態を説明し、それが引き起こす疾患を列挙できる。
\item
  抗酸菌の細菌学的特徴、リスク因子、感染経路と病態を説明し、それが引き起こす疾患を列挙できる。
\item
  らせん状細菌、マイコプラズマ、リケッチア、クラミジアの微生物学的特徴とそれが引き起こす疾患を列挙できる。
\item
  生体各部の細菌叢(マイクロバイオーム)の構成菌、細菌叢の機能について理解している。
\item
  ウイルス粒子の構造と性状によりウイルスを分類できる。
\item
  ウイルス感染の種特異性、組織特異性と吸着、侵入、複製、成熟と放出の各過程、ウイルス感染細胞に起こる変化について理解している。
\item
  主なDNAウイルス(2本鎖DNA、1本鎖DNA、不完全1本鎖DNAウイルス)の特徴、リスク因子、感染経路と病態を説明し、これらのウイルスが引き起こす疾患名を列挙できる。
\item
  主なRNAウイルス(2本鎖RNA、1本鎖(+)RNA、1本鎖(−)RNAウイルス)の特徴、リスク因子、感染経路と病態を説明し、これらのウイルスが引き起こす疾患名を列挙できる。
\item
  逆転写酵素を持つRNAウイルスの特徴、リスク因子、感染経路と病態を説明し、これらのウイルスが引き起こす疾患名を列挙できる。
\item
  真菌(接合菌、子嚢菌、担子菌、不完全菌)の微生物学的特徴、リスク因子、感染経路と病態を説明し、それが引き起こす疾患を列挙できる。
\item
  原虫類(寄生虫)の分類及び形態学的特徴、生活史、リスク因子、感染経路と病態、感染疫学的意義、寄生虫感染宿主の生体防御の特徴について理解している。
\item
  微生物の特性に応じた治療薬の作用機序について理解している。
\item
  微生物感染症に対するワクチンの原理、種類とそれに対する問題点について理解している。
\item
  人獣共通感染症の原因となる微生物について、その生活史、リスク因子、感染経路と病態、感染疫学的意義について理解している。
\item
  昆虫媒介性感染症の原因となる微生物について、その生活史、リスク因子、感染経路と病態、感染疫学的意義について理解している。
\item
  自然免疫と獲得免疫の抗原認識や作用の違いを特異性、多様性、寛容、記憶の観点から理解している。
\item
  免疫反応に関わる組織と細胞について理解している。
\item
  補体及び自然免疫細胞が病原体により活性化し、炎症をおこす仕組みについて理解している。
\item
  主要組織適合遺伝子複合体(MHC)クラスIとクラスIIの基本構造と機能、抗原提示によるT細胞活性化の仕組みについて理解している。
\item
  免疫グロブリンとT細胞抗原レセプターの構造と反応様式、免疫グロブリンとT細胞抗原レセプター遺伝子の構造と遺伝子再構成に基づき、多様性獲得の機構について理解している。
\item
  ヘルパーT細胞(Th1 cell、Th2 cell、Th17
  cell)、細胞傷害性T細胞(CTL)、制御性T細胞(Treg)それぞれが担当する生体防御反応について理解している。
\item
  B細胞の活性化による抗体産生の機構、および抗体の役割を理解している。
\item
  自然免疫系を構成する主な細胞とそれらの活性化機構(TLR等)を理解している。
\item
  抗原提示細胞の種類と役割、抗原提示機構について理解している。
\item
  ウイルス、細菌、真菌と寄生虫に対する免疫応答の特徴について理解している。
\item
  原発性免疫不全症と後天性免疫不全症候群(AIDS)について理解している。
\item
  免疫寛容の維持機構とその破綻による自己免疫疾患の発症について理解している。
\item
  アレルギー発症の機序について理解している。
\item
  癌免疫に関わる細胞性機序について理解している。
\item
  生体(あるいは生体群)の薬物・毒物反応性について、用量反応曲線を描き理解している。
\item
  薬物の受容体結合と薬理作用との関連性及び作動薬・拮抗薬について理解している。
\item
  薬物の投与経路とその吸収、分布、代謝、排泄機構について理解している。
\item
  薬物の有害作用、薬物間相互作用について理解している。
\item
  薬物開発のプロセスと臨床試験における薬物の評価をについて理解している。
\end{enumerate}

\hypertarget{ux75c5ux56e0ux3068ux75c5ux614b}{%
\subsubsection{病因と病態}\label{ux75c5ux56e0ux3068ux75c5ux614b}}

\begin{enumerate}
\def\labelenumi{\arabic{enumi}.}
\tightlist
\item
  ゲノムの多様性に基づく個体の多様性について理解している。
\item
  単一遺伝子疾患、染色体異常による疾患、ミトコンドリア遺伝子の変異による疾患を挙げ、遺伝様式を含め理解している。
\item
  多因子疾患における遺伝要因と環境要因の関係について理解している。
\item
  薬剤の有効性や安全性とゲノムの多様性との関係について理解している。
\item
  ネクローシスとアポトーシスの違いを含め、細胞傷害・変性と細胞死の多様性、病因と意義について理解している。
\item
  細胞傷害・変性と細胞死の細胞と組織の形態的変化の特徴について理解している。
\item
  糖代謝異常の病態について理解している。
\item
  タンパク質・アミノ酸代謝異常の病態について理解している。
\item
  脂質代謝異常の病態について理解している。
\item
  核酸・ヌクレオチド代謝異常の病態について理解している。
\item
  ビタミン、微量元素の代謝異常の病態について理解している。
\item
  メタボリックシンドロームの病態について理解している。
\item
  血行障害(阻血、低酸素血、充血、うっ血、出血)の違いとそれぞれの病因と病態、梗塞(血栓、塞栓)の種類と病態について理解している。
\item
  血圧異常(高血圧、低血圧)について理解している。
\item
  臓器不全(多臓器不全、サイトカインストーム、播種性血管内凝固症候群)について理解している。
\item
  炎症の定義について理解している。
\item
  炎症の分類、組織形態学的変化と経時的変化(局所的変化と全身的変化)について理解している。
\item
  炎症組織の治癒過程について理解している。
\item
  炎症とメタボリックシンドローム、動脈硬化、腫瘍、老化への関わりについて理解している。
\item
  自律性の増殖と、良性腫瘍と悪性腫瘍の違いについて理解している。
\item
  癌の原因や遺伝子変化について理解している。
\item
  腫瘍の分類、分化度、グレード、ステージについて理解している。
\item
  用語(異形成、上皮内癌、進行癌、早期癌、異型性、多形性等)について理解している。
\item
  癌の病理診断と治療の関わりについて理解している。
\item
  癌の転移について理解している。
\item
  癌の免疫系による排除機構について理解している。
\end{enumerate}

\hypertarget{ux4ebaux4f53ux5404ux5668ux5b98ux306eux6b63ux5e38ux69cbux9020ux3068ux6a5fux80fdux75c5ux614bux8a3aux65adux6cbbux7642}{%
\subsection{人体各器官の正常構造と機能、病態、診断、治療}\label{ux4ebaux4f53ux5404ux5668ux5b98ux306eux6b63ux5e38ux69cbux9020ux3068ux6a5fux80fdux75c5ux614bux8a3aux65adux6cbbux7642}}

\hypertarget{ux8840ux6db2ux9020ux8840ux5668ux30eaux30f3ux30d1ux7cfb}{%
\subsubsection{血液・造血器・リンパ系}\label{ux8840ux6db2ux9020ux8840ux5668ux30eaux30f3ux30d1ux7cfb}}

\begin{enumerate}
\def\labelenumi{\arabic{enumi}.}
\tightlist
\item
  血液・造血器・リンパ系の構造と機能について基本的事項について理解している。
\item
  血液・造血器・リンパ系でみられる症候について理解している。
\item
  血液・造血器・リンパ系で行う検査方法について基本的事項について理解している。
\item
  血液・造血器・リンパ系疾患に特異的な治療法について基本的事項について理解している。
\item
  血液・造血器・リンパ系の疾患・病態について病因、疫学、症候、主な検査・診断、治療法について理解している。
\end{enumerate}

\hypertarget{ux795eux7d4cux7cfb}{%
\subsubsection{神経系}\label{ux795eux7d4cux7cfb}}

\begin{enumerate}
\def\labelenumi{\arabic{enumi}.}
\tightlist
\item
  神経系の構造と機能について基本的事項について理解している。
\item
  神経系でみられる症候について理解している。
\item
  神経系で行う検査方法について基本的事項について理解している。
\item
  神経系疾患に特異的な治療法について基本的事項について理解している。
\item
  神経系の疾患・病態について病因、疫学、症候、主な検査・診断、治療法について理解している。
\end{enumerate}

\hypertarget{ux76aeux819aux7cfb}{%
\subsubsection{皮膚系}\label{ux76aeux819aux7cfb}}

\begin{enumerate}
\def\labelenumi{\arabic{enumi}.}
\tightlist
\item
  皮膚系の構造と機能について基本的事項について理解している。
\item
  皮膚系でみられる症候について理解している。
\item
  皮膚系で行う検査方法について基本的事項について理解している。
\item
  皮膚系疾患に特異的な治療法について基本的事項について理解している。
\item
  皮膚系の疾患・病態について病因、疫学、症候、主な検査・診断、治療法について理解している。
\end{enumerate}

\hypertarget{ux904bux52d5ux5668ux7b4bux9aa8ux683cux7cfb}{%
\subsubsection{運動器(筋骨格)系}\label{ux904bux52d5ux5668ux7b4bux9aa8ux683cux7cfb}}

\begin{enumerate}
\def\labelenumi{\arabic{enumi}.}
\tightlist
\item
  運動器(筋骨格)系の構造と機能について基本的事項について理解している。
\item
  運動器(筋骨格)系でみられる症候について理解している。
\item
  運動器(筋骨格)系で行う検査方法について基本的事項について理解している。
\item
  運動器(筋骨格)系疾患に特異的な治療法について基本的事項について理解している。
\item
  運動器(筋骨格)系の疾患・病態について病因、疫学、症候、主な検査・診断、治療法について理解している。
\end{enumerate}

\hypertarget{ux5faaux74b0ux5668ux7cfb}{%
\subsubsection{循環器系}\label{ux5faaux74b0ux5668ux7cfb}}

\begin{enumerate}
\def\labelenumi{\arabic{enumi}.}
\tightlist
\item
  循環器系の構造と機能について基本的事項について理解している。
\item
  循環器系でみられる症候について理解している。
\item
  循環器系で行う検査方法について基本的事項について理解している。
\item
  循環器系疾患に特異的な治療法について基本的事項について理解している。
\item
  循環器系の疾患・病態について病因、疫学、症候、主な検査・診断、治療法について理解している。
\end{enumerate}

\hypertarget{ux547cux5438ux5668ux7cfb}{%
\subsubsection{呼吸器系}\label{ux547cux5438ux5668ux7cfb}}

\begin{enumerate}
\def\labelenumi{\arabic{enumi}.}
\tightlist
\item
  呼吸器系の構造と機能について基本的事項について理解している。
\item
  呼吸器系でみられる症候について理解している。
\item
  呼吸器系で行う検査方法について基本的事項について理解している。
\item
  呼吸器系疾患に特異的な治療法について基本的事項について理解している。
\item
  呼吸器系の疾患・病態について病因、疫学、症候、主な検査・診断、治療法について理解している。
\end{enumerate}

\hypertarget{ux6d88ux5316ux5668ux7cfb}{%
\subsubsection{消化器系}\label{ux6d88ux5316ux5668ux7cfb}}

\begin{enumerate}
\def\labelenumi{\arabic{enumi}.}
\tightlist
\item
  消化器系の構造と機能について基本的事項について理解している。
\item
  消化器系でみられる症候について理解している。
\item
  消化器系で行う検査方法について基本的事項について理解している。
\item
  消化器系疾患に特異的な治療法について基本的事項について理解している。
\item
  消化器系の疾患・病態について病因、疫学、症候、主な検査・診断、治療法について理解している。
\end{enumerate}

\hypertarget{ux814eux5c3fux8defux7cfbux4f53ux6db2ux96fbux89e3ux8ceaux30d0ux30e9ux30f3ux30b9ux3092ux542bux3080}{%
\subsubsection{腎・尿路系(体液・電解質バランスを含む)}\label{ux814eux5c3fux8defux7cfbux4f53ux6db2ux96fbux89e3ux8ceaux30d0ux30e9ux30f3ux30b9ux3092ux542bux3080}}

\begin{enumerate}
\def\labelenumi{\arabic{enumi}.}
\tightlist
\item
  腎・尿路系の構造と機能について基本的事項について理解している。
\item
  腎・尿路系でみられる症候について理解している。
\item
  腎・尿路系で行う検査方法について基本的事項について理解している。
\item
  腎・尿路系疾患に特異的な治療法について基本的事項について理解している。
\item
  腎・尿路系の疾患・病態について病因、疫学、症候、主な検査・診断、治療法について理解している。
\end{enumerate}

\hypertarget{ux751fux6b96ux6a5fux80fd}{%
\subsubsection{生殖機能}\label{ux751fux6b96ux6a5fux80fd}}

\begin{enumerate}
\def\labelenumi{\arabic{enumi}.}
\tightlist
\item
  生殖機能の構造と機能について基本的事項について理解している。
\item
  生殖機能でみられる症候について理解している。
\item
  生殖機能で行う検査方法について基本的事項について理解している。
\item
  生殖機能に関する疾患に特異的な治療法について基本的事項について理解している。
\item
  生殖機能に関する疾患・病態について病因、疫学、症候、主な検査・診断、治療法について理解している。
\end{enumerate}

\hypertarget{ux598aux5a20ux3068ux5206ux5a29}{%
\subsubsection{妊娠と分娩}\label{ux598aux5a20ux3068ux5206ux5a29}}

\begin{enumerate}
\def\labelenumi{\arabic{enumi}.}
\tightlist
\item
  妊娠と分娩に関する構造と機能について基本的事項について理解している。
\item
  妊娠と分娩でみられる症候について理解している。
\item
  妊娠と分娩で行う検査方法について基本的事項について理解している。
\item
  妊娠と分娩に特異的な治療法について基本的事項について理解している。
\item
  妊娠と分娩に関する疾患・病態について病因、疫学、症候、主な検査・診断、治療法について理解している。
\end{enumerate}

\hypertarget{ux5c0fux5150}{%
\subsubsection{小児}\label{ux5c0fux5150}}

\begin{enumerate}
\def\labelenumi{\arabic{enumi}.}
\tightlist
\item
  小児にみられる症候について理解している。
\item
  小児で行う検査方法について基本的事項について理解している。
\item
  小児に特異的な治療法について基本的事項について理解している。
\item
  小児の疾患・病態について病因、疫学、症候、主な検査・診断、治療法について理解している。
\end{enumerate}

\hypertarget{ux4e73ux623f}{%
\subsubsection{乳房}\label{ux4e73ux623f}}

\begin{enumerate}
\def\labelenumi{\arabic{enumi}.}
\tightlist
\item
  乳房の構造と機能について基本的事項について理解している。
\item
  乳房でみられる症候について理解している。
\item
  乳房に関して行う検査方法について基本的事項について理解している。
\item
  乳房疾患に特異的な治療法について基本的事項について理解している。
\item
  乳房に関する疾患・病態について病因、疫学、症候、主な検査・診断、治療法について理解している。
\end{enumerate}

\hypertarget{ux5185ux5206ux6cccux6804ux990aux4ee3ux8b1dux7cfb}{%
\subsubsection{内分泌・栄養・代謝系}\label{ux5185ux5206ux6cccux6804ux990aux4ee3ux8b1dux7cfb}}

\begin{enumerate}
\def\labelenumi{\arabic{enumi}.}
\tightlist
\item
  内分泌・栄養・代謝系の構造と機能について基本的事項について理解している。
\item
  内分泌・栄養・代謝系でみられる症候について理解している。
\item
  内分泌・栄養・代謝系で行う検査方法について基本的事項について理解している。
\item
  内分泌・栄養・代謝系疾患に特異的な治療法について基本的事項について理解している。
\item
  内分泌・栄養・代謝系の疾患・病態について病因、疫学、症候、主な検査・診断、治療法について理解している。
\item
  メタボリックシンドロームの病態について理解している。
\end{enumerate}

\hypertarget{ux773cux8996ux899aux7cfb}{%
\subsubsection{眼・視覚系}\label{ux773cux8996ux899aux7cfb}}

\begin{enumerate}
\def\labelenumi{\arabic{enumi}.}
\tightlist
\item
  眼・視覚系の構造と機能について基本的事項について理解している。
\item
  眼・視覚系でみられる症候について理解している。
\item
  眼・視覚系で行う検査方法について基本的事項について理解している。
\item
  眼・視覚系疾患に特異的な治療法について基本的事項について理解している。
\item
  眼・視覚系の疾患・病態について病因、疫学、症候、主な検査・診断、治療法について理解している。
\end{enumerate}

\hypertarget{ux8033ux9f3bux54bdux5589ux53e3ux8154ux7cfb}{%
\subsubsection{耳鼻・咽喉・口腔系}\label{ux8033ux9f3bux54bdux5589ux53e3ux8154ux7cfb}}

\begin{enumerate}
\def\labelenumi{\arabic{enumi}.}
\tightlist
\item
  耳鼻・咽喉・口腔系の構造と機能について基本的事項について理解している。
\item
  耳鼻・咽喉・口腔系でみられる症候について理解している。
\item
  耳鼻・咽喉・口腔系で行う検査方法について基本的事項について理解している。
\item
  耳鼻・咽喉・口腔系疾患に特異的な治療法について基本的事項について理解している。
\item
  耳鼻・咽喉・口腔系の疾患・病態について病因、疫学、症候、主な検査・診断、治療法について理解している。
\end{enumerate}

\hypertarget{ux7cbeux795eux7cfb}{%
\subsubsection{精神系}\label{ux7cbeux795eux7cfb}}

\begin{enumerate}
\def\labelenumi{\arabic{enumi}.}
\tightlist
\item
  精神系の構造と機能について基本的事項について理解している。
\item
  精神系でみられる症候について理解している。
\item
  精神系で行う検査方法について基本的事項について理解している。
\item
  精神系疾患に特異的な治療法について基本的事項について理解している。
\item
  精神系の疾患・病態について病因、疫学、症候、主な検査・診断、治療法について理解している。
\end{enumerate}

\hypertarget{ux5168ux8eabux306bux53caux3076ux751fux7406ux7684ux5909ux5316ux75c5ux614bux8a3aux65adux6cbbux7642}{%
\subsection{全身に及ぶ生理的変化、病態、診断、治療}\label{ux5168ux8eabux306bux53caux3076ux751fux7406ux7684ux5909ux5316ux75c5ux614bux8a3aux65adux6cbbux7642}}

\hypertarget{ux907aux4f1dux533bux7642ux30b2ux30ceux30e0ux533bux7642}{%
\subsubsection{遺伝医療・ゲノム医療}\label{ux907aux4f1dux533bux7642ux30b2ux30ceux30e0ux533bux7642}}

\begin{enumerate}
\def\labelenumi{\arabic{enumi}.}
\tightlist
\item
  集団遺伝学の基礎としてハーディ・ワインベルグの法則について理解している。
\item
  家系図を作成、評価(ベイズの定理、リスク評価)できる。
\item
  生殖細胞系列変異と体細胞変異の違い、遺伝学的検査の目的と意義について理解している。
\item
  遺伝情報の特性(不変性、予見性、共有性)について理解している。
\item
  遺伝カウンセリングの意義と方法について理解している。
\item
  遺伝医療における倫理的・法的・社会的配慮について理解している。
\item
  遺伝医学関連情報にアクセスすることができる。
\item
  遺伝情報に基づく治療や予防をはじめとする適切な対処法について理解している。
\end{enumerate}

\hypertarget{ux514dux75abux30a2ux30ecux30ebux30aeux30fc}{%
\subsubsection{免疫・アレルギー}\label{ux514dux75abux30a2ux30ecux30ebux30aeux30fc}}

\begin{enumerate}
\def\labelenumi{\arabic{enumi}.}
\tightlist
\item
  膠原病、血管炎、リウマチ性疾患、アレルギー性疾患、自己免疫疾患の概念を区別して理解し、それぞれに含まれる疾患を列挙できる。
\item
  膠原病、血管炎、リウマチ性疾患、アレルギー性疾患、自己免疫疾患でみられる症状・症候について説明できる。
\item
  免疫血清学検査の原理と検査結果の臨床的意義について理解している。
\item
  膠原病、血管炎、リウマチ性疾患、アレルギー性疾患、自己免疫疾患に使用する治療薬について理解している。
\item
  膠原病、血管炎、リウマチ性疾患、アレルギー性疾患、自己免疫疾患の疾患・病態について病因、疫学、症候、主な検査・診断、治療法、合併症を説明できる。
\end{enumerate}

\hypertarget{ux611fux67d3ux75c7}{%
\subsubsection{感染症}\label{ux611fux67d3ux75c7}}

\begin{enumerate}
\def\labelenumi{\arabic{enumi}.}
\tightlist
\item
  代表的な市中感染症の原因微生物について理解している。
\item
  代表的な医療関連感染の原因微生物について理解している。
\item
  代表的な免疫不全患者の罹患しやすい微生物について理解している。
\item
  薬剤耐性
  (AMR)の現状と代表的な薬剤耐性菌(メチシリン耐性黄色ブドウ球菌(MRSA))について理解し、抗菌薬適正使用等の予防策について理解している。
\item
  患者(宿主)、感染臓器・部位、原因微生物の関係について理解している。
\item
  代表的な市中感染症のリスク因子、感染経路・侵入門戸、病態生理について理解している。
\item
  代表的な医療関連感染のリスク因子、感染経路・侵入門戸、病態生理を理解し理解している。
\item
  敗血症と血流感染の相違を理解し、病態について理解している。
\item
  新興感染症についてその感染経路を理解し、必要な感染対策について理解している。
\item
  医療面接と身体所見から感染臓器と原因微生物を想定し理解している。
\item
  医療面接と身体診察から想定した原因微生物の診断方法について理解している。
\item
  発熱患者への基本検査(血液培養2セット、尿検査・尿培養、胸部X線写真)について理解している。
\item
  抗菌薬投与の原則(抗菌薬投与前に培養検体を提出する、微生物と臓器による標準薬を選択し投与期間を設定する)について理解している。
\item
  抗菌薬の初期治療(経験的治療)について理解している。
\item
  抗菌薬の最適治療(標的治療)について理解している。
\item
  ワクチン予防可能な疾患について理解している
\item
  職業感染対策(ワクチン接種、針刺切創・体液曝露、結核曝露等)について理解している。
\item
  標準予防策(スタンダード・プリコーション)、感染経路別予防策(飛沫感染予防策、接触感染予防策や空気感染予防策等)が必要となる病原微生物、患者から医療従事者への病原微生物曝露を防ぐための個人防護具、予防接種等、医療従事者の体液曝露後の感染予防策について理解している。
\end{enumerate}

\hypertarget{ux816bux760d}{%
\subsubsection{腫瘍}\label{ux816bux760d}}

\begin{enumerate}
\def\labelenumi{\arabic{enumi}.}
\tightlist
\item
  腫瘍の定義とその特性について、ゲノム異常や分子機構とともに理解している。
\item
  我が国及び世界における各腫瘍の頻度等について理解している。
\item
  腫瘍性疾患発症の遺伝的素因・基礎疾患・感染症・環境生活習慣等のリスク因子や、腫瘍の予防・健診について理解している。
\item
  腫瘍マーカー、バイオマーカー、がん遺伝子パネル検査等、腫瘍に特化した検査とその所見について理解している。
\item
  腫瘍の内視鏡検査・画像検査(単純X線、CT、MRI、PET・核医学、超音波等)の異常所見がわかり診断できる。
\item
  腫瘍の生検・細胞診や病理検査とその所見について理解している。
\item
  腫瘍のTNM分類、ステージについて理解している。
\item
  がん患者の症候について理解している。
\item
  主な造血器腫瘍の症候、診断、治療について理解している。
\item
  主な脳腫瘍の症候、診断、治療について理解している。
\item
  主な皮膚腫瘍の症候、診断、治療について理解している。
\item
  主な骨軟部腫瘍の症候、診断、治療について理解している。
\item
  主な胸部腫瘍(呼吸器系)の症候、診断、治療について理解している。
\item
  主な消化器腫瘍の症候、診断、治療について理解している。
\item
  主な泌尿器系腫瘍の症候、診断、治療について理解している。
\item
  主な生殖器系腫瘍の症候、診断、治療について理解している。
\item
  主な乳腺腫瘍の症候、診断、治療について理解している。
\item
  主な内分泌系腫瘍の症候、診断、治療について理解している。
\item
  主な頭頚部癌の症候、診断、治療について理解している。
\item
  主な小児腫瘍の種類、症候、診断、治療について理解している。
\item
  原発不明癌、転移性腫瘍、重複癌、AYA世代の腫瘍、希少がんの種類、症候、診断、治療について理解している。
\item
  主な遺伝性腫瘍の症候、診断、治療について理解している。
\item
  オンコロジーエマージェンシー(脊髄圧迫、腫瘍崩壊、上大静脈症候群、代謝障害、治療の有害事象等)の起こりやすいがん、病態生理、症候と対応について理解している。
\item
  主な腫瘍の手術療法について理解している。
\item
  主な腫瘍の放射線療法・インターベンショナルラジオロジー(IVR)の適応について理解している。
\item
  主な腫瘍の薬物療法(細胞障害性抗癌薬、分子標的治療薬)、造血幹細胞移植、がん免疫に関する治療法について理解している。
\item
  がん患者に対する支持療法および緩和ケアを理解している。
\item
  腫瘍性疾患患者が直面する社会的・精神的な課題について理解している。
\end{enumerate}

\hypertarget{ux6551ux6025ux7cfbux4e2dux6bd2ux74b0ux5883ux56e0ux5b50ux306bux3088ux308bux75beux60a3ux3092ux542bux3080}{%
\subsubsection{救急系(中毒・環境因子による疾患を含む)}\label{ux6551ux6025ux7cfbux4e2dux6bd2ux74b0ux5883ux56e0ux5b50ux306bux3088ux308bux75beux60a3ux3092ux542bux3080}}

\begin{enumerate}
\def\labelenumi{\arabic{enumi}.}
\tightlist
\item
  地域の救急医療体制について病院前救護体制、メディカルコントロール、初期・二次・三次救急医療の概念を用いて理解している。
\item
  ショックの原因分類としての①血流分布異常性ショック(アナフィラキシー、敗血症性、神経原性)、②循環血液量減少性ショック(出血性、体液喪失)、③心原性ショック(心筋収縮力低下、弁疾患、不整脈)、④閉塞性ショック(心タンポナーデ、肺塞栓症、緊張性気胸)とそれぞれの病態および診断の要点について理解している。
\item
  ショックの患者の初期対応および原因に応じた治療について理解している。
\item
  心停止の原因分類としての①心血管原性(急性心筋梗塞、急性大動脈解離、大動脈瘤破裂、肺塞栓)、②呼吸原性(気道閉塞、緊張性気胸、肺実質病変による低酸素血症)、③
  神経原性(重症頭部・脊髄外傷、急性くも膜下出血)、④中毒・環境要因(中毒、熱中症、低体温症)、⑤電解質・酸塩基平衡異常(低・高カリウム血症、アシドーシス、低血糖)と病態および診断の要点について理解している。
\item
  心停止患者の初期対応(一次・二次救命処置)および原因に応じた治療について理解している。
\item
  中毒患者が呈するトキシドロームに基づき病歴および身体所見から中毒の起因物質を推定できる。
\item
  中毒患者への治療としての吸収の阻害、排泄の促進、拮抗薬の適応および禁忌について理解している。
\item
  食中毒、ガス中毒(一酸化炭素中毒、硫化水素、青酸ガス)農薬(有機リン・有機塩素)、アルコール、薬物(睡眠薬・向精神薬・解熱鎮痛薬・麻薬・覚醒剤)による中毒の病因(発生機序)、症候、診断と治療について理解している。
\item
  水銀、鉛、青酸、ヒ素、パラコート、自然毒、腐食剤による中毒(酸、アルカリ、フッ化水素)、ボタン電池誤飲
  による中毒について理解している。
\item
  高温による障害(熱中症)、低温による障害(低体温症)の症候、主な検査・診断、治療法について理解している。
\item
  気圧、振動、騒音による障害の症候、主な検査・診断、治療法について理解している。
\item
  外傷の病態と診断の要点について理解している。
\item
  熱傷の重症度を評価し(気道熱傷の有無、熱傷面積および深達度)、治療方針について理解している。
\end{enumerate}

\hypertarget{ux653eux5c04ux7ddaux306eux751fux4f53ux5f71ux97ffux3068ux653eux5c04ux7ddaux969cux5bb3ux653eux5c04ux7ddaux306eux751fux4f53ux5f71ux97ffux3068ux9069ux5207ux306aux5229ux7528ux3068ux653eux5c04ux7ddaux969cux5bb3}{%
\subsubsection{放射線の生体影響と放射線障害放射線の生体影響と適切な利用と放射線障害}\label{ux653eux5c04ux7ddaux306eux751fux4f53ux5f71ux97ffux3068ux653eux5c04ux7ddaux969cux5bb3ux653eux5c04ux7ddaux306eux751fux4f53ux5f71ux97ffux3068ux9069ux5207ux306aux5229ux7528ux3068ux653eux5c04ux7ddaux969cux5bb3}}

\begin{enumerate}
\def\labelenumi{\arabic{enumi}.}
\tightlist
\item
  放射線の種類と放射能、これらの性質・定量法・単位について理解している。
\item
  内部被ばくと外部被ばくについて、線量評価やその病態、症候、診断と治療について理解している。
\item
  放射線及び電磁波の人体(胎児を含む)への影響(急性影響と晩発影響)と適切な利用法について理解している。
\item
  種々の正常組織の放射線の透過性や放射線感受性の違いについて理解している。
\item
  磁気共鳴画像法(MRI)で用いられている磁場や電磁波の特徴を理解し、人体や植え込みデバイスの発熱等の現象について理解している。
\item
  医療被ばく・職業被ばくも含めた放射線被ばく低減の3原則と安全管理を理解し、放射線を用いる画像検査(X線写真、CT、血管造影インターベンショナルラジオロジー、X線透視等)の被ばく軽減を実行できる。
\item
  放射線診断(X線撮影、コンピュータ断層撮影(CT)、核医学)や血管造影及びインターベンショナルラジオロジーの利益とコスト・リスク(被ばく線量、急性・晩発影響等)を知り、適応の有無を判断できる。
\item
  放射線治療の生物学的原理と、放射線の遺伝子・細胞への作用と放射線による細胞死の機序、局所的・全身的影響について理解している。
\end{enumerate}

\hypertarget{ux305dux306eux4ed6}{%
\subsubsection{その他}\label{ux305dux306eux4ed6}}

\begin{enumerate}
\def\labelenumi{\arabic{enumi}.}
\tightlist
\item
  臓器不全(多臓器不全、サイトカインストーム、播種性血管内凝固症候群)について理解している。
\end{enumerate}

\newpage

\hypertarget{ux60c5ux5831ux79d1ux5b66ux6280ux8853ux3092ux6d3bux304bux3059ux80fdux529b}{%
\section{情報・科学技術を活かす能力}\label{ux60c5ux5831ux79d1ux5b66ux6280ux8853ux3092ux6d3bux304bux3059ux80fdux529b}}

発展し続ける情報社会を理解し、人工知能を含めた高度科学技術を活用しながら、医療・医学研究を最適化する。

\hypertarget{ux60c5ux5831ux79d1ux5b66ux6280ux8853ux306bux5411ux304dux5408ux3046ux305fux3081ux306eux502bux7406ux89b3ux3068ux30ebux30fcux30eb}{%
\subsection{情報・科学技術に向き合うための倫理観とルール}\label{ux60c5ux5831ux79d1ux5b66ux6280ux8853ux306bux5411ux304dux5408ux3046ux305fux3081ux306eux502bux7406ux89b3ux3068ux30ebux30fcux30eb}}

医療や研究等の場面で、情報科学技術を取り扱う際に必要な倫理観・デジタルプロフェッショナリズム及び基本的原則を理解する。

\hypertarget{ux60c5ux5831ux79d1ux5b66ux6280ux8853ux306bux5411ux304dux5408ux3046ux305fux3081ux306eux6e96ux5099}{%
\subsubsection{情報・科学技術に向き合うための準備}\label{ux60c5ux5831ux79d1ux5b66ux6280ux8853ux306bux5411ux304dux5408ux3046ux305fux3081ux306eux6e96ux5099}}

\begin{enumerate}
\def\labelenumi{\arabic{enumi}.}
\tightlist
\item
  情報・科学技術を医療に活用することの重要性と社会的意義を理解している。
\item
  医療における情報・科学技術に関連する規制(法律、ガイドライン等)を理解している。
\item
  デジタルデバイド※による医療格差等、情報・科学技術の医療への活用で起こりうる倫理的問題を議論できる。
\end{enumerate}

\hypertarget{ux60c5ux5831ux79d1ux5b66ux6280ux8853ux5229ux7528ux306bux3042ux305fux3063ux3066ux306eux502bux7406ux89b3ux3068ux30ebux30fcux30eb}{%
\subsubsection{情報・科学技術利用にあたっての倫理観とルール}\label{ux60c5ux5831ux79d1ux5b66ux6280ux8853ux5229ux7528ux306bux3042ux305fux3063ux3066ux306eux502bux7406ux89b3ux3068ux30ebux30fcux30eb}}

\begin{enumerate}
\def\labelenumi{\arabic{enumi}.}
\tightlist
\item
  電子カルテをはじめとする医療情報の管理・保管の原則について理解し、関連する規制(法律、倫理基準、個人情報保護のための規定等)を遵守できる。
\item
  ソーシャルメディア(インターネット、SNS等)の利用における医療者として相応しい情報発信のあり方を理解し、実践できる。
\end{enumerate}

\hypertarget{ux533bux7642ux3068ux305dux308cux3092ux53d6ux308aux5dfbux304fux793eux4f1aux306bux5fc5ux8981ux306aux60c5ux5831ux79d1ux5b66ux6280ux8853ux306eux539fux7406}{%
\subsection{医療とそれを取り巻く社会に必要な情報・科学技術の原理}\label{ux533bux7642ux3068ux305dux308cux3092ux53d6ux308aux5dfbux304fux793eux4f1aux306bux5fc5ux8981ux306aux60c5ux5831ux79d1ux5b66ux6280ux8853ux306eux539fux7406}}

安全かつ質の高い医療・医学研究に必要な情報・科学技術に関する基本理論を理解し、その知識を自身の学習や医療への適応する姿勢を体得する。

\hypertarget{ux60c5ux5831ux79d1ux5b66ux6280ux8853ux3092ux6d3bux7528ux3057ux305fux533bux7642}{%
\subsubsection{情報・科学技術を活用した医療}\label{ux60c5ux5831ux79d1ux5b66ux6280ux8853ux3092ux6d3bux7528ux3057ux305fux533bux7642}}

\begin{enumerate}
\def\labelenumi{\arabic{enumi}.}
\tightlist
\item
  情報端末(コンピューター、スマートフォン等)を用いてインターネットやアプリ等を医療の実践に活用できる。
\item
  情報・科学技術を用いて収集した情報およびデータを基に問題解決を図る。
\end{enumerate}

\hypertarget{ux60c5ux5831ux79d1ux5b66ux6280ux8853ux306eux5148ux7aefux77e5ux8b58}{%
\subsubsection{情報・科学技術の先端知識}\label{ux60c5ux5831ux79d1ux5b66ux6280ux8853ux306eux5148ux7aefux77e5ux8b58}}

\begin{enumerate}
\def\labelenumi{\arabic{enumi}.}
\tightlist
\item
  医療に関連する情報・科学技術(医療情報システム、ウェアラブルデバイス、アプリ、人工知能、遠隔医療技術、IoT※等)を理解し、それらの応用可能性について議論できる。
\item
  情報・科学技術の専門家とともに、技術を医療へ応用する際に、医療者に求められる役割を理解している。
\end{enumerate}

\hypertarget{ux8a3aux7642ux73feux5834ux306bux304aux3051ux308bux60c5ux5831ux79d1ux5b66ux6280ux8853ux306eux6d3bux7528}{%
\subsection{診療現場における情報・科学技術の活用}\label{ux8a3aux7642ux73feux5834ux306bux304aux3051ux308bux60c5ux5831ux79d1ux5b66ux6280ux8853ux306eux6d3bux7528}}

遠隔医療を含む患者診療、学習の最適化に有効なICTツールの実践スキル及びデジタルコミュニケーションスキルを修得する。

\hypertarget{ux60c5ux5831ux79d1ux5b66ux6280ux8853ux3092ux6d3bux7528ux3057ux305fux30b3ux30dfux30e5ux30cbux30b1ux30fcux30b7ux30e7ux30f3ux30b9ux30adux30eb}{%
\subsubsection{情報・科学技術を活用したコミュニケーションスキル}\label{ux60c5ux5831ux79d1ux5b66ux6280ux8853ux3092ux6d3bux7528ux3057ux305fux30b3ux30dfux30e5ux30cbux30b1ux30fcux30b7ux30e7ux30f3ux30b9ux30adux30eb}}

\begin{enumerate}
\def\labelenumi{\arabic{enumi}.}
\tightlist
\item
  電子カルテの特性を踏まえた適切な記載や活用ができる。
\item
  遠隔コミュニケーションのあり方を理解し、その目的に応じて適切なツール(電子メール、テレビ会議システム、SNS等)を選択し利用できる。
\end{enumerate}

\hypertarget{ux60c5ux5831ux79d1ux5b66ux6280ux8853ux3092ux6d3bux7528ux3057ux305fux5b66ux7fd2ux30b9ux30adux30eb}{%
\subsubsection{情報・科学技術を活用した学習スキル}\label{ux60c5ux5831ux79d1ux5b66ux6280ux8853ux3092ux6d3bux7528ux3057ux305fux5b66ux7fd2ux30b9ux30adux30eb}}

\begin{enumerate}
\def\labelenumi{\arabic{enumi}.}
\tightlist
\item
  自己学習や協同学習の場に適切なICT(eラーニング、モバイル技術等)を活用できる。
\item
  新たに登場する情報・科学技術を自身の学び及び医療に活用する柔軟性を有する。
\end{enumerate}

\newpage

\hypertarget{ux60a3ux8005ux30b1ux30a2ux306eux305fux3081ux306eux8a3aux7642ux6280ux80fd}{%
\section{患者ケアのための診療技能}\label{ux60a3ux8005ux30b1ux30a2ux306eux305fux3081ux306eux8a3aux7642ux6280ux80fd}}

安全で質の高い医療を実践するために、匠(たくみ)としての技(診療技能)を磨き、それを遺憾無く発揮して診療を実践する。

\hypertarget{ux60a3ux8005ux306eux60c5ux5831ux53ceux96c6}{%
\subsection{患者の情報収集}\label{ux60a3ux8005ux306eux60c5ux5831ux53ceux96c6}}

患者本人、家族、医療スタッフなど関係する様々なリソースを活用し、診療に必要な情報を収集できる。

\hypertarget{ux533bux7642ux9762ux63a5}{%
\subsubsection{医療面接}\label{ux533bux7642ux9762ux63a5}}

\begin{enumerate}
\def\labelenumi{\arabic{enumi}.}
\tightlist
\item
  医療面接における基本的コミュニケーション技法を用いることができる。
\item
  病歴(主訴、現病歴、常用薬、アレルギー歴、既往歴、家族歴、嗜好、生活習慣、社会歴・職業歴、生活環境、家庭環境、海外渡航歴、システムレビュー)を聴き取り、情報を取捨選択し整理できる。
\item
  患者に関わる人たちから必要な情報を得ることができる。
\end{enumerate}

\hypertarget{ux8eabux4f53ux6240ux898b}{%
\subsubsection{身体所見}\label{ux8eabux4f53ux6240ux898b}}

\begin{enumerate}
\def\labelenumi{\arabic{enumi}.}
\tightlist
\item
  患者の状態に応じた診察ができる。
\item
  全身の外観(体型、栄養、姿勢、歩行、顔貌、皮膚、発声)を評価できる。
\item
  バイタルサイン(体温、脈拍、血圧、呼吸数、酸素飽和度)の測定ができる。
\item
  適切な体位(立位、座位、半座位、臥位、砕石位)で診察できる。
\item
  部位毎の身体診察表.~\ref{tbl:physical}ができる。
\item
  主要診療科表.~\ref{tbl:ux4e3bux8981ux8a3aux7642ux79d1}において必要な診察ができる。
\end{enumerate}

\hypertarget{ux60a3ux8005ux60c5ux5831ux306eux7d71ux5408ux5206ux6790ux3068ux8a55ux4fa1ux8a3aux7642ux8a08ux753b}{%
\subsection{患者情報の統合、分析と評価、診療計画}\label{ux60a3ux8005ux60c5ux5831ux306eux7d71ux5408ux5206ux6790ux3068ux8a55ux4fa1ux8a3aux7642ux8a08ux753b}}

得られたすべての情報を統合し、様々な観点から分析し、必要な医療について評価した上で提供すべき医療を計画できる。

\hypertarget{ux30abux30ebux30c6ux8a18ux8f09}{%
\subsubsection{カルテ記載}\label{ux30abux30ebux30c6ux8a18ux8f09}}

\begin{enumerate}
\def\labelenumi{\arabic{enumi}.}
\tightlist
\item
  適切に患者の情報を収集し、問題志向型医療記録を作成できる。
\item
  診療経過をSOAP(主観的所見・客観的所見・評価・計画)で記載できる。
\item
  過去の診療経過をまとめて診療録に記載できる。
\end{enumerate}

\hypertarget{ux81e8ux5e8aux63a8ux8ad6}{%
\subsubsection{臨床推論}\label{ux81e8ux5e8aux63a8ux8ad6}}

\begin{enumerate}
\def\labelenumi{\arabic{enumi}.}
\tightlist
\item
  主要症候表.~\ref{tbl:ux4e3bux8981ux75c7ux5019}について原因と病態生理を理解している。
\item
  主要症候表.~\ref{tbl:ux4e3bux8981ux75c7ux5019}について鑑別診断を検討し、診断の要点を説明できる。
\item
  主要診療科表.~\ref{tbl:ux4e3bux8981ux8a3aux7642ux79d1}で主訴からの診断推論を組み立てられる。
\item
  主要診療科表.~\ref{tbl:ux4e3bux8981ux8a3aux7642ux79d1}における疾患の病態や疫学を理解している。
\end{enumerate}

\hypertarget{ux691cux67fbux8a08ux753bux5206ux6790ux8a55ux4fa1}{%
\subsubsection{検査(計画・分析評価)}\label{ux691cux67fbux8a08ux753bux5206ux6790ux8a55ux4fa1}}

\begin{enumerate}
\def\labelenumi{\arabic{enumi}.}
\tightlist
\item
  主要な臨床・画像検査の目的と意義を理解し、診断仮説の検証に最低限必要な検査項目を選択して、結果を解釈できる。
\item
  主要な臨床・画像検査の正しい検体採取方法と検体保存方法を理解している。
\item
  主要な臨床・画像検査の安全な実施方法(患者確認と検体確認、検査の合併症、感染症予防、精度管理)を理解している。
\item
  主要な臨床・画像検査の特性(感度、特異度、偽陽性、偽陰性、検査前確率・検査後確率、尤度比、ROC曲線)と判定基準(基準値・基準範囲、カットオフ値、パニック値)を理解している。
\item
  主要な臨床・画像検査の生理的変動、測定誤差、精度管理、ヒューマンエラーについて理解している。
\item
  患者に応じた検査値特性を理解し、結果を解釈できる。
\item
  主要な臨床・画像検査表.~\ref{tbl:ux4e3bux8981ux306aux81e8ux5e8a}の目的と適応を理解し、解釈できる。
\end{enumerate}

\hypertarget{ux6cbbux7642ux8a08ux753bux7d4cux904eux306eux8a55ux4fa1}{%
\subsubsection{治療(計画・経過の評価)}\label{ux6cbbux7642ux8a08ux753bux7d4cux904eux306eux8a55ux4fa1}}

\begin{enumerate}
\def\labelenumi{\arabic{enumi}.}
\tightlist
\item
  主要症候表.~\ref{tbl:ux4e3bux8981ux75c7ux5019}について初期対応を計画し、専門的診療が必要かどうかを考えることができる。
\item
  服薬の基本・アドヒアランスについて理解している。
\item
  処方箋の下書きを作成することができる。
\item
  薬の薬理作用、適応、有害事象、投与時の注意事項について理解している。
\item
  年齢や臓器障害に応じた薬物動態の特徴を考慮した薬剤投与の注意点について理解している。
\item
  薬物動態的相互作用について理解している。
\item
  使用禁忌、特定条件下での薬物使用(アンチ・ドーピング等)について理解している。
\item
  主な薬物アレルギーの症候、診察、診断、予防策と対処法について理解している。
\item
  薬物の蓄積、耐性、タキフィラキシー※、依存について理解している。
\item
  抗腫瘍薬の適応、有害事象、投与時の注意事項について理解している。
\item
  抗微生物薬の薬理作用、適応、有害事象、投与時の注意事項について理解している。
\item
  麻薬性鎮痛薬・鎮静薬の適応、有害事象、投与時の注意事項について理解している。
\item
  分子標的薬・バイオ医薬の薬理作用と有害事象について理解している。
\item
  漢方医学の特徴や、主な和漢薬(漢方薬)の適応、薬理作用について理解している。
\item
  主な放射線治療法の適応を理解している。
\item
  インターベンショナルラジオロジー(IVR)※について理解している。
\item
  内視鏡を用いる治療の概要を理解している。
\item
  超音波を用いる治療の概要を理解している。
\item
  被覆材の種類と適応、効果について理解している。
\item
  外科的治療の適応と合併症について理解している。
\item
  手術の危険因子とその対応の基本について理解している。
\item
  主な術後合併症とその予防の基本について理解している。
\item
  手術に関するインフォームド・コンセントの注意点について理解している。
\item
  周術期における事前のリスク評価について理解している。
\item
  周術期における主な薬剤の服薬管理(継続、中止等)の必要性とそれに伴うリスクについて理解している。
\item
  周術期における輸液・輸血について理解している。
\item
  周術期における疼痛の管理について理解している。
\item
  局所麻酔、末梢神経ブロック、神経叢ブロック、脊髄くも膜下麻酔、硬膜外麻酔の適応、禁忌と合併症について理解している。
\item
  麻酔管理を安全に行うための術前評価について理解している。
\item
  安全な麻酔のためのモニタリングの方法、重要な異常所見と対処法について理解している。
\item
  麻酔薬と筋弛緩薬の種類と使用上の原則について理解している。
\item
  吸入麻酔と静脈麻酔の適応、禁忌、方法、事故と合併症について理解している。
\item
  栄養アセスメント、栄養ケア・マネジメント、栄養サポートチーム(NST)、疾患別の栄養療法について理解している。
\item
  経静脈栄養と経管・経腸栄養の適応、方法と合併症、長期投与時の注意事項について理解している。
\item
  主な医療機器の種類と原理について理解している。
\item
  主な人工臓器の種類と原理について理解している。
\item
  血液製剤及び血漿分画製剤の種類と適応について理解している。
\item
  輸血副反応、輸血使用記録保管義務、不適合輸血の防止手順について理解している。
\item
  輸血の適正使用、成分輸血、自己血輸血、緊急時の輸血について理解している。
\item
  移植医療(臓器移植、組織移植、造血幹細胞移植等)の我が国と世界の状況について理解している。
\item
  終末期医療における臓器・組織提供選択提示の意義について理解している。
\item
  移植における免疫応答(拒絶反応、移植片対宿主病)について理解している。
\item
  移植後の免疫抑制について理解している。
\item
  リハビリテーションの概念と適応について理解している。
\item
  機能障害と日常生活動作(ADL)の評価ができる。
\item
  理学療法、作業療法と言語聴覚療法について理解している。
\item
  主な歩行補助具、車椅子、義肢(義手、義足)と装具について理解している。
\item
  主要診療科表.~\ref{tbl:ux4e3bux8981ux8a3aux7642ux79d1}の基本的な治療計画を立案できる。
\end{enumerate}

\hypertarget{ux6559ux80b2ux8a08ux753b}{%
\subsubsection{教育計画}\label{ux6559ux80b2ux8a08ux753b}}

\begin{enumerate}
\def\labelenumi{\arabic{enumi}.}
\tightlist
\item
  代表的な疾患における患者指導の計画を立案できる。
\end{enumerate}

\hypertarget{ux6cbbux7642ux3092ux542bux3080ux5bfeux5fdcux306eux5b9fux65bd}{%
\subsection{治療を含む対応の実施}\label{ux6cbbux7642ux3092ux542bux3080ux5bfeux5fdcux306eux5b9fux65bd}}

患者の状態の評価に基づいて患者本人、家族、医療スタッフと連携し、必要な医療を提案または実施できる。

\hypertarget{ux691cux67fbux624bux6280}{%
\subsubsection{検査手技}\label{ux691cux67fbux624bux6280}}

\begin{enumerate}
\def\labelenumi{\arabic{enumi}.}
\tightlist
\item
  検査に関する基本的臨床手技表.~\ref{tbl:ux57faux672cux7684ux81e8ux5e8aux624bux6280}を実施できる
\end{enumerate}

\hypertarget{ux6cbbux7642ux624bux6280}{%
\subsubsection{治療手技}\label{ux6cbbux7642ux624bux6280}}

\begin{enumerate}
\def\labelenumi{\arabic{enumi}.}
\tightlist
\item
  治療に関する基本的臨床手技表.~\ref{tbl:ux57faux672cux7684ux81e8ux5e8aux624bux6280}を実施できる。
\end{enumerate}

\hypertarget{ux6551ux6025ux521dux671fux5bfeux5fdc}{%
\subsubsection{救急・初期対応}\label{ux6551ux6025ux521dux671fux5bfeux5fdc}}

\begin{enumerate}
\def\labelenumi{\arabic{enumi}.}
\tightlist
\item
  バイタルサインや身体徴候から緊急性の高い状態にある患者を認識できる。
\item
  一次救命処置を実施できる。
\item
  頻度・緊急性の高い患者に対する初期対応(二次救命処置を含む)の実施を補助できる。
\item
  外傷の病態生理と診断について理解している。
\item
  外傷の初期対応の実施を補助できる。
\item
  アナフィラキシーショックの対応を補助できる。
\end{enumerate}

\hypertarget{ux66f8ux985eux306eux4f5cux6210}{%
\subsubsection{書類の作成}\label{ux66f8ux985eux306eux4f5cux6210}}

\begin{enumerate}
\def\labelenumi{\arabic{enumi}.}
\tightlist
\item
  各種診断書・証明書の下書きを作成できる。
\item
  各種検案書の下書きを作成できる。
\end{enumerate}

\hypertarget{ux60a3ux8005ux30b1ux30a2ux306bux5fc5ux8981ux306aux9023ux643a}{%
\subsubsection{患者ケアに必要な連携}\label{ux60a3ux8005ux30b1ux30a2ux306bux5fc5ux8981ux306aux9023ux643a}}

\begin{enumerate}
\def\labelenumi{\arabic{enumi}.}
\tightlist
\item
  主要診療科表.~\ref{tbl:ux4e3bux8981ux8a3aux7642ux79d1}にどのようにコンサルテーションすればよいかを理解している。
\item
  褥瘡の予防、評価、処置・治療について理解している。
\end{enumerate}

\hypertarget{ux8a3aux7642ux8a08ux753bux30abux30f3ux30d5ux30a1ux30ecux30f3ux30b9}{%
\subsubsection{診療計画カンファレンス}\label{ux8a3aux7642ux8a08ux753bux30abux30f3ux30d5ux30a1ux30ecux30f3ux30b9}}

\begin{enumerate}
\def\labelenumi{\arabic{enumi}.}
\tightlist
\item
  症例検討会において適切にプレゼンテーションできる。
\item
  診察で得た情報を上級医にわかりやすく報告できる。
\end{enumerate}

\hypertarget{ux8a3aux7642ux7d4cux904eux306eux632fux308aux8fd4ux308aux3068ux6539ux5584}{%
\subsection{診療経過の振り返りと改善}\label{ux8a3aux7642ux7d4cux904eux306eux632fux308aux8fd4ux308aux3068ux6539ux5584}}

実施された医療を省察し、言語化して他者に説明し、次回に向けて改善につなげることができる。

\hypertarget{ux632fux308aux8fd4ux308aux30abux30f3ux30d5ux30a1ux30ecux30f3ux30b9}{%
\subsubsection{振り返りカンファレンス}\label{ux632fux308aux8fd4ux308aux30abux30f3ux30d5ux30a1ux30ecux30f3ux30b9}}

\begin{enumerate}
\def\labelenumi{\arabic{enumi}.}
\tightlist
\item
  M\&Mカンファレンスに参加して自身の意見を述べることができる。
\item
  CPCに参加して自身の意見を述べることができる。
\end{enumerate}

\hypertarget{ux533bux7642ux306eux8ceaux3068ux60a3ux8005ux5b89ux5168}{%
\subsection{医療の質と患者安全}\label{ux533bux7642ux306eux8ceaux3068ux60a3ux8005ux5b89ux5168}}

医療の質と患者安全の観点で自己の行動を省察し、組織改善と患者中心の視点を獲得する。

\hypertarget{ux533bux7642ux306eux8ceaux5411ux4e0a}{%
\subsubsection{医療の質向上}\label{ux533bux7642ux306eux8ceaux5411ux4e0a}}

\begin{enumerate}
\def\labelenumi{\arabic{enumi}.}
\tightlist
\item
  品質改善の手法を用いて医療を改善する重要性を理解し、繰り返し評価する姿勢を身に着ける。
\end{enumerate}

\hypertarget{ux533bux7642ux5f93ux4e8bux8005ux306eux5065ux5eb7ux7ba1ux7406}{%
\subsubsection{医療従事者の健康管理}\label{ux533bux7642ux5f93ux4e8bux8005ux306eux5065ux5eb7ux7ba1ux7406}}

\begin{enumerate}
\def\labelenumi{\arabic{enumi}.}
\tightlist
\item
  医療従事者に求められる健康管理(生活習慣改善、予防接種、被ばく低減策)、職業感染対策(結核スクリーニング、ワクチン接種)を実践する。
\item
  自身を含む医療者の労働環境の改善の必要性を理解し、実際の医療現場において改善に努めることができる。
\end{enumerate}

\hypertarget{ux5b89ux5168ux7ba1ux7406ux4f53ux5236}{%
\subsubsection{安全管理体制}\label{ux5b89ux5168ux7ba1ux7406ux4f53ux5236}}

\begin{enumerate}
\def\labelenumi{\arabic{enumi}.}
\tightlist
\item
  患者安全のための管理体制と各々の役割(リスクマネージャー、医療安全管理委員会等)を理解している。
\item
  医療過誤に関連した刑事・民事責任や医師法に基づく行政処分を理解している。
\end{enumerate}

\hypertarget{ux611fux67d3ux5236ux5fa1}{%
\subsubsection{感染制御}\label{ux611fux67d3ux5236ux5fa1}}

\begin{enumerate}
\def\labelenumi{\arabic{enumi}.}
\tightlist
\item
  医療関連感染症に関連したシステム(院内感染対策委員会、院内感染サーベイランス、感染制御チーム(ICT)、感染対策マニュアル等)の役割や意義を理解して参加する。
\item
  標準予防策(Standard precautions)の必要性を説明し、実践できる。
\item
  針刺切創、体液暴露等に遭遇した際、適切に対処できる。
\end{enumerate}

\hypertarget{ux60a3ux8005ux5b89ux5168ux306eux914dux616eux3068ux4fc3ux9032}{%
\subsubsection{患者安全の配慮と促進}\label{ux60a3ux8005ux5b89ux5168ux306eux914dux616eux3068ux4fc3ux9032}}

\begin{enumerate}
\def\labelenumi{\arabic{enumi}.}
\tightlist
\item
  基本的予防策(患者確認、ダブルチェック、チェックリスト法、類似名称薬への注意喚起、フェイルセイフ・フールプルーフの考え方等)を実践できる。
\item
  医療の安全性に関する情報(薬剤等の副作用、薬害、医療過誤、やってはいけないこと、優れた取組事例等)を共有し、事後に役立てるための分析ができる。
\end{enumerate}

\hypertarget{ux60a3ux8005ux5b89ux5168ux306eux5b9fux8df5}{%
\subsubsection{患者安全の実践}\label{ux60a3ux8005ux5b89ux5168ux306eux5b9fux8df5}}

\begin{enumerate}
\def\labelenumi{\arabic{enumi}.}
\tightlist
\item
  個人及び組織における患者安全管理の重要性を理解し、行動できる。
\item
  診療録の重要性を理解し、適切に記載し取り扱うことができる。
\item
  真摯に疑義に応じることができる。
\item
  医療上の事故等(インシデントを含む)が発生した際に、緊急対応や記録、報告することができる。
\end{enumerate}

\hypertarget{ux533bux7642ux306eux8ceaux3068ux60a3ux8005ux5b89ux5168-1}{%
\subsection{医療の質と患者安全}\label{ux533bux7642ux306eux8ceaux3068ux60a3ux8005ux5b89ux5168-1}}

\hypertarget{ux533bux7642ux306eux8ceaux5411ux4e0a-1}{%
\subsubsection{医療の質向上}\label{ux533bux7642ux306eux8ceaux5411ux4e0a-1}}

\begin{enumerate}
\def\labelenumi{\arabic{enumi}.}
\tightlist
\item
  品質改善の手法を用いて医療を改善する重要性を理解し、繰り返し評価する姿勢を身に着ける。
\end{enumerate}

\hypertarget{ux533bux7642ux5f93ux4e8bux8005ux306eux5065ux5eb7ux7ba1ux7406-1}{%
\subsubsection{医療従事者の健康管理}\label{ux533bux7642ux5f93ux4e8bux8005ux306eux5065ux5eb7ux7ba1ux7406-1}}

\begin{enumerate}
\def\labelenumi{\arabic{enumi}.}
\tightlist
\item
  医療従事者に求められる健康管理(生活習慣改善、予防接種、被ばく低減策)、職業感染対策(結核スクリーニング、ワクチン接種)を実践する。
\item
  自身を含む医療者の労働環境の改善の必要性を理解し、実際の医療現場において改善に努めることができる。
\end{enumerate}

\hypertarget{ux5b89ux5168ux7ba1ux7406ux4f53ux5236-1}{%
\subsubsection{安全管理体制}\label{ux5b89ux5168ux7ba1ux7406ux4f53ux5236-1}}

\begin{enumerate}
\def\labelenumi{\arabic{enumi}.}
\tightlist
\item
  患者安全のための管理体制と各々の役割(リスクマネージャー、医療安全管理委員会等)を理解している。
\item
  医療過誤に関連した刑事・民事責任や医師法に基づく行政処分を理解している。
\end{enumerate}

\hypertarget{ux611fux67d3ux5236ux5fa1-1}{%
\subsubsection{感染制御}\label{ux611fux67d3ux5236ux5fa1-1}}

\begin{enumerate}
\def\labelenumi{\arabic{enumi}.}
\tightlist
\item
  医療関連感染症に関連したシステム(院内感染対策委員会、院内感染サーベイランス、感染制御チーム(ICT)、感染対策マニュアル等)の役割や意義を理解して参加する。
\item
  標準予防策(Standard precautions)の必要性を説明し、実践できる。
\item
  針刺切創、体液暴露等に遭遇した際、適切に対処できる。
\end{enumerate}

\hypertarget{ux60a3ux8005ux5b89ux5168ux306eux914dux616eux3068ux4fc3ux9032-1}{%
\subsubsection{患者安全の配慮と促進}\label{ux60a3ux8005ux5b89ux5168ux306eux914dux616eux3068ux4fc3ux9032-1}}

\begin{enumerate}
\def\labelenumi{\arabic{enumi}.}
\tightlist
\item
  基本的予防策(患者確認、ダブルチェック、チェックリスト法、類似名称薬への注意喚起、フェイルセイフ・フールプルーフの考え方等)を実践できる。
\item
  医療の安全性に関する情報(薬剤等の副作用、薬害、医療過誤、やってはいけないこと、優れた取組事例等)を共有し、事後に役立てるための分析ができる。
\end{enumerate}

\hypertarget{ux60a3ux8005ux5b89ux5168ux306eux5b9fux8df5-1}{%
\subsubsection{患者安全の実践}\label{ux60a3ux8005ux5b89ux5168ux306eux5b9fux8df5-1}}

\begin{enumerate}
\def\labelenumi{\arabic{enumi}.}
\tightlist
\item
  個人及び組織における患者安全管理の重要性を理解し、行動できる。
\item
  診療録の重要性を理解し、適切に記載し取り扱うことができる。
\item
  真摯に疑義に応じることができる。
\item
  医療上の事故等(インシデントを含む)が発生した際に、緊急対応や記録、報告することができる。
\end{enumerate}

\newpage

\hypertarget{ux30b3ux30dfux30e5ux30cbux30b1ux30fcux30b7ux30e7ux30f3ux80fdux529b}{%
\section{コミュニケーション能力}\label{ux30b3ux30dfux30e5ux30cbux30b1ux30fcux30b7ux30e7ux30f3ux80fdux529b}}

患者及び患者に関わる全ての人と、相手の状況を考慮した上で良好なコミュニケーションをとり、患者の意思決定を支援して、安全で質の高い医療を実践する。

\hypertarget{ux60a3ux8005ux306bux63a5ux3059ux308bux8a00ux8449ux9063ux3044ux614bux5ea6ux8eabux3060ux3057ux306aux307fux914dux616e}{%
\subsection{患者に接する言葉遣い・態度・身だしなみ・配慮}\label{ux60a3ux8005ux306bux63a5ux3059ux308bux8a00ux8449ux9063ux3044ux614bux5ea6ux8eabux3060ux3057ux306aux307fux914dux616e}}

患者のプライバシー、苦痛などに配慮し、非言語コミュニケーションを含めた適切なコミュニケーションスキルにより良好な人間関係を築くことができる。

\hypertarget{ux60a3ux8005ux5bb6ux65cfux3078ux306eux9069ux5207ux306aux30b3ux30dfux30e5ux30cbux30b1ux30fcux30b7ux30e7ux30f3ux30b9ux30adux30ebux306eux6d3bux7528}{%
\subsubsection{患者・家族への適切なコミュニケーションスキルの活用}\label{ux60a3ux8005ux5bb6ux65cfux3078ux306eux9069ux5207ux306aux30b3ux30dfux30e5ux30cbux30b1ux30fcux30b7ux30e7ux30f3ux30b9ux30adux30ebux306eux6d3bux7528}}

\begin{enumerate}
\def\labelenumi{\arabic{enumi}.}
\tightlist
\item
  コミュニケーションが患者-医師間の互いの態度・行動や役割に及ぼす影響を考慮することができる。
\item
  言語的コミュニケーション技能を発揮して、良好な人間関係を築くことができる。
\item
  非言語的コミュニケーション(身だしなみ、視線、表情、ジェスチャーなど)を意識できる。
\item
  患者や家族に敬意を持った言葉遣いや態度で接することができる。
\item
  対人関係に関わる心理的要因(陽性感情・陰性感情等)を認識しながらコミュニケーションをとることができる。
\item
  相手の話を聞き、事実や自分の意見を相手にわかるように述べることができる
\end{enumerate}

\hypertarget{ux60a3ux8005ux306eux7acbux5834ux306eux5c0aux91cdux3068ux82e6ux75dbux3078ux306eux914dux616e}{%
\subsubsection{患者の立場の尊重と苦痛への配慮}\label{ux60a3ux8005ux306eux7acbux5834ux306eux5c0aux91cdux3068ux82e6ux75dbux3078ux306eux914dux616e}}

\begin{enumerate}
\def\labelenumi{\arabic{enumi}.}
\tightlist
\item
  患者や家族の精神的・身体的・社会的苦痛に十分配慮できる。
\item
  患者や家族の話を傾聴し、怒りや悲しみ、不安などの感情を理解し、共感することができる。
\end{enumerate}

\hypertarget{ux60a3ux8005ux306eux610fux601dux6c7aux5b9aux306eux652fux63f4ux3068ux305dux306eux305fux3081ux306eux60c5ux5831ux53ceux96c6ux308fux304bux308aux3084ux3059ux3044ux8aacux660e}{%
\subsection{患者の意思決定の支援とそのための情報収集・わかりやすい説明}\label{ux60a3ux8005ux306eux610fux601dux6c7aux5b9aux306eux652fux63f4ux3068ux305dux306eux305fux3081ux306eux60c5ux5831ux53ceux96c6ux308fux304bux308aux3084ux3059ux3044ux8aacux660e}}

患者や家族の多様性に配慮し、必要な情報についてわかりやすく説明を行い、患者の主体的な治療やマネジメントに関する最善の意思決定を支援することができる。

\hypertarget{ux60a3ux8005ux3078ux306eux308fux304bux308aux3084ux3059ux3044ux8a00ux8449ux306eux8aacux660e}{%
\subsubsection{患者へのわかりやすい言葉の説明}\label{ux60a3ux8005ux3078ux306eux308fux304bux308aux3084ux3059ux3044ux8a00ux8449ux306eux8aacux660e}}

\begin{enumerate}
\def\labelenumi{\arabic{enumi}.}
\tightlist
\item
  患者や家族の多様性(高齢者、小児、障害者、LGBTQ、国籍、人種、文化・言語・慣習の違い等)に配慮してコミュニケーションをとることができる。
\item
  患者の不安を軽減するためにわかりやすい言葉で説明や対話ができる。
\end{enumerate}

\hypertarget{ux60a3ux8005ux306eux610fux601dux6c7aux5b9aux306eux652fux63f4}{%
\subsubsection{患者の意思決定の支援}\label{ux60a3ux8005ux306eux610fux601dux6c7aux5b9aux306eux652fux63f4}}

\begin{enumerate}
\def\labelenumi{\arabic{enumi}.}
\tightlist
\item
  患者の自己決定を阻害する問題点を理解する。
\item
  患者の経験を尊重し、価値観を明確にできるように傾聴することができる。
\item
  患者の意思決定支援のために、最善のエビデンスをできるだけ専門用語を使わずに、わかりやすく説明することができる。
\item
  患者の価値観に沿った目標に基づいた治療方針を計画することができる。
\end{enumerate}

\hypertarget{ux60a3ux8005ux3084ux5bb6ux65cfux306eux30cbux30fcux30baux306eux628aux63e1ux3068ux914dux616e}{%
\subsection{患者や家族のニーズの把握と配慮}\label{ux60a3ux8005ux3084ux5bb6ux65cfux306eux30cbux30fcux30baux306eux628aux63e1ux3068ux914dux616e}}

患者や家族の心理的、社会的背景を広い視野で捉える姿勢を持ち、患者の持つ困難や必要な情報提供に対応することができる。

\hypertarget{ux60a3ux8005ux3084ux5bb6ux65cfux306eux8ab2ux984cux3092ux628aux63e1ux3057ux5fc5ux8981ux306aux60c5ux5831ux3092ux5f97ux308bux3053ux3068ux304cux3067ux304dux308b}{%
\subsubsection{患者や家族の課題を把握し、必要な情報を得ることができる}\label{ux60a3ux8005ux3084ux5bb6ux65cfux306eux8ab2ux984cux3092ux628aux63e1ux3057ux5fc5ux8981ux306aux60c5ux5831ux3092ux5f97ux308bux3053ux3068ux304cux3067ux304dux308b}}

\begin{enumerate}
\def\labelenumi{\arabic{enumi}.}
\tightlist
\item
  削除:患者自身から情報が得られない場合、代理人や保護者等から必要な情報を得ることができる。
\end{enumerate}

\hypertarget{ux60a3ux8005ux3084ux5bb6ux65cfux306eux30cbux30fcux30baux3092ux628aux63e1ux3057ux5fc5ux8981ux306aux60c5ux5831ux3092ux5f97ux308bux3053ux3068ux304cux3067ux304dux308b}{%
\subsubsection{患者や家族のニーズを把握し、必要な情報を得ることができる}\label{ux60a3ux8005ux3084ux5bb6ux65cfux306eux30cbux30fcux30baux3092ux628aux63e1ux3057ux5fc5ux8981ux306aux60c5ux5831ux3092ux5f97ux308bux3053ux3068ux304cux3067ux304dux308b}}

\begin{enumerate}
\def\labelenumi{\arabic{enumi}.}
\tightlist
\item
  患者の心理的及び社会的背景や自立した生活を送るためのニーズを把握することができる。
\item
  患者が抱える課題、問題点を抽出・整理できる。
\item
  患者自身から情報が得られない場合、代理人や保護者等から必要な情報を得ることができる。
\end{enumerate}

\hypertarget{ux60a3ux8005ux3084ux5bb6ux65cfux306eux8996ux70b9ux304bux3089ux5fc3ux7406ux793eux4f1aux7684ux80ccux666fux306bux914dux616eux3057ux305fux8a3aux7642ux3092ux884cux3046ux3053ux3068ux304cux3067ux304dux308b}{%
\subsubsection{患者や家族の視点から、心理・社会的背景に配慮した診療を行うことができる}\label{ux60a3ux8005ux3084ux5bb6ux65cfux306eux8996ux70b9ux304bux3089ux5fc3ux7406ux793eux4f1aux7684ux80ccux666fux306bux914dux616eux3057ux305fux8a3aux7642ux3092ux884cux3046ux3053ux3068ux304cux3067ux304dux308b}}

\begin{enumerate}
\def\labelenumi{\arabic{enumi}.}
\tightlist
\item
  家族や地域といった視点をもちながら、コミュニケーションをとることができる。
\item
  心理・社会的背景に配慮した診療に可能な範囲で参加することができる。
\item
  医療の不確実性を理解した上で適切な行動や態度がとれる。
\end{enumerate}

\newpage

\hypertarget{ux591aux8077ux7a2eux9023ux643aux80fdux529b}{%
\section{多職種連携能力}\label{ux591aux8077ux7a2eux9023ux643aux80fdux529b}}

保健、医療、福祉、介護など患者・家族に関わる全ての人々の役割を理解し、お互いに良好な関係を築きながら、患者・家族・地域の課題を共有し、関わる人々と協働することができる。

\hypertarget{ux9023ux643aux306eux57faux76e4}{%
\subsection{連携の基盤}\label{ux9023ux643aux306eux57faux76e4}}

患者や利用者、家族、地域の重要な課題について、協働する関係者と共通の目標を設定する過程で、背景が異なることに互いに配慮し、役割、知識、意見、価値を伝え合うことができる。

\hypertarget{ux60a3ux8005ux4e2dux5fc3ux306eux4fddux5065ux533bux7642ux798fux7949}{%
\subsubsection{患者中心の保健医療福祉}\label{ux60a3ux8005ux4e2dux5fc3ux306eux4fddux5065ux533bux7642ux798fux7949}}

\begin{enumerate}
\def\labelenumi{\arabic{enumi}.}
\tightlist
\item
  患者・利用者・家族に関連する情報について、多職種及び他の医療系学部の学生と共有できる。
\end{enumerate}

\hypertarget{ux8077ux7a2eux9593ux30b3ux30dfux30e5ux30cbux30b1ux30fcux30b7ux30e7ux30f3}{%
\subsubsection{職種間コミュニケーション}\label{ux8077ux7a2eux9593ux30b3ux30dfux30e5ux30cbux30b1ux30fcux30b7ux30e7ux30f3}}

\begin{enumerate}
\def\labelenumi{\arabic{enumi}.}
\tightlist
\item
  多職種及び他の医療系学部の学生の役割や意見を尊重した説明や返答、問いかけができる。
\end{enumerate}

\hypertarget{ux533bux5e2bux9593ux306eux7d39ux4ecbux3068ux76f8ux8ac7}{%
\subsubsection{医師間の紹介と相談}\label{ux533bux5e2bux9593ux306eux7d39ux4ecbux3068ux76f8ux8ac7}}

\begin{enumerate}
\def\labelenumi{\arabic{enumi}.}
\tightlist
\item
  適切な診断・検査・治療のために、適切な施設・専門科・医師への紹介や相談ができる。
\item
  患者のケアと責任が継続できるよう、医師間での考えや期待を共有できる。
\end{enumerate}

\hypertarget{ux5354ux50cdux5b9fux8df5}{%
\subsection{協働実践}\label{ux5354ux50cdux5b9fux8df5}}

自他の役割や思考・行為・感情・価値観を踏まえ、協働する職種で信頼関係を構築し、時に生じる職種間の葛藤にも適切に対応しながら、
互いの知識・技術を活かし合い、職種としての役割を全うできる。

\hypertarget{ux8077ux7a2eux5f79ux5272}{%
\subsubsection{職種役割}\label{ux8077ux7a2eux5f79ux5272}}

\begin{enumerate}
\def\labelenumi{\arabic{enumi}.}
\tightlist
\item
  自らの知識や価値観を、多職種及び他の医療系学部の学生に伝えられる。
\item
  多職種及び他の医療系学部の学生の中で自らの役割を果たせる。
\end{enumerate}

\hypertarget{ux95a2ux4fc2ux6027ux3078ux306eux50cdux304dux304bux3051}{%
\subsubsection{関係性への働きかけ}\label{ux95a2ux4fc2ux6027ux3078ux306eux50cdux304dux304bux3051}}

\begin{enumerate}
\def\labelenumi{\arabic{enumi}.}
\tightlist
\item
  多職種及び他の医療系学部の学生とともに学びあい、成長できる。
\item
  対人関係や対人行動に関わる概念について理解している。
\end{enumerate}

\hypertarget{ux81eaux8077ux7a2eux306eux7701ux5bdf}{%
\subsubsection{自職種の省察}\label{ux81eaux8077ux7a2eux306eux7701ux5bdf}}

\begin{enumerate}
\def\labelenumi{\arabic{enumi}.}
\tightlist
\item
  医師の役割を多職種及び他の医療系学部の学生に説明できる。
\item
  自らの価値観や言動について、多職種及び他の医療系学部の学生との関係性の中で、相対化できる。
\end{enumerate}

\hypertarget{ux4ed6ux8077ux7a2eux306eux7406ux89e3}{%
\subsubsection{他職種の理解}\label{ux4ed6ux8077ux7a2eux306eux7406ux89e3}}

\begin{enumerate}
\def\labelenumi{\arabic{enumi}.}
\tightlist
\item
  病院・診療所・施設などの職場環境やチームや部門などの所属に応じた他職種の役割を理解している。
\end{enumerate}

\newpage

\hypertarget{ux793eux4f1aux306bux304aux3051ux308bux533bux7642ux306eux5f79ux5272ux306eux7406ux89e3}{%
\section{社会における医療の役割の理解}\label{ux793eux4f1aux306bux304aux3051ux308bux533bux7642ux306eux5f79ux5272ux306eux7406ux89e3}}

医療は社会の一部であるという認識を持ち、経済的な観点・地域性の視点・国際的な視野も持ちながら、公正な医療を提供し、健康の代弁者として公衆衛生の向上に努める。

\hypertarget{ux793eux4f1aux4fddux969c}{%
\subsection{社会保障}\label{ux793eux4f1aux4fddux969c}}

憲法で定められた「生存権」を守る社会保障制度、公衆衛生とは何か、地域保健、産業保健、健康危機管理を理解する。保健統計の意義・利用法を学ぶ。

\hypertarget{ux516cux8846ux885bux751f}{%
\subsubsection{公衆衛生}\label{ux516cux8846ux885bux751f}}

\begin{enumerate}
\def\labelenumi{\arabic{enumi}.}
\tightlist
\item
  公衆衛生の概念を理解している。
\item
  地域共生社会の概念を理解している。
\item
  予防の段階とそれらの戦略を理解している。
\item
  公衆衛生活動(健診、健康づくりイベント等)の意義を理解し、役割の一部を担うことができる。
\end{enumerate}

\hypertarget{ux793eux4f1aux4fddux967aux516cux7684ux6276ux52a9ux793eux4f1aux798fux7949}{%
\subsubsection{社会保険、公的扶助、社会福祉}\label{ux793eux4f1aux4fddux967aux516cux7684ux6276ux52a9ux793eux4f1aux798fux7949}}

\begin{enumerate}
\def\labelenumi{\arabic{enumi}.}
\tightlist
\item
  生存権等の健康に関する基本的人権と社会保障(社会保険、社会福祉、公的扶助)の意義と内容を理解している。
\item
  国民皆保険としての医療保険、介護保険、年金保険を含む社会保険の仕組みと問題点を理解し、改善策を議論できる。
\item
  障害者の日常生活及び社会生活を総合的に支援するための法律(障害者総合支援法)等の障害者福祉を理解している。
\end{enumerate}

\hypertarget{ux5730ux57dfux4fddux5065}{%
\subsubsection{地域保健}\label{ux5730ux57dfux4fddux5065}}

\begin{enumerate}
\def\labelenumi{\arabic{enumi}.}
\tightlist
\item
  保健所・市町村保健センター・地方衛生研究所の役割を理解している。
\item
  健康増進法、栄養、身体活動、休養等の健康増進施策の意義と内容を理解している。
\item
  地域保健に関連する基本的な制度や法律を理解している。
\end{enumerate}

\hypertarget{ux7523ux696dux4fddux5065ux74b0ux5883ux4fddux5065}{%
\subsubsection{産業保健・環境保健}\label{ux7523ux696dux4fddux5065ux74b0ux5883ux4fddux5065}}

\begin{enumerate}
\def\labelenumi{\arabic{enumi}.}
\tightlist
\item
  産業保健の意義、労働衛生の3管理等、産業保健の基本的な考え方を理解している。
\item
  産業保健・環境保健に関連する基本的な制度や法律を理解している。
\item
  労働災害及び職業性疾病とのその対策を理解している。
\item
  有害物質による産業中毒とその対策を理解している。
\end{enumerate}

\hypertarget{ux5065ux5eb7ux5371ux6a5fux7ba1ux7406}{%
\subsubsection{健康危機管理}\label{ux5065ux5eb7ux5371ux6a5fux7ba1ux7406}}

\begin{enumerate}
\def\labelenumi{\arabic{enumi}.}
\tightlist
\item
  健康危機の概念と種類、それらへの対応(リスクコミュニケーションを含む)について理解している。
\item
  健康危機管理(食品感染症、放射線事故、災害等さまざまな有事)に関連する基本的な制度や法律を理解している。
\item
  災害拠点病院、種々の活動チーム等、災害保健医療の意義を理解している。
\end{enumerate}

\hypertarget{ux75abux5b66ux533bux5b66ux7d71ux8a08}{%
\subsection{疫学・医学統計}\label{ux75abux5b66ux533bux5b66ux7d71ux8a08}}

人間集団を対象とする研究法である疫学の考え方と意義、主な研究デザインを学ぶ。医学、生物学における統計手法の基本的な考え方を理解する。

\hypertarget{ux4fddux5065ux7d71ux8a08}{%
\subsubsection{保健統計}\label{ux4fddux5065ux7d71ux8a08}}

\begin{enumerate}
\def\labelenumi{\arabic{enumi}.}
\tightlist
\item
  主な人口統計(人口静態と人口動態)、疾病・障害の分類・統計(ICD等)を理解している。
\item
  平均寿命、健康寿命について説明できる。
\end{enumerate}

\hypertarget{ux75abux5b66}{%
\subsubsection{疫学}\label{ux75abux5b66}}

\begin{enumerate}
\def\labelenumi{\arabic{enumi}.}
\tightlist
\item
  公衆衛生と臨床の視点から見た疫学の役割を理解している。
\item
  割合・比・率の違いおよび代表的な疫学指標(有病割合、リスク比、罹患率等)を理解している。
\item
  主なバイアス・交絡を例示できる。
\item
  年齢調整における直接法と間接法の違いを説明できる。
\item
  主な疫学の研究デザインとして、観察研究(記述研究、横断研究、症例対照研究、コホート研究)および介入研究(ランダム化比較試験等)を理解している。
\item
  急性感染症に特異的な疫学的アプローチを理解している。
\item
  エビデンスの限界を踏まえながら、集団に影響する意思決定を支援できる。
\end{enumerate}

\hypertarget{ux30c7ux30fcux30bfux89e3ux6790ux3068ux7d71ux8a08ux624bux6cd5}{%
\subsubsection{データ解析と統計手法}\label{ux30c7ux30fcux30bfux89e3ux6790ux3068ux7d71ux8a08ux624bux6cd5}}

\begin{enumerate}
\def\labelenumi{\arabic{enumi}.}
\tightlist
\item
  尺度(間隔、比、順序、名義)について説明できる。
\item
  データの分布(欠損値を含む)について説明できる。
\item
  正規分布の母平均の信頼区間について説明できる。
\item
  相関分析、平均値と割合の検定等を実施できる。
\item
  多変量解析の意義を理解している。
\end{enumerate}

\hypertarget{ux6cd5ux533bux5b66}{%
\subsection{法医学}\label{ux6cd5ux533bux5b66}}

死の判定や死亡診断と死体検案を理解する。

\hypertarget{ux6b7bux3068ux6cd5}{%
\subsubsection{死と法}\label{ux6b7bux3068ux6cd5}}

\begin{enumerate}
\def\labelenumi{\arabic{enumi}.}
\tightlist
\item
  植物状態、脳死、心臓死及び脳死判定について理解している。
\item
  異状死・異状死体の取扱いと死体検案について理解している。
\item
  死亡診断書と死体検案書を作成できる。
\item
  個人識別の方法を理解している。
\item
  病理解剖、法医解剖(司法解剖、行政解剖、死因・身元調査法解剖、承諾解剖)について理解している。
\end{enumerate}

\hypertarget{ux793eux4f1aux306eux69cbux9020ux3084ux5909ux5316ux304bux3089ux6349ux3048ux308bux533bux7642}{%
\subsection{社会の構造や変化から捉える医療}\label{ux793eux4f1aux306eux69cbux9020ux3084ux5909ux5316ux304bux3089ux6349ux3048ux308bux533bux7642}}

患者の抱える健康に関する問題の背景にある社会的な課題を適切に捉え、その解決のために積極的に行動する。

\hypertarget{ux793eux4f1aux683cux5deeux3068ux533bux7642}{%
\subsubsection{社会格差と医療}\label{ux793eux4f1aux683cux5deeux3068ux533bux7642}}

\begin{enumerate}
\def\labelenumi{\arabic{enumi}.}
\tightlist
\item
  社会格差を解消するために社会に対して行動できる。
\end{enumerate}

\hypertarget{ux5065ux5eb7ux3068ux533bux7642}{%
\subsubsection{健康と医療}\label{ux5065ux5eb7ux3068ux533bux7642}}

\begin{enumerate}
\def\labelenumi{\arabic{enumi}.}
\tightlist
\item
  健康寿命を延ばすために働きかけを行うことができる。
\item
  バリアフリー等の障害と社会環境に関連する概念を理解した行動をとることができる。
\end{enumerate}

\hypertarget{ux30b8ux30a7ux30f3ux30c0ux30fcux3068ux533bux7642}{%
\subsubsection{ジェンダーと医療}\label{ux30b8ux30a7ux30f3ux30c0ux30fcux3068ux533bux7642}}

\begin{enumerate}
\def\labelenumi{\arabic{enumi}.}
\tightlist
\item
  ジェンダーの形成並びに性的指向及び性自認への配慮方法を理解している。
\item
  女性やLGBTQに対する差別等のジェンダー不平等をなくすために積極的な行動をとることができる。
\end{enumerate}

\hypertarget{ux6c17ux5019ux5909ux52d5ux3068ux533bux7642}{%
\subsubsection{気候変動と医療}\label{ux6c17ux5019ux5909ux52d5ux3068ux533bux7642}}

\begin{enumerate}
\def\labelenumi{\arabic{enumi}.}
\tightlist
\item
  気候変動と医療との関係性を理解し、患者が抱える健康に関する課題と気候変動との関係を想像できる。
\item
  自然災害(新興感染症を含む)が起きた際に必要とされる医師の役割を理解している。
\item
  地球環境が抱える諸課題を認識し、その解決のために行動できる。
\end{enumerate}

\hypertarget{ux54f2ux5b66ux502bux7406ux3068ux533bux7642}{%
\subsubsection{哲学・倫理と医療}\label{ux54f2ux5b66ux502bux7406ux3068ux533bux7642}}

\begin{enumerate}
\def\labelenumi{\arabic{enumi}.}
\tightlist
\item
  現代思想・哲学の語彙を概説することができる。
\item
  診療現場における倫理的問題について、倫理学の考え方に依拠し、分析した上で、自身の考えを述べることができる。
\end{enumerate}

\hypertarget{ux6b74ux53f2ux3068ux533bux5b66ux533bux7642}{%
\subsubsection{歴史と医学・医療}\label{ux6b74ux53f2ux3068ux533bux5b66ux533bux7642}}

\begin{enumerate}
\def\labelenumi{\arabic{enumi}.}
\tightlist
\item
  医学・医療の歴史的変遷を踏まえ現代の医学的問題を相対化できる。
\end{enumerate}

\hypertarget{ux533bux7642ux7d4cux6e08}{%
\subsubsection{医療経済}\label{ux533bux7642ux7d4cux6e08}}

\begin{enumerate}
\def\labelenumi{\arabic{enumi}.}
\tightlist
\item
  経済が医療に与える影響について理解している。
\end{enumerate}

\hypertarget{ux56fdux5185ux5916ux306eux8996ux70b9ux304bux3089ux6349ux3048ux308bux533bux7642}{%
\subsection{国内外の視点から捉える医療}\label{ux56fdux5185ux5916ux306eux8996ux70b9ux304bux3089ux6349ux3048ux308bux533bux7642}}

国内、及び、国際社会の中で規定される医療の役割と医療体制について概説できる。

\hypertarget{ux56fdux5185ux306eux533bux7642ux8077ux306eux5f79ux5272ux3084ux533bux7642ux4f53ux5236}{%
\subsubsection{国内の医療職の役割や医療体制}\label{ux56fdux5185ux306eux533bux7642ux8077ux306eux5f79ux5272ux3084ux533bux7642ux4f53ux5236}}

\begin{enumerate}
\def\labelenumi{\arabic{enumi}.}
\tightlist
\item
  医師法が定める医師の職権と義務を理解している。
\item
  医療職を規定する法律・制度を説明できる。
\item
  医療法が定める医療施設の種類と機能について理解している。
\item
  医療計画について理解している。
\item
  地域医療提供体制に関する諸課題の相互関連性を理解している。
\item
  医療提供体制と医師の働き方について自身の考えを述べることができる。
\end{enumerate}

\hypertarget{ux30b0ux30ebux30fcux30d0ux30ebux30d8ux30ebux30b9ux306eux5f79ux5272ux3084ux533bux7642ux4f53ux5236}{%
\subsubsection{グルーバルヘルスの役割や医療体制}\label{ux30b0ux30ebux30fcux30d0ux30ebux30d8ux30ebux30b9ux306eux5f79ux5272ux3084ux533bux7642ux4f53ux5236}}

\begin{enumerate}
\def\labelenumi{\arabic{enumi}.}
\tightlist
\item
  国際的に取り組む必要のある医療・健康課題について、歴史・社会的背景を踏まえて、理解している。
\item
  ユニバーサル・ヘルス・カバレッジ(UHC)の意義を理解し、世界各国の医療制度が抱える問題を例示できる。
\item
  保健関連の国連開発目標や国際機関・国際協力に関わる組織・団体について概説できる。
\end{enumerate}

\hypertarget{ux793eux4f1aux79d1ux5b66ux306eux8996ux70b9ux304bux3089ux6349ux3048ux308bux533bux7642}{%
\subsection{社会科学の視点から捉える医療}\label{ux793eux4f1aux79d1ux5b66ux306eux8996ux70b9ux304bux3089ux6349ux3048ux308bux533bux7642}}

医学的・文化的・社会的文脈のなかで生成される健康観や人びとの言動・関係性を理解し、社会科学
(主に医療人類学・医療社会学)の視点・理論・方法から、それを臨床実践に活用することができる。

\hypertarget{ux793eux4f1aux79d1ux5b66ux3068ux533bux7642ux3068ux306eux95a2ux4fc2}{%
\subsubsection{社会科学と医療との関係}\label{ux793eux4f1aux79d1ux5b66ux3068ux533bux7642ux3068ux306eux95a2ux4fc2}}

\begin{enumerate}
\def\labelenumi{\arabic{enumi}.}
\tightlist
\item
  日常生活や外来診療・在宅療養・入院・施設入所等において、健康・病気・死の捉え方を探索できる。
\item
  時代の流れ、社会の状況や諸制度との関わりのなかで医療に関する諸事象を捉え、構造的に説明できる。
\item
  個や集団に及ぼす文化・慣習による影響(コミュニケーションのあり方等)を理解している。
  \newpage
\end{enumerate}

\hypertarget{ux5225ux8868}{%
\section{別表}\label{ux5225ux8868}}

\begin{xltabular}{\linewidth}{XX}
\caption{\label{tbl:physical}身体診察} \\
\toprule
診察項目 & id \\
\midrule
\endhead
頭部(顔貌、頭髪、頭皮、頭蓋)の診察 & TBL-physical-001 \\
眼(視野、瞳孔、対光反射、眼球運動・突出、結膜)の診察 & TBL-physical-002 \\
耳(耳介、聴力)の診察 & TBL-physical-003 \\
耳鏡を用いた外耳道、鼓膜の観察 & TBL-physical-004 \\
口唇、口腔、咽頭、扁桃の診察 & TBL-physical-005 \\
副鼻腔の診察 & TBL-physical-006 \\
鼻鏡を用いた前鼻腔の観察 & TBL-physical-007 \\
甲状腺、頸部血管、気管、唾液腺の診察 & TBL-physical-008 \\
頭頸部リンパ節の診察 & TBL-physical-009 \\
胸部の視診、触診、打診 & TBL-physical-010 \\
呼吸音と副雑音の聴診 & TBL-physical-011 \\
心音と心雑音の聴診 & TBL-physical-012 \\
腹部の視診、聴診(腸雑音、血管雑音)、打診、触診 & TBL-physical-013 \\
背部の叩打痛 & TBL-physical-014 \\
直腸(前立腺を含む)指診 & TBL-physical-015 \\
乳房の診察 & TBL-physical-016 \\
意識レベルの判定 & TBL-physical-017 \\
脳神経系の診察 & TBL-physical-018 \\
眼底検査 & TBL-physical-019 \\
腱反射の診察 & TBL-physical-020 \\
小脳機能・運動系の診察 & TBL-physical-021 \\
感覚系(痛覚、温度覚、触覚、深部感覚)の診察 & TBL-physical-022 \\
髄膜刺激所見 & TBL-physical-023 \\
四肢と脊柱(弯曲、疼痛)の診察 & TBL-physical-024 \\
関節(可動域、腫脹、疼痛、変形)の診察 & TBL-physical-025 \\
筋骨格系の診察(徒手筋力テスト) & TBL-physical-026 \\
\bottomrule
\end{xltabular}

\begin{xltabular}{\linewidth}{XX}
\caption{\label{tbl:labo}主要な臨床検査} \\
\toprule
検査項目 & id \\
\midrule
\endhead
血算 & TBL-labo-001 \\
生化学検査 & TBL-labo-002 \\
凝固・線溶検査 & TBL-labo-003 \\
免疫血清学検査 & TBL-labo-004 \\
尿検査 & TBL-labo-005 \\
便検査 & TBL-labo-006 \\
 & TBL-labo-007 \\
 & TBL-labo-008 \\
血液型(ABO、RhD)検査、血液交差適合(クロスマッチ)試験、不規則抗体検査 & TBL-labo-009 \\
動脈血ガス分析 & TBL-labo-010 \\
妊娠反応検査 & TBL-labo-011 \\
細菌学検査(細菌の塗抹、培養、同定、薬剤感受性試験) & TBL-labo-012 \\
脳脊髄液 & TBL-labo-013 \\
胸水検査 & TBL-labo-014 \\
腹水検査 & TBL-labo-015 \\
病理組織検査や細胞診検査(術中迅速診断を含む) & TBL-labo-016 \\
染色体・遺伝子検査 & TBL-labo-017 \\
心電図 & TBL-labo-018 \\
呼吸機能検査 & TBL-labo-019 \\
内分泌・代謝機能検査 & TBL-labo-020 \\
脳波検査 & TBL-labo-021 \\
超音波検査 & TBL-labo-022 \\
X線撮影 & TBL-labo-023 \\
CT検査 & TBL-labo-024 \\
MRI検査 & TBL-labo-025 \\
核医学検査 & TBL-labo-026 \\
内視鏡検査 & TBL-labo-027 \\
\bottomrule
\end{xltabular}

\begin{xltabular}{\linewidth}{XXXX}
\caption{\label{tbl:skills}基本的臨床手技} \\
\toprule
分類 & 基本的臨床手技 & 目標レベル & id \\
\midrule
\endhead
一般手技 & 体位交換、移送 & 実施できる & TBL-skills-001 \\
 & 気道内吸引 & 実施できる & TBL-skills-002 \\
 & 静脈採血 & 実演できる & TBL-skills-003 \\
 & 末梢静脈の血管確保 & 実演できる & TBL-skills-004 \\
 & 動脈血採血 & 実演できる & TBL-skills-005 \\
 & 腰椎穿刺 & 実演できる & TBL-skills-006 \\
 & 胃管の挿入と抜去 & 実演できる & TBL-skills-007 \\
 & 尿道カテーテルの挿入と抜去 & 実演できる & TBL-skills-008 \\
 & 皮内注射 & 実演できる & TBL-skills-009 \\
 & 皮下注射 & 実演できる & TBL-skills-010 \\
 & 筋肉注射 & 実演できる & TBL-skills-011 \\
 & 静脈内注射 & 実演できる & TBL-skills-012 \\
検査手技 & 微生物学検査(Gram 染色を含む) & 実施できる & TBL-skills-013 \\
 & 12 誘導心電図の記録 & 実施できる & TBL-skills-014 \\
 & 臨床判断のための簡易エコー(FAST含む) & 実演できる & TBL-skills-015 \\
 & 病原体抗原の迅速検査 & 実演できる & TBL-skills-016 \\
 & 簡易血糖測定 & 実演できる & TBL-skills-017 \\
外科手技 & 清潔操作 & 実演できる & TBL-skills-018 \\
 & 手術や手技のための手洗い & 実施できる & TBL-skills-019 \\
 & 手術室におけるガウンテクニック & 実施できる & TBL-skills-020 \\
 & 基本的な縫合と抜糸 & 実演できる & TBL-skills-021 \\
\bottomrule
\end{xltabular}

\begin{xltabular}{\linewidth}{XX}
\caption{\label{tbl:departments}主要診療科} \\
\toprule
診療科 & id \\
\midrule
\endhead
総合診療科 & TBL-departments-001 \\
救急科 & TBL-departments-002 \\
内科 & TBL-departments-003 \\
外科 & TBL-departments-004 \\
小児科 & TBL-departments-005 \\
産婦人科 & TBL-departments-006 \\
精神科 & TBL-departments-007 \\
皮膚科 & TBL-departments-008 \\
整形外科 & TBL-departments-009 \\
眼科 & TBL-departments-010 \\
耳鼻咽喉科 & TBL-departments-011 \\
泌尿器科 & TBL-departments-012 \\
脳神経外科 & TBL-departments-013 \\
放射線科 & TBL-departments-014 \\
麻酔科 & TBL-departments-015 \\
病理診断科 & TBL-departments-016 \\
臨床検査科 & TBL-departments-017 \\
形成外科 & TBL-departments-018 \\
リハビリテーション科 & TBL-departments-019 \\
歯科口腔外科 & TBL-departments-020 \\
\bottomrule
\end{xltabular}

\begin{xltabular}{\linewidth}{XXX}
\caption{\label{tbl:symptoms}主要症候} \\
\toprule
主要症候 & 検討すべき鑑別疾患 & id \\
\midrule
\endhead
発熱 & 髄膜炎,上気道炎,扁桃炎,肺炎,結核,急性副鼻腔炎,尿路感染症,胆嚢炎,胆管炎,薬剤性,インフルエンザ,蜂巣炎,感染性心内膜炎 & TBL-symptoms-001 \\
全身倦怠感 & 結核,肝炎,心不全,うつ病,甲状腺機能低下症,鉄欠乏性貧血 & TBL-symptoms-002 \\
食思(欲)不振 & 消化性潰瘍,急性肝炎,うつ病,急性副腎不全 & TBL-symptoms-003 \\
体重減少 & 心不全,ネフローゼ症候群,甲状腺機能低下症 & TBL-symptoms-004 \\
体重増加 & 悪性腫瘍全般,糖尿病,甲状腺機能亢進症,うつ病,慢性閉塞性肺疾患<COPD>,神経性食思<欲>不振症<拒食症> & TBL-symptoms-005 \\
意識障害 & くも膜下出血,頭蓋内血腫,脳炎,脳出血,脳梗塞,髄膜炎,薬物中毒,アルコール性中毒,てんかん,心筋梗塞,急性大動脈解離,急性消化管出血,敗血症,低血糖,ショック,CO2ナルコーシス,ナトリウム代謝異常 & TBL-symptoms-006 \\
失神 & 不整脈,弁膜症(大動脈弁膜症),てんかん,肺塞栓症 & TBL-symptoms-007 \\
けいれん & 脳梗塞,脳出血,脳炎,脳症,熱性けいれん,てんかん & TBL-symptoms-008 \\
めまい & 良性発作性頭位めまい症,脳出血,脳梗塞,Meniere病,前庭神経炎 & TBL-symptoms-009 \\
浮腫 & 深部静脈血栓症,心不全,ネフローゼ症候群,慢性腎臓病,肝硬変,甲状腺機能低下症,薬剤性,リンパ浮腫,血管性浮腫 & TBL-symptoms-010 \\
発疹 & ウイルス性発疹症(麻疹),ウイルス性発疹症(風疹),ウイルス性発疹症(水痘),ウイルス性発疹症(ヘルペス),蕁麻疹,薬疹,皮膚炎(アトピー性皮膚炎),湿疹,結節性紅斑,伝染性紅斑,帯状疱疹 & TBL-symptoms-011 \\
咳・痰 & 上気道炎,副鼻腔炎,気管支炎,肺炎,肺結核,肺癌,間質性肺疾患,薬剤性,気管支喘息,アレルギー性鼻炎,胃食道逆流症<GERD>,感冒<かぜ症候群>,百日咳 & TBL-symptoms-012 \\
血痰・喀血 & 肺結核,肺癌,気管支拡張症 & TBL-symptoms-013 \\
呼吸困難 & 肺塞栓症,急性呼吸促(窮)迫症候群<ARDS>,気管支喘息,慢性閉塞性肺疾患<COPD>,肺炎,間質性肺疾患,肺結核,緊張性気胸,自然気胸,心不全,アナフィラキシー,急性喉頭蓋炎,窒息 & TBL-symptoms-014 \\
胸痛 & 肺塞栓症,気胸,急性冠症候群,急性大動脈解離,大動脈瘤破裂,パニック障害,帯状疱疹,胸膜炎,急性心膜炎 & TBL-symptoms-015 \\
動悸 & 不整脈,甲状腺機能亢進症,鉄欠乏性貧血,二次性貧血,パニック障害,不安障害 & TBL-symptoms-016 \\
嚥下困難 & 脳出血,脳梗塞,扁桃炎,食道癌 & TBL-symptoms-017 \\
腹痛 & 消化性潰瘍,機能性ディスペプシア<FD>,急性胃腸炎,急性虫垂炎,便秘症,汎発性腹膜炎,過敏性腸症候群,腸閉塞,腸重積症,鼠径ヘルニア,胆嚢炎,胆石症,急性膵炎,急性冠症候群,急性大動脈解離,流・早産,卵巣嚢腫(捻転),卵巣癌(捻転),子宮内膜症,尿路結石,憩室炎,虚血性大腸炎,腸間膜動脈塞栓症,異所性妊娠,糖尿病性ケトアシドーシス & TBL-symptoms-018 \\
悪心・嘔吐 & 急性胃腸炎,急性虫垂炎,腸閉塞,食中毒,脳出血,片頭痛,くも膜下出血,頭蓋内血腫,髄膜炎,急性心筋梗塞,妊娠,糖尿病性ケトアシドーシス,カルシウム代謝異常 & TBL-symptoms-019 \\
吐血 & 食道静脈瘤,胃,消化性潰瘍,胃癌,Mallory-Weiss症候群 & TBL-symptoms-020 \\
下血 & 消化性潰瘍,炎症性腸疾患,大腸癌,痔核,裂肛,虚血性大腸炎,憩室出血 & TBL-symptoms-021 \\
便秘 & 便秘症,過敏性腸症候群,甲状腺機能低下症,薬剤性,Parkinson病,腸閉塞,大腸癌 & TBL-symptoms-022 \\
下痢 & 急性胃腸炎,炎症性腸疾患,過敏性腸症候群,甲状腺機能亢進症,薬剤性 & TBL-symptoms-023 \\
黄疸 & 急性肝炎,慢性肝炎,肝硬変,肝癌,胆管炎,膵癌,薬剤性,溶血性貧血,薬剤性,胆管癌,生理的黄疸 & TBL-symptoms-024 \\
腹部膨隆・腫瘤 & 腸閉塞,鼠径ヘルニア,肝硬変,妊娠 & TBL-symptoms-025 \\
リンパ節腫脹 & 扁桃炎,ウイルス性発疹症(風疹),結核,悪性リンパ腫,その他の悪性腫瘍全般,伝染性単核{球}症 & TBL-symptoms-026 \\
尿量・排尿の異常 & 糖尿病,薬剤性,蓄尿障害,尿路感染症,前立腺肥大症,過活動膀胱,神経因性膀胱 & TBL-symptoms-027 \\
血尿 & 糸球体腎炎症候群,腎細胞癌,尿路結石,尿路感染症,膀胱癌 & TBL-symptoms-028 \\
月経異常 & 妊娠,薬剤性,月経困難症,子宮内膜症,子宮体癌,更年期障害 & TBL-symptoms-029 \\
不安・抑うつ & うつ病,双極性障害,不安障害,甲状腺機能亢進症,悪性腫瘍全般,認知症,Parkinson病,甲状腺機能低下症,悪性腫瘍全般,薬剤性,適応障害 & TBL-symptoms-030 \\
認知障害 & 脳梗塞,認知症,Parkinson 病,うつ病,甲状腺機能低下症,薬剤性,正常圧水頭症,慢性硬膜下血腫 & TBL-symptoms-031 \\
頭痛 & 緊張型頭痛,片頭痛,薬剤性,髄膜炎,脳出血,くも膜下出血,緑内障,急性副鼻腔炎,群発頭痛,巨細胞性動脈炎<側頭動脈炎> & TBL-symptoms-032 \\
運動麻痺・筋力低下 & 脳梗塞,一過性脳虚血発作,脳出血,頭蓋内血腫,てんかん,脊髄損傷,椎間板ヘルニア,多発性筋炎,皮膚筋炎,筋萎縮性側索硬化症,Guillain-Barre症候群,カリウム代謝異常 & TBL-symptoms-033 \\
歩行障害 & 脳出血,頭蓋内血腫,脳梗塞,Parkinson病,変形性脊椎症,,脊柱管狭窄症,椎間板ヘルニア,変形性関節症,骨折 & TBL-symptoms-034 \\
感覚障害 & 脊柱管狭窄症,椎間板ヘルニア,糖尿病,多発神経炎 & TBL-symptoms-035 \\
腰背部痛 & 急性大動脈解離,急性膵炎,膵癌,尿管結石,椎間板ヘルニア,変形性脊椎症,脊柱管狭窄症,脊椎圧迫骨折,急性腰痛症,化膿性脊椎炎 & TBL-symptoms-036 \\
関節痛・関節腫脹 & 痛風,外傷,関節リウマチ,全身性エリテマトーデス<SLE>,偽痛風,反応性関節炎,化膿性関節炎 & TBL-symptoms-037 \\
\bottomrule
\end{xltabular}

\begin{xltabular}{\linewidth}{XXXXX}
\caption{\label{tbl:diseases}疾患} \\
\toprule
臓器 & カテゴリ & サブカテゴリ & 疾患 & id \\
\midrule
\endhead
血液・造血器・リンパ系 & 貧血 &  & 鉄欠乏性貧血 & TBL-diseases-001 \\
 &  &  & 慢性疾患に伴う貧血 & TBL-diseases-002 \\
 &  &  & 発作性夜間ヘモグロビン尿症 & TBL-diseases-003 \\
 &  & 二次性貧血 & 再生不良性貧血 & TBL-diseases-004 \\
 &  & 造血不全症 & 赤芽球癆 & TBL-diseases-005 \\
 &  &  & 骨髄異形成症候群 & TBL-diseases-006 \\
 &  &  & 自己免疫性 & TBL-diseases-007 \\
 &  &  & 薬剤誘発性 & TBL-diseases-008 \\
 &  & 溶血性貧血 & 出血性貧血 & TBL-diseases-009 \\
 &  &  & 腎性貧血 & TBL-diseases-010 \\
 &  & 巨赤芽球性貧血 & ビタミンB12欠乏性貧血 & TBL-diseases-011 \\
 &  &  & 葉酸欠乏性貧血 & TBL-diseases-012 \\
 &  & 遺伝性貧血 & サラセミア & TBL-diseases-013 \\
 &  &  & 遺伝性球状赤血球症 & TBL-diseases-014 \\
 &  &  & 鎌状赤血球症 & TBL-diseases-015 \\
 & 出血傾向 &  & 免疫性血小板減少症 <ITP> & TBL-diseases-016 \\
 &  &  & 脾機能亢進症 & TBL-diseases-017 \\
 &  &  & 薬剤性 & TBL-diseases-018 \\
 &  &  & 血友病 & TBL-diseases-019 \\
 &  &  & 播種性血管内凝固 <DIC> & TBL-diseases-020 \\
 &  &  & 溶血性尿毒症症候群 <HUS> & TBL-diseases-021 \\
 &  &  & 血栓性血小板減少性紫斑病 <TTP> & TBL-diseases-022 \\
 &  & 二次性血小板減少症 & ビタミンK欠乏症 & TBL-diseases-023 \\
 &  &  & von Willebrand病 & TBL-diseases-024 \\
 & 血栓傾向 & 先天性 & プロテインC・プロテインS・アンチトロンビン欠乏症 & TBL-diseases-025 \\
 &  &  & 抗リン脂質抗体症候群 & TBL-diseases-026 \\
 &  &  & TTP & TBL-diseases-027 \\
 &  &  & HUS & TBL-diseases-028 \\
 &  &  & DIC & TBL-diseases-029 \\
 & 腫瘍性疾患 &  & 急性骨髄性白血病 & TBL-diseases-030 \\
 &  &  & 急性リンパ性白血病 & TBL-diseases-031 \\
 &  &  & 慢性骨髄性白血病 & TBL-diseases-032 \\
 &  &  & 真性赤血球増加症 & TBL-diseases-033 \\
 &  &  & 本態性血小板血症 & TBL-diseases-034 \\
 &  &  & 原発性骨髄線維症 & TBL-diseases-035 \\
 &  &  & 慢性リンパ性白血病 & TBL-diseases-036 \\
 &  &  & リンパ増殖性疾患 & TBL-diseases-037 \\
 &  & 骨髄増殖性疾患 & 成人T 細胞白血病 & TBL-diseases-038 \\
 &  &  & Hodgkinリンパ腫 & TBL-diseases-039 \\
 &  &  & 濾胞性リンパ腫 & TBL-diseases-040 \\
 &  &  & びまん性大細胞型B細胞リンパ腫 & TBL-diseases-041 \\
 &  & 悪性リンパ腫 & 末梢T細胞性リンパ腫 & TBL-diseases-042 \\
 &  &  & Burkittリンパ腫 & TBL-diseases-043 \\
 &  &  & MALTリンパ腫 & TBL-diseases-044 \\
 &  &  & 多発性骨髄腫 & TBL-diseases-045 \\
 &  &  & マクログロブリン血症 & TBL-diseases-046 \\
 &  &  & 意義不明の単クローン性免疫グロブリン症〈MGUS〉 ※腫瘍の項目にも掲載 & TBL-diseases-047 \\
 & その他の重要な造血系疾患 &  & 無顆粒球症 & TBL-diseases-048 \\
 &  &  & 血球貪食症候群 & TBL-diseases-049 \\
 &  &  & 移植片対宿主病 <GVHD> & TBL-diseases-050 \\
神経系 & 脳血管障害 &  & 脳出血 & TBL-diseases-051 \\
 &  &  & くも膜下出血 & TBL-diseases-052 \\
 &  &  & 頭蓋内血腫 & TBL-diseases-053 \\
 &  &  & 脳梗塞 & TBL-diseases-054 \\
 &  &  & 一過性脳虚血発作 & TBL-diseases-055 \\
 &  &  & 脳動脈瘤 & TBL-diseases-056 \\
 &  &  & 脳動静脈奇形 & TBL-diseases-057 \\
 &  &  & もやもや病 & TBL-diseases-058 \\
 & 感染性・炎症性疾患・脱髄性疾患 &  & 脳炎・髄膜炎 & TBL-diseases-059 \\
 &  &  & 脳症 & TBL-diseases-060 \\
 &  &  & 脳膿瘍 & TBL-diseases-061 \\
 &  &  & 多発性硬化症 & TBL-diseases-062 \\
 & 認知症と変性疾患 & 認知症 & Alzheimer型 & TBL-diseases-063 \\
 &  &  & Lewy小体型 & TBL-diseases-064 \\
 &  &  & 脳血管性 & TBL-diseases-065 \\
 &  &  & Parkinson病 & TBL-diseases-066 \\
 &  &  & 筋萎縮性側索硬化症 & TBL-diseases-067 \\
 &  &  & 多系統萎縮症 & TBL-diseases-068 \\
 & 末梢神経・神経 筋接合部・筋疾患 & ニューロパチー & 栄養障害 & TBL-diseases-069 \\
 &  &  & 中毒 & TBL-diseases-070 \\
 &  &  & 遺伝性 & TBL-diseases-071 \\
 &  &  & Guillain-Barré症候群 & TBL-diseases-072 \\
 &  &  & 顔面神経麻痺(Bell麻痺、Ramsay Hunt症候群を含む) & TBL-diseases-073 \\
 &  &  & 反回神経麻痺 & TBL-diseases-074 \\
 &  &  & 主な神経痛(三叉・坐骨神経痛) & TBL-diseases-075 \\
 &  &  & 重症筋無力症 & TBL-diseases-076 \\
 &  &  & 進行性筋ジストロフィー & TBL-diseases-077 \\
 &  &  & 周期性四肢麻痺 & TBL-diseases-078 \\
 & 発作性・機能性・自律神経系疾患 &  & 全般てんかん & TBL-diseases-079 \\
 &  &  & 局在関連てんかん & TBL-diseases-080 \\
 &  & 慢性頭痛 & 片頭痛 & TBL-diseases-081 \\
 &  &  & 緊張型頭痛 & TBL-diseases-082 \\
 & 頭部外傷 &  & 脳挫傷 & TBL-diseases-083 \\
 &  &  & 脳震盪 & TBL-diseases-084 \\
 &  &  & びまん性軸索損傷 & TBL-diseases-085 \\
 &  &  & 急性硬膜外血腫 & TBL-diseases-086 \\
 &  &  & 硬膜下血腫(急性・慢性) & TBL-diseases-087 \\
 &  &  & 頭蓋骨骨折 & TBL-diseases-088 \\
 &  &  & 頭部外傷後の高次機能障害 & TBL-diseases-089 \\
 & 小児領域 &  & 熱性けいれん & TBL-diseases-090 \\
 &  &  & 脳性麻痺 & TBL-diseases-091 \\
 &  &  & 水頭症 & TBL-diseases-092 \\
 & 腫瘍性疾患 &  & ※腫瘍の項目を参照 & TBL-diseases-093 \\
皮膚系 & 湿疹・皮膚炎 &  & 湿疹反応(湿疹三角) & TBL-diseases-094 \\
 &  &  & 接触皮膚炎 & TBL-diseases-095 \\
 &  &  & アトピー性皮膚炎 & TBL-diseases-096 \\
 &  &  & 脂漏性皮膚炎 & TBL-diseases-097 \\
 &  &  & 貨幣状湿疹 & TBL-diseases-098 \\
 &  &  & 皮脂欠乏性湿疹 & TBL-diseases-099 \\
 &  &  & 自家感作性皮膚炎 & TBL-diseases-100 \\
 &  &  & うっ滞性皮膚炎 & TBL-diseases-101 \\
 & 蕁麻疹、紅斑症、紅皮症と皮膚そう痒症 &  & 蕁麻疹 & TBL-diseases-102 \\
 &  &  & 血管性浮腫 & TBL-diseases-103 \\
 &  &  & 多形滲出性紅斑 & TBL-diseases-104 \\
 &  &  & 結節性紅斑 & TBL-diseases-105 \\
 &  &  & 環状紅斑 & TBL-diseases-106 \\
 &  &  & 紅皮症 & TBL-diseases-107 \\
 &  &  & 皮膚そう痒症 & TBL-diseases-108 \\
 & 紫斑・皮膚血流障害 &  & 紫斑(単純性・老人性) & TBL-diseases-109 \\
 &  & 皮膚血流障害 & 血栓性静脈炎 & TBL-diseases-110 \\
 &  &  & 網状皮斑 & TBL-diseases-111 \\
 & 薬疹・薬物障害 &  & 固定薬疹 & TBL-diseases-112 \\
 &  &  & Stevens-Johnson症候群 & TBL-diseases-113 \\
 &  &  & 中毒性表皮壊死症 <TEN> & TBL-diseases-114 \\
 &  &  & 薬剤性過敏症症候群 <DIHS> & TBL-diseases-115 \\
 & 水疱症と膿疱 & 自己免疫性水疱症 & 天疱瘡 & TBL-diseases-116 \\
 &  &  & 水疱性類天疱瘡 & TBL-diseases-117 \\
 &  & 膿疱症 & 掌蹠膿疱症 & TBL-diseases-118 \\
 &  &  & 膿疱性乾癬 & TBL-diseases-119 \\
 & 乾癬と角化症 &  & 尋常性乾癬 & TBL-diseases-120 \\
 &  &  & 扁平苔癬 & TBL-diseases-121 \\
 &  &  & Gibert 薔薇色粃糠疹 & TBL-diseases-122 \\
 &  &  & 魚鱗癬脂漏性角化症 & TBL-diseases-123 \\
 & 皮膚感染症 &  & 伝染性膿痂疹 & TBL-diseases-124 \\
 &  &  & せつ・癰 & TBL-diseases-125 \\
 &  &  & 毛嚢炎 & TBL-diseases-126 \\
 &  &  & ひょう疽 & TBL-diseases-127 \\
 &  &  & 丹毒 & TBL-diseases-128 \\
 &  &  & ブドウ球菌性熱傷様皮膚症候群 <SSSS> & TBL-diseases-129 \\
 &  &  & 蜂窩織炎 & TBL-diseases-130 \\
 &  &  & 壊死性筋膜炎 & TBL-diseases-131 \\
 &  &  & 皮膚真菌症(表在性・深在性) & TBL-diseases-132 \\
 &  &  & 皮膚抗酸菌症 & TBL-diseases-133 \\
 &  &  & 疥癬 & TBL-diseases-134 \\
 &  &  & 単純ヘルペス & TBL-diseases-135 \\
 &  & 皮膚ウィルス感染症 & 帯状疱疹 & TBL-diseases-136 \\
 &  &  & 尋常性疣贅 & TBL-diseases-137 \\
 &  &  & 伝染性軟属腫 & TBL-diseases-138 \\
 &  &  & 麻疹 & TBL-diseases-139 \\
 &  &  & 風疹 & TBL-diseases-140 \\
 &  &  & 水痘 & TBL-diseases-141 \\
 &  &  & 伝染性紅斑 & TBL-diseases-142 \\
 &  &  & 手足口病 & TBL-diseases-143 \\
 &  &  & 後天性免疫不全症候群 <AIDS>に伴う皮膚症状(梅毒・難治性ヘルペス・伝染性軟属腫・Kaposi肉腫など) & TBL-diseases-144 \\
 & 母斑性皮膚疾患 &  & 母斑 & TBL-diseases-145 \\
 &  & 母斑症 & 神経線維腫症1型 & TBL-diseases-146 \\
 &  &  & 結節性硬化症 & TBL-diseases-147 \\
 & 付属器疾患 & 毛の疾患 & 円形脱毛症 & TBL-diseases-148 \\
 &  &  & 男性型脱毛症 & TBL-diseases-149 \\
 &  &  & 爪の疾患(匙状爪・嵌入爪) & TBL-diseases-150 \\
 &  & 汗腺の疾患 & 汗疹 & TBL-diseases-151 \\
 &  &  & 多汗症 & TBL-diseases-152 \\
 &  &  & 無汗症 & TBL-diseases-153 \\
 & その他 &  & 尋常性痤瘡 & TBL-diseases-154 \\
 &  &  & 酒皶様皮膚炎 & TBL-diseases-155 \\
 &  &  & 褥瘡 & TBL-diseases-156 \\
 &  &  & ケロイド & TBL-diseases-157 \\
 &  &  & 粉瘤 & TBL-diseases-158 \\
 &  &  & 尋常性白斑 & TBL-diseases-159 \\
 &  &  & 壊疽性膿皮症 & TBL-diseases-160 \\
 &  &  & 痒疹 & TBL-diseases-161 \\
 &  &  & 色素性乾皮症 & TBL-diseases-162 \\
 & 腫瘍性疾患 &  & ※腫瘍の項目を参照 & TBL-diseases-163 \\
 &  &  & 凍傷・電撃傷 & TBL-diseases-164 \\
運動器(筋骨格)系 &  &  & 四肢・脊椎外傷 & TBL-diseases-165 \\
 &  &  & 関節の脱臼 & TBL-diseases-166 \\
 &  &  & 肘内障 & TBL-diseases-167 \\
 &  &  & 腱・靱帯・半月板損傷 & TBL-diseases-168 \\
 &  &  & 関節拘縮 & TBL-diseases-169 \\
 &  &  & Dupuytren拘縮 & TBL-diseases-170 \\
 &  &  & 骨折 & TBL-diseases-171 \\
 &  &  & 筋損傷・挫滅症候群・コンパートメント症候群 & TBL-diseases-172 \\
 &  &  & 骨粗鬆症 & TBL-diseases-173 \\
 &  &  & くる病 & TBL-diseases-174 \\
 &  &  & 骨軟化症 & TBL-diseases-175 \\
 &  &  & 関節炎 & TBL-diseases-176 \\
 &  &  & 腱鞘炎 & TBL-diseases-177 \\
 &  &  & 滑液包炎 & TBL-diseases-178 \\
 &  &  & 腱付着部炎 & TBL-diseases-179 \\
 &  &  & 変形性関節症 & TBL-diseases-180 \\
 &  &  & 外反母趾 & TBL-diseases-181 \\
 &  &  & 外反膝 & TBL-diseases-182 \\
 &  &  & 内反膝 & TBL-diseases-183 \\
 &  & 代謝性骨疾患 & 反張膝 & TBL-diseases-184 \\
 &  &  & 神経病性関節症 & TBL-diseases-185 \\
 &  &  & 胸郭出口症候群 & TBL-diseases-186 \\
 &  & 関節症 & 手根管症候群 & TBL-diseases-187 \\
 &  &  & 肘部管症候群等 & TBL-diseases-188 \\
 &  &  & 脊椎症・脊髄症・神経根症(脊柱靭帯骨化症を含む) & TBL-diseases-189 \\
 &  &  & 脊髄損傷 & TBL-diseases-190 \\
 &  &  & 脊椎椎間板ヘルニア & TBL-diseases-191 \\
 &  &  & 脊柱管狭窄症 & TBL-diseases-192 \\
 &  & 絞扼性末梢神経障害 & 脊椎分離・すべり症 & TBL-diseases-193 \\
 &  &  & 腰背部痛 & TBL-diseases-194 \\
 &  &  & 頸部痛 & TBL-diseases-195 \\
 &  & 運動器慢性疼痛 & 肩こり & TBL-diseases-196 \\
 &  &  & 肩関節周囲炎 & TBL-diseases-197 \\
 &  &  & 化膿性関節炎 & TBL-diseases-198 \\
 &  &  & 骨髄炎 & TBL-diseases-199 \\
 &  & 感染性疾患 & 椎間板炎・化膿性脊椎炎・脊椎カリエス & TBL-diseases-200 \\
 &  &  & 斜頸 & TBL-diseases-201 \\
 &  & 先天性疾患 & 側弯症 & TBL-diseases-202 \\
 &  &  & 先天性股関節脱臼 & TBL-diseases-203 \\
 &  &  & 内反足 & TBL-diseases-204 \\
 &  &  & 骨形成不全症 & TBL-diseases-205 \\
 &  &  & 特発性大腿骨頭壊死症 & TBL-diseases-206 \\
 &  & 骨壊死・骨端症・軟骨の障害 & Perthes病 & TBL-diseases-207 \\
 &  &  & Osgood-Schlatter病 & TBL-diseases-208 \\
 &  &  & 離断性骨軟骨炎 & TBL-diseases-209 \\
 &  &  & 膝蓋軟骨軟化症 & TBL-diseases-210 \\
 &  &  & 腫瘍性疾患 ※腫瘍の項目を参照 & TBL-diseases-211 \\
循環器系 & 心不全 & 心不全 & 左心 & TBL-diseases-212 \\
 &  &  & 右心 & TBL-diseases-213 \\
 &  &  & 急性 & TBL-diseases-214 \\
 &  &  & 慢性 & TBL-diseases-215 \\
 & 虚血性心疾患 & 狭心症 & 労作性 & TBL-diseases-216 \\
 &  &  & 冠攣縮性 & TBL-diseases-217 \\
 &  & 急性冠症候群 & 不安定狭心症 & TBL-diseases-218 \\
 &  &  & 急性心筋梗塞 & TBL-diseases-219 \\
 & 不整脈 & 徐脈性不整脈 & 洞不全症候群 & TBL-diseases-220 \\
 &  &  & 房室ブロック & TBL-diseases-221 \\
 &  & 上室性頻脈性不整脈 & 心房細動 & TBL-diseases-222 \\
 &  &  & 心房粗動 & TBL-diseases-223 \\
 &  &  & 発作性上室頻拍症 & TBL-diseases-224 \\
 &  & 心室性頻脈性不整脈 & 心室頻拍 & TBL-diseases-225 \\
 &  &  & 多源性心室頻拍(torsades de pointes) & TBL-diseases-226 \\
 &  &  & 心室細動 & TBL-diseases-227 \\
 &  &  & 上室性 & TBL-diseases-228 \\
 &  &  & 心室性 & TBL-diseases-229 \\
 &  & 期外収縮 & WPW症候群 & TBL-diseases-230 \\
 &  &  & Brugada症候群 & TBL-diseases-231 \\
 & 弁膜症 & 僧帽弁疾患 & 狭窄 & TBL-diseases-232 \\
 &  &  & 閉鎖不全 & TBL-diseases-233 \\
 &  & 大動脈弁疾患 & 狭窄 & TBL-diseases-234 \\
 &  &  & 閉鎖不全 & TBL-diseases-235 \\
 &  &  & 三尖弁閉鎖不全 & TBL-diseases-236 \\
 & 心筋・心膜疾患 & 特発性心筋症 & 肥大型 & TBL-diseases-237 \\
 &  &  & 拡張型 & TBL-diseases-238 \\
 &  &  & 拘束型 & TBL-diseases-239 \\
 &  &  & 二次性心筋疾患 & TBL-diseases-240 \\
 &  &  & 急性心筋炎 & TBL-diseases-241 \\
 &  &  & 感染性心内膜炎 & TBL-diseases-242 \\
 &  &  & 急性心膜炎 & TBL-diseases-243 \\
 &  &  & 収縮性心膜炎 & TBL-diseases-244 \\
 &  &  & 心タンポナーデ & TBL-diseases-245 \\
 & 先天性心疾患 &  & 心房中隔欠損症 & TBL-diseases-246 \\
 &  &  & 心室中隔欠損症 & TBL-diseases-247 \\
 &  &  & 動脈管開存 & TBL-diseases-248 \\
 &  &  & Fallot 四徴症 & TBL-diseases-249 \\
 & 動脈疾患 &  & 動脈硬化症 & TBL-diseases-250 \\
 &  &  & 急性大動脈解離 & TBL-diseases-251 \\
 &  &  & 大動脈瘤(胸部・腹部) & TBL-diseases-252 \\
 &  &  & 閉塞性動脈硬化症 & TBL-diseases-253 \\
 &  &  & Buerger病 & TBL-diseases-254 \\
 &  &  & 高安動脈炎(大動脈炎症候群) & TBL-diseases-255 \\
 & 静脈・リンパ管疾患 &  & 深部静脈血栓症 & TBL-diseases-256 \\
 &  &  & 血栓性静脈炎 & TBL-diseases-257 \\
 &  &  & 上大静脈症候群 & TBL-diseases-258 \\
 &  &  & 下肢静脈瘤 & TBL-diseases-259 \\
 &  &  & リンパ浮腫 & TBL-diseases-260 \\
 & 高血圧症 & 高血圧症 & 本態性 & TBL-diseases-261 \\
 &  &  & 二次性 & TBL-diseases-262 \\
 &  &  & 高血圧緊急症 & TBL-diseases-263 \\
 & 低血圧症 &  & 起立性低血圧 & TBL-diseases-264 \\
 &  &  & 神経失調性失神 & TBL-diseases-265 \\
 & 腫瘍性疾患 &  & ※腫瘍の項目を参照 & TBL-diseases-266 \\
呼吸器系 & 呼吸不全 &  & 低酸素血症と高二酸化炭素血症 & TBL-diseases-267 \\
 & 呼吸器感染症 &  & 急性上気道感染症(かぜ症候群) & TBL-diseases-268 \\
 &  &  & 扁桃炎・急性喉頭蓋炎 & TBL-diseases-269 \\
 &  &  & 気管支炎 & TBL-diseases-270 \\
 &  &  & 細気管支炎 & TBL-diseases-271 \\
 &  &  & 肺炎(定型・非定型) & TBL-diseases-272 \\
 &  &  & 肺結核症 & TBL-diseases-273 \\
 &  &  & 非結核性(非定型)抗酸菌症 & TBL-diseases-274 \\
 &  &  & 肺真菌症 & TBL-diseases-275 \\
 &  &  & 嚥下性肺疾患 & TBL-diseases-276 \\
 &  &  & 肺化膿症・膿胸 & TBL-diseases-277 \\
 &  &  & クループ症候群 & TBL-diseases-278 \\
 & 閉塞性換気障害・拘束性換気障害 &  & 慢性閉塞性肺疾患 (COPD) & TBL-diseases-279 \\
 &  &  & 気腫性嚢胞(ブレブ・ブラ) & TBL-diseases-280 \\
 &  &  & 気管支喘息(咳喘息を含む) & TBL-diseases-281 \\
 &  &  & 特発性 & TBL-diseases-282 \\
 &  &  & 膠原病血管炎関連性 & TBL-diseases-283 \\
 &  & 間質性肺炎 & びまん性汎細気管支炎 & TBL-diseases-284 \\
 &  &  & 放射線肺炎 & TBL-diseases-285 \\
 &  & じん肺症 & 珪肺 & TBL-diseases-286 \\
 &  &  & 石綿肺 & TBL-diseases-287 \\
 & 肺循環障害 &  & 肺性心 & TBL-diseases-288 \\
 &  &  & 急性呼吸促(窮)迫症候群 <ARDS> & TBL-diseases-289 \\
 &  &  & 肺血栓塞栓症 & TBL-diseases-290 \\
 &  &  & 原発性 & TBL-diseases-291 \\
 &  & 肺高血圧症 & 二次性 & TBL-diseases-292 \\
 &  &  & 肺動静脈瘻 & TBL-diseases-293 \\
 & 免疫学的機序による肺疾患 &  & 過敏性肺炎 & TBL-diseases-294 \\
 &  &  & サルコイドーシス & TBL-diseases-295 \\
 &  &  & 好酸球性肺炎 & TBL-diseases-296 \\
 &  &  & 薬剤性肺炎 & TBL-diseases-297 \\
 &  &  & アレルギー性気管支肺アスペルギルス症 <ABPA> & TBL-diseases-298 \\
 & 異常呼吸 &  & 過換気症候群 & TBL-diseases-299 \\
 &  &  & 睡眠時無呼吸症候群 & TBL-diseases-300 \\
 &  &  & 肺胞低換気症候群 & TBL-diseases-301 \\
 & その他の機序による肺疾患 &  & 気管支拡張症 & TBL-diseases-302 \\
 &  &  & 無気肺 & TBL-diseases-303 \\
 &  &  & 新生児呼吸促迫症候群 <IRDS> & TBL-diseases-304 \\
 &  &  & 肺リンパ脈管筋腫症 <LAM> & TBL-diseases-305 \\
 &  &  & 肺胞タンパク症 & TBL-diseases-306 \\
 &  &  & 癌性リンパ管症 & TBL-diseases-307 \\
 & 胸膜・縦隔疾患・横隔膜 &  & 胸膜炎 & TBL-diseases-308 \\
 &  &  & 自然 & TBL-diseases-309 \\
 &  &  & 緊張性 & TBL-diseases-310 \\
 &  &  & 外傷性 & TBL-diseases-311 \\
 &  &  & 縦隔気腫 & TBL-diseases-312 \\
 &  &  & 血胸 & TBL-diseases-313 \\
 &  &  & 乳び胸 & TBL-diseases-314 \\
 &  &  & 縦隔炎 & TBL-diseases-315 \\
 &  & 気胸 & 胸郭変形(漏斗胸) & TBL-diseases-316 \\
 &  &  & 横隔神経麻痺 & TBL-diseases-317 \\
 &  &  & 横隔膜ヘルニア & TBL-diseases-318 \\
 & 腫瘍性疾患 &  & ※腫瘍の項目を参照 & TBL-diseases-319 \\
消化器系 & 食道疾患 &  & 食道・胃静脈瘤 & TBL-diseases-320 \\
 &  &  & 胃食道逆流症<GERD>  & TBL-diseases-321 \\
 &  &  & 逆流性食道炎 & TBL-diseases-322 \\
 &  &  & Mallory-Weiss 症候群 & TBL-diseases-323 \\
 &  &  & 食道アカラシア & TBL-diseases-324 \\
 & 胃・十二指腸疾患 &  & 消化性潰瘍(胃潰瘍・十二指腸潰瘍) & TBL-diseases-325 \\
 &  &  & Helicobacter pylori 感染症 & TBL-diseases-326 \\
 &  &  & 胃ポリープ & TBL-diseases-327 \\
 &  &  & 急性胃粘膜病変 & TBL-diseases-328 \\
 &  &  & 急性胃腸炎 & TBL-diseases-329 \\
 &  &  & 慢性胃炎 & TBL-diseases-330 \\
 &  &  & 胃切除後症候群 & TBL-diseases-331 \\
 &  &  & 機能性消化管障害(機能性ディスペプシア(FD)) & TBL-diseases-332 \\
 &  &  & 肥厚性幽門狭窄症 & TBL-diseases-333 \\
 &  &  & 胃アニサキス症 & TBL-diseases-334 \\
 & 小腸・大腸疾患 &  & 急性虫垂炎 & TBL-diseases-335 \\
 &  &  & 腸閉塞 & TBL-diseases-336 \\
 &  &  & 潰瘍性大腸炎 & TBL-diseases-337 \\
 &  &  & Crohn 病 & TBL-diseases-338 \\
 &  &  & 痔核・痔瘻 & TBL-diseases-339 \\
 &  &  & 機能性消化管障害(過敏性腸症候群) & TBL-diseases-340 \\
 &  &  & 大腸憩室炎 & TBL-diseases-341 \\
 &  &  & 大腸憩室出血 & TBL-diseases-342 \\
 &  &  & 薬物性腸炎 & TBL-diseases-343 \\
 &  &  & 消化管ポリポーシス & TBL-diseases-344 \\
 &  &  & 鎖肛 & TBL-diseases-345 \\
 &  &  & Hirschsprung 病 & TBL-diseases-346 \\
 &  &  & 腸重積症 & TBL-diseases-347 \\
 &  & 炎症性腸疾患 & 便秘症 & TBL-diseases-348 \\
 &  &  & 感染性腸炎 & TBL-diseases-349 \\
 &  & 腸管憩室症 & 乳児下痢症 & TBL-diseases-350 \\
 &  &  & 虚血性大腸炎 & TBL-diseases-351 \\
 &  & 先天性疾患 & 急性出血性直腸潰瘍 & TBL-diseases-352 \\
 &  &  & 上腸間膜動脈閉塞症 & TBL-diseases-353 \\
 & 胆道疾患 &  & 胆石症 & TBL-diseases-354 \\
 &  &  & 胆嚢炎 & TBL-diseases-355 \\
 &  &  & 胆管炎 & TBL-diseases-356 \\
 &  &  & 胆嚢ポリープ & TBL-diseases-357 \\
 &  &  & 先天性胆道拡張症 & TBL-diseases-358 \\
 &  &  & 膵・胆管合流異常症 & TBL-diseases-359 \\
 & 肝疾患 & 肝炎 & A 型 & TBL-diseases-360 \\
 &  &  & B 型 & TBL-diseases-361 \\
 &  &  & C 型 & TBL-diseases-362 \\
 &  &  & D 型 & TBL-diseases-363 \\
 &  &  & E 型 & TBL-diseases-364 \\
 &  &  & 急性 & TBL-diseases-365 \\
 &  &  & 慢性 & TBL-diseases-366 \\
 &  &  & 劇症型 & TBL-diseases-367 \\
 &  & 肝硬変および合併症 & 門脈圧亢進症 & TBL-diseases-368 \\
 &  &  & 肝性脳症 & TBL-diseases-369 \\
 &  &  & 肝癌 & TBL-diseases-370 \\
 &  &  & アルコール性肝障害 & TBL-diseases-371 \\
 &  &  & 薬物性肝障害 & TBL-diseases-372 \\
 &  &  & 肝膿瘍 & TBL-diseases-373 \\
 &  &  & 原発性胆汁性肝硬変<PBC> & TBL-diseases-374 \\
 &  &  & 原発性硬化性胆管炎 & TBL-diseases-375 \\
 &  &  & 自己免疫性肝炎 & TBL-diseases-376 \\
 &  &  & 脂肪肝 & TBL-diseases-377 \\
 & 膵臓疾患 &  & 急性膵炎(アルコール性・胆石性・特発性) & TBL-diseases-378 \\
 &  &  & 慢性膵炎(アルコール性・特発性) & TBL-diseases-379 \\
 &  &  & 自己免疫性膵炎 & TBL-diseases-380 \\
 & 腹膜・腹壁・横隔膜疾患 &  & 腹膜炎 & TBL-diseases-381 \\
 &  &  & ヘルニア(滑脱・嵌頓・絞扼) & TBL-diseases-382 \\
 &  &  & 鼠径部ヘルニア & TBL-diseases-383 \\
 & 腫瘍性疾患 &  & ※腫瘍の項目を参照 & TBL-diseases-384 \\
腎・尿路系(体液・電解質バランスを含む) & 腎機能の障害 &  & 急性腎障害 <AKI> & TBL-diseases-385 \\
 &  &  & 慢性腎臓病 <CKD> & TBL-diseases-386 \\
 &  &  & 慢性腎不全 & TBL-diseases-387 \\
 & 電解質異常 &  & 高・低ナトリウム 血症 & TBL-diseases-388 \\
 &  &  & 高・低カリウム血症 & TBL-diseases-389 \\
 &  &  & 高・低カルシウム 血症 & TBL-diseases-390 \\
 &  &  & 高・低リン血症 & TBL-diseases-391 \\
 &  &  & 高・低マグネシウム血症 & TBL-diseases-392 \\
 & 酸・塩基平衡障害 &  & アシドーシス(代謝性・呼吸性) & TBL-diseases-393 \\
 &  &  & アルカローシス(代謝性・呼吸性) & TBL-diseases-394 \\
 & 原発性糸球体疾患 &  & 急性糸球体腎炎 & TBL-diseases-395 \\
 &  &  & IgA腎症 & TBL-diseases-396 \\
 &  &  & 膜性腎症 & TBL-diseases-397 \\
 &  &  & 巣状分節性糸球体硬化症 & TBL-diseases-398 \\
 &  &  & 微小変化群 & TBL-diseases-399 \\
 &  &  & 膜性増殖性糸球体腎炎 & TBL-diseases-400 \\
 & 高血圧および腎血管障害 &  & 腎硬化症 & TBL-diseases-401 \\
 &  &  & 腎血管性高血圧症 & TBL-diseases-402 \\
 & 尿細管・間質性疾患 &  & 尿細管性アシドーシス & TBL-diseases-403 \\
 &  &  & 尿細管間質性腎炎(急性・慢性) & TBL-diseases-404 \\
 &  &  & 腎盂腎炎(急性・慢性) & TBL-diseases-405 \\
 & 全身性疾患による腎障害 &  & 糖尿病腎障害 & TBL-diseases-406 \\
 &  &  & IgA血管炎(紫斑病性腎炎) & TBL-diseases-407 \\
 &  &  & アミロイド腎症 & TBL-diseases-408 \\
 &  &  & ループス腎炎 & TBL-diseases-409 \\
 &  & 膠原病類縁疾患 & 血管炎症候群 & TBL-diseases-410 \\
 &  &  & 抗糸球体基底膜病 <抗GBM病> (Goodpasture症候群) & TBL-diseases-411 \\
 & 先天異常と外傷 &  & 多発性嚢胞腎 & TBL-diseases-412 \\
 &  &  & 膀胱尿管逆流 & TBL-diseases-413 \\
 &  &  & 腎外傷 & TBL-diseases-414 \\
 & 尿路疾患 &  & 尿路結石 & TBL-diseases-415 \\
 &  &  & 膀胱炎 & TBL-diseases-416 \\
 &  & 尿路の炎症 & 前立腺炎 & TBL-diseases-417 \\
 &  &  & 尿道炎 & TBL-diseases-418 \\
 &  &  & 神経因性膀胱 & TBL-diseases-419 \\
 & 腫瘍性疾患 &  & ※腫瘍の項目を参照 & TBL-diseases-420 \\
生殖機能 & 男性生殖器疾患 &  & 男性不妊症 & TBL-diseases-421 \\
 &  &  & 前立腺肥大症 & TBL-diseases-422 \\
 &  &  & 前立腺炎 & TBL-diseases-423 \\
 &  &  & 停留精巣 & TBL-diseases-424 \\
 &  &  & 陰嚢内腫瘤 & TBL-diseases-425 \\
 &  &  & 精巣捻転症 & TBL-diseases-426 \\
 & 女性生殖器疾患 &  & 内外生殖器の先天異常 & TBL-diseases-427 \\
 &  &  & 卵巣機能障害 & TBL-diseases-428 \\
 &  &  & 月経前症候群 & TBL-diseases-429 \\
 &  &  & 機能性月経困難症 & TBL-diseases-430 \\
 &  &  & 視床下部性無月経 & TBL-diseases-431 \\
 &  &  & 更年期障害 & TBL-diseases-432 \\
 &  &  & 不妊症 & TBL-diseases-433 \\
 &  &  & 子宮筋腫・子宮腺筋症 & TBL-diseases-434 \\
 &  & 月経異常 & 子宮内膜症 & TBL-diseases-435 \\
 &  &  & 外陰・腟と骨盤内感染症 & TBL-diseases-436 \\
 & 腫瘍性疾患(男性) &  & ※腫瘍の項目を参照 & TBL-diseases-437 \\
 & 腫瘍性疾患(女性) &  & ※腫瘍の項目を参照 & TBL-diseases-438 \\
妊娠と分娩 & 異常妊娠 &  & 妊娠悪阻 & TBL-diseases-439 \\
 &  &  & 異所性妊娠 & TBL-diseases-440 \\
 &  &  & 流産・切迫流産 & TBL-diseases-441 \\
 &  &  & ハイリスク妊娠 & TBL-diseases-442 \\
 &  &  & 妊娠高血圧症候群 & TBL-diseases-443 \\
 &  &  & 多胎妊娠 & TBL-diseases-444 \\
 &  &  & 前期破水 & TBL-diseases-445 \\
 &  &  & 切迫早産 & TBL-diseases-446 \\
 &  &  & 胎児機能不全 & TBL-diseases-447 \\
 & 異常分娩 &  & 早産 & TBL-diseases-448 \\
 &  &  & 微弱陣痛 & TBL-diseases-449 \\
 &  &  & 遷延分娩 & TBL-diseases-450 \\
 &  &  & 回旋異常 & TBL-diseases-451 \\
 &  &  & 前置胎盤 & TBL-diseases-452 \\
 &  &  & 癒着胎盤 & TBL-diseases-453 \\
 &  &  & 常位胎盤早期剥離 & TBL-diseases-454 \\
 &  &  & 分娩外傷 & TBL-diseases-455 \\
 & 異常産褥 &  & 子宮復古不全 & TBL-diseases-456 \\
 &  &  & 産褥熱 & TBL-diseases-457 \\
 &  &  & 乳腺炎 & TBL-diseases-458 \\
 & 産科出血 &  & 弛緩出血 & TBL-diseases-459 \\
 &  &  & 羊水塞栓症 & TBL-diseases-460 \\
 &  &  & 播種性血管内凝固 <DIC>  & TBL-diseases-461 \\
 & 合併症妊娠 &  & 貧血 & TBL-diseases-462 \\
 &  &  & 耐糖能異常 & TBL-diseases-463 \\
 &  &  & 甲状腺疾患 & TBL-diseases-464 \\
 &  &  & 免疫性血小板減少症 <ITP> & TBL-diseases-465 \\
 &  &  & HBV & TBL-diseases-466 \\
 &  & 感染症 & HCV & TBL-diseases-467 \\
 &  &  & HIV & TBL-diseases-468 \\
 &  &  & HTLV-Ⅰ & TBL-diseases-469 \\
 &  &  & パルボウイルスB19 & TBL-diseases-470 \\
 &  &  & B群連鎖球菌 & TBL-diseases-471 \\
 &  &  & TORCH症候群 & TBL-diseases-472 \\
小児 & 血液系 &  & 小児白血病・悪性リンパ腫 & TBL-diseases-473 \\
 &  &  & ビタミンK欠乏症 & TBL-diseases-474 \\
 & 神経系 &  & 熱性けいれん & TBL-diseases-475 \\
 &  &  & 脳性麻痺 & TBL-diseases-476 \\
 &  &  & 水頭症 & TBL-diseases-477 \\
 & 皮膚系 &  & 伝染性膿痂疹 ブドウ球菌性熱傷様皮膚症候群<SSSS> & TBL-diseases-478 \\
 &  &  & 伝染性軟属腫 & TBL-diseases-479 \\
 &  &  & 麻疹 & TBL-diseases-480 \\
 &  &  & 風疹 & TBL-diseases-481 \\
 &  &  & 水痘 & TBL-diseases-482 \\
 &  &  & 伝染性紅斑 & TBL-diseases-483 \\
 &  &  & 手足口病 & TBL-diseases-484 \\
 & 循環器系 &  & 心房中隔欠損症 & TBL-diseases-485 \\
 &  &  & 心室中隔欠損症 & TBL-diseases-486 \\
 &  &  & 動脈管開存 & TBL-diseases-487 \\
 &  &  & Fallot 四徴症 & TBL-diseases-488 \\
 & 呼吸器系 &  & クループ症候群 & TBL-diseases-489 \\
 &  &  & 細気管支炎 & TBL-diseases-490 \\
 &  &  & 小児気管支喘息 & TBL-diseases-491 \\
 & 消化器系 &  & 肥厚性幽門狭窄症 & TBL-diseases-492 \\
 &  &  & 鎖肛 & TBL-diseases-493 \\
 &  &  & Hirschsprung病 & TBL-diseases-494 \\
 &  &  & 腸重積症 & TBL-diseases-495 \\
 &  &  & 便秘症 & TBL-diseases-496 \\
 &  &  & 乳児下痢症 & TBL-diseases-497 \\
 &  &  & 胆道閉鎖症 & TBL-diseases-498 \\
 &  &  & 鼠径ヘルニア & TBL-diseases-499 \\
 & 腎臓 &  & 溶血性尿毒症症候群 <HUS> & TBL-diseases-500 \\
 &  &  & 小児ネフローゼ症候群 & TBL-diseases-501 \\
 &  &  & 紫斑病性腎炎 & TBL-diseases-502 \\
 &  &  & 先天性腎尿路奇形 & TBL-diseases-503 \\
 &  &  & 膀胱尿管逆流 & TBL-diseases-504 \\
 & 内分泌系・代謝 &  & 成長ホルモン分泌不全型低身長 & TBL-diseases-505 \\
 &  &  & 先天性副腎皮質過形成 & TBL-diseases-506 \\
 &  &  & アセトン血性嘔吐症 & TBL-diseases-507 \\
 & 精神系 & 神経発達症群 & 自閉スペクトラム症 <ASD> & TBL-diseases-508 \\
 &  &  & 注意・欠如多動症 <ADHD> & TBL-diseases-509 \\
 &  &  & 限局性学習症 & TBL-diseases-510 \\
 &  &  & チック症 & TBL-diseases-511 \\
 &  &  & 小児心身症 & TBL-diseases-512 \\
 & 腫瘍 &  & 網膜芽細胞腫 & TBL-diseases-513 \\
 &  &  & 神経芽腫 & TBL-diseases-514 \\
 &  &  & 腎芽腫 & TBL-diseases-515 \\
 &  &  & 胚芽腫 & TBL-diseases-516 \\
 &  &  & 奇形腫 & TBL-diseases-517 \\
 & 免疫・アレルギー &  & IgA血管炎 & TBL-diseases-518 \\
 &  &  & 川崎病 & TBL-diseases-519 \\
 & 染色体異常 &  & Down症 & TBL-diseases-520 \\
 & 新生児 &  & 新生児仮死 & TBL-diseases-521 \\
 &  &  & 新生児呼吸促迫症候群 <IRDS> & TBL-diseases-522 \\
 &  &  & 新生児黄疸(高ビリルビン血症) & TBL-diseases-523 \\
 &  &  & 早産低出生体重児 & TBL-diseases-524 \\
 &  &  & 胎便吸引症候群 & TBL-diseases-525 \\
 &  &  & 新生児一過性多呼吸 & TBL-diseases-526 \\
 & 救急 &  & 乳幼児突然死症候群 <SIDS> & TBL-diseases-527 \\
 &  &  & 被虐待児症候群 & TBL-diseases-528 \\
乳房 &  &  & 良性乳腺疾患(乳腺炎・乳腺症) & TBL-diseases-529 \\
 &  &  & 腫瘍性疾患 ※腫瘍の項目を参照 & TBL-diseases-530 \\
内分泌・栄養・代謝系 & 間脳・下垂体疾患 &  & 先端巨大症 & TBL-diseases-531 \\
 &  &  & Cushing病 & TBL-diseases-532 \\
 &  &  & 高プロラクチン血症 & TBL-diseases-533 \\
 &  &  & 下垂体前葉機能低下症 & TBL-diseases-534 \\
 &  &  & 視床下部下垂体炎 & TBL-diseases-535 \\
 &  &  & 中枢性尿崩症 & TBL-diseases-536 \\
 &  &  & 抗利尿ホルモン不適合分泌症候群(SIADH) & TBL-diseases-537 \\
 &  &  & 下垂体腫瘍 & TBL-diseases-538 \\
 & 甲状腺疾患 &  & 甲状腺機能亢進症 & TBL-diseases-539 \\
 &  &  & 甲状腺機能低下症 & TBL-diseases-540 \\
 &  &  & 甲状腺炎(慢性・無痛性・亜急性) & TBL-diseases-541 \\
 & 副甲状腺疾患 &  & 副甲状腺機能亢進症 & TBL-diseases-542 \\
 &  &  & 副甲状腺機能低下症 & TBL-diseases-543 \\
 &  &  & 悪性腫瘍に伴う高カルリウム血症 & TBL-diseases-544 \\
 & 副腎皮質・髄質疾患 &  & Cushing 症候群 & TBL-diseases-545 \\
 &  &  & アルドステロン過剰症 & TBL-diseases-546 \\
 &  &  & 原発性アルドステロン症 & TBL-diseases-547 \\
 &  &  & 副腎不全(急性・慢性(Addison 病)) & TBL-diseases-548 \\
 & 糖代謝異常 &  & 1型糖尿病 & TBL-diseases-549 \\
 &  &  & 2型糖尿病 & TBL-diseases-550 \\
 &  &  & 糖尿病ケトアシドーシス & TBL-diseases-551 \\
 &  &  & 高血糖高浸透圧症候群 & TBL-diseases-552 \\
 &  & 糖尿病の急性期合併症 & 乳酸アシドーシス & TBL-diseases-553 \\
 &  &  & 網膜症 & TBL-diseases-554 \\
 &  &  & 腎症 & TBL-diseases-555 \\
 &  & 糖尿病の慢性合併症 & 神経障害 & TBL-diseases-556 \\
 &  &  & 足病変 & TBL-diseases-557 \\
 &  &  & 低血糖症 & TBL-diseases-558 \\
 & 脂質代謝異常 &  & 脂質異常症 & TBL-diseases-559 \\
 &  &  & 肥満症 & TBL-diseases-560 \\
 & ビタミン・核酸・その他の代謝異常 &  & ビタミン欠乏症 & TBL-diseases-561 \\
 &  &  & 高尿酸血症・痛風 & TBL-diseases-562 \\
 &  &  & 全身性アミロイドーシス & TBL-diseases-563 \\
 & 小児疾患 &  & 成長ホルモン分泌不全型低身長 & TBL-diseases-564 \\
 &  &  & 先天性副腎皮質過形成 & TBL-diseases-565 \\
 & 腫瘍性疾患 &  & ※腫瘍の項目を参照 & TBL-diseases-566 \\
眼・視覚系 &  &  & 屈折異常(近視・遠視・乱視)と調節障害 & TBL-diseases-567 \\
 &  &  & 結膜炎・角膜炎・眼瞼炎 & TBL-diseases-568 \\
 &  &  & 麦粒腫・霰粒腫 & TBL-diseases-569 \\
 &  &  & 白内障 & TBL-diseases-570 \\
 &  &  & 緑内障 & TBL-diseases-571 \\
 &  &  & 裂孔原性網膜剥離 & TBL-diseases-572 \\
 &  &  & 加齢黄斑変性・網膜色素変性 & TBL-diseases-573 \\
 &  &  & 糖尿病・高血圧による眼底変化(糖尿病網膜症など) & TBL-diseases-574 \\
 &  &  & ぶどう膜炎 & TBL-diseases-575 \\
 &  &  & 視神経炎(症)・うっ血乳頭 & TBL-diseases-576 \\
 &  &  & 化学損傷(アルカリ・酸) & TBL-diseases-577 \\
 &  &  & 網膜静脈閉塞症と動脈閉塞症 & TBL-diseases-578 \\
 &  &  & 腫瘍性疾患 ※腫瘍の項目を参照 & TBL-diseases-579 \\
耳鼻・咽喉・口腔系 &  &  & 中耳炎(急性・慢性・滲出性・真珠腫性) & TBL-diseases-580 \\
 &  &  & 外耳炎 & TBL-diseases-581 \\
 &  &  & 耳せつ & TBL-diseases-582 \\
 &  &  & 難聴(騒音性・薬剤性・突発性・老人性) & TBL-diseases-583 \\
 &  &  & 乳幼児の難聴 & TBL-diseases-584 \\
 &  &  & めまい(末梢性・中枢性) & TBL-diseases-585 \\
 &  &  & 動揺病 & TBL-diseases-586 \\
 &  &  & 良性発作性頭位めまい症 & TBL-diseases-587 \\
 &  &  & Ménière病 & TBL-diseases-588 \\
 &  &  & 前庭神経炎 & TBL-diseases-589 \\
 &  &  & 鼻出血 & TBL-diseases-590 \\
 &  &  & 副鼻腔炎(急性・慢性) & TBL-diseases-591 \\
 &  &  & アレルギー性鼻炎 & TBL-diseases-592 \\
 &  &  & 鼻炎 & TBL-diseases-593 \\
 &  &  & 扁桃炎 & TBL-diseases-594 \\
 &  &  & 咽頭炎 & TBL-diseases-595 \\
 &  &  & 喉頭炎 & TBL-diseases-596 \\
 &  &  & 喉頭蓋炎 & TBL-diseases-597 \\
 &  &  & 扁桃周囲炎 & TBL-diseases-598 \\
 &  &  & 膿瘍 & TBL-diseases-599 \\
 &  &  & う蝕 & TBL-diseases-600 \\
 &  &  & 歯周病等の歯科疾患(全身への影響や口腔機能管理を含めて) & TBL-diseases-601 \\
 &  &  & 口角炎 & TBL-diseases-602 \\
 &  &  & 口内炎 & TBL-diseases-603 \\
 &  &  & 舌炎 & TBL-diseases-604 \\
 &  & 鼻咽喉頭の炎症性疾患 & 鵞口瘡 & TBL-diseases-605 \\
 &  &  & 白板症等 & TBL-diseases-606 \\
 &  &  & 外耳道・鼻腔・咽頭・喉頭・気管・食道の代表的な異物 & TBL-diseases-607 \\
 &  &  & 急性唾液腺炎 & TBL-diseases-608 \\
 &  &  & 唾石症 & TBL-diseases-609 \\
 &  &  & Sjögren症候群 & TBL-diseases-610 \\
 &  &  & Mikulicz病 & TBL-diseases-611 \\
 &  & 唾液腺疾患 & 顎関節症 & TBL-diseases-612 \\
 &  &  & 鼻骨骨折 & TBL-diseases-613 \\
 &  &  & 吹き抜け骨折 & TBL-diseases-614 \\
 &  &  & 耳介血腫 & TBL-diseases-615 \\
 &  & 耳鼻・咽頭・口腔系の外傷・損傷 & 鼓膜損傷等 & TBL-diseases-616 \\
 &  &  & 唇裂 & TBL-diseases-617 \\
 &  &  & 口蓋裂など & TBL-diseases-618 \\
 &  &  & 口蓋扁桃肥大症 & TBL-diseases-619 \\
 &  & 先天異常 & 咽頭扁桃(アデノイド)増殖症 & TBL-diseases-620 \\
 &  &  & 声帯ポリープ & TBL-diseases-621 \\
 &  & 頭頸部疾患 & 頸部リンパ節炎 & TBL-diseases-622 \\
 &  &  & 頸部膿瘍 & TBL-diseases-623 \\
 &  &  & 頸部リンパ節転移等 & TBL-diseases-624 \\
 & 腫瘍性疾患 &  & ※腫瘍の項目を参照 & TBL-diseases-625 \\
精神系 &  &  & 認知症 ※神経系の項目を参照 & TBL-diseases-626 \\
 &  &  & 症状性精神病 & TBL-diseases-627 \\
 &  &  & 依存症(薬物使用 & TBL-diseases-628 \\
 &  &  & アルコール & TBL-diseases-629 \\
 &  &  & ギャンブル) & TBL-diseases-630 \\
 &  &  & うつ病 & TBL-diseases-631 \\
 &  &  & 双極性障害(躁うつ病) & TBL-diseases-632 \\
 &  &  & 統合失調症 & TBL-diseases-633 \\
 &  &  & 強迫性障害 & TBL-diseases-634 \\
 &  &  & 不安障害(パニック障害・社交不安障害) & TBL-diseases-635 \\
 &  &  & 解離性障害 & TBL-diseases-636 \\
 &  &  & 身体表現性障害(身体化障害・疼痛性障害・心気症) & TBL-diseases-637 \\
 &  &  & 心身症 & TBL-diseases-638 \\
 &  &  & 急性ストレス障害 & TBL-diseases-639 \\
 &  &  & 心的外傷後ストレス障害 & TBL-diseases-640 \\
 &  &  & 過換気症候群 & TBL-diseases-641 \\
 &  & ストレス関連障害 & 摂食障害(神経性食思不振症(神経性やせ症)・神経性過食症) & TBL-diseases-642 \\
 &  &  & パーソナリティ障害 & TBL-diseases-643 \\
免疫・アレルギー &  &  & 関節リウマチ & TBL-diseases-644 \\
 &  &  & 成人スチル病、若年性特発性関節炎(JIA) & TBL-diseases-645 \\
 &  &  & 全身性エリテマトーデス(SLE)および合併症(中枢神経ループス、ループス腎炎、抗リン脂質抗体症候群) & TBL-diseases-646 \\
 &  &  & 全身性強皮症 & TBL-diseases-647 \\
 &  &  & 皮膚筋炎・多発性筋炎 & TBL-diseases-648 \\
 &  &  & 混合性結合組織病 & TBL-diseases-649 \\
 &  &  & Sjögren症候群 & TBL-diseases-650 \\
 &  &  & Behçet病 & TBL-diseases-651 \\
 &  &  & 全身性アレルギー性疾患 & TBL-diseases-652 \\
 &  &  & 後天性免疫不全症(AIDS) & TBL-diseases-653 \\
 &  &  & 原発性免疫不全症 & TBL-diseases-654 \\
 &  &  & 二次性免疫不全症候群(悪性腫瘍・医原性・自己免疫疾患による) & TBL-diseases-655 \\
 &  &  & 自己炎症性疾患 & TBL-diseases-656 \\
 & 他のリウマチ性疾患 &  & 強直性脊椎炎 & TBL-diseases-657 \\
 &  &  & 反応性関節炎 & TBL-diseases-658 \\
 &  &  & 乾癬性関節炎 & TBL-diseases-659 \\
 &  &  & 掌蹠膿疱症性関節炎 & TBL-diseases-660 \\
 &  &  & 結晶誘発性関節炎 & TBL-diseases-661 \\
 &  &  & 変形性関節症 & TBL-diseases-662 \\
 &  &  & リウマチ性多発筋痛症 & TBL-diseases-663 \\
 &  &  & 線維筋痛症 & TBL-diseases-664 \\
 &  &  & IgG4関連疾患 & TBL-diseases-665 \\
 &  &  & 再発性多発軟骨炎 & TBL-diseases-666 \\
 &  &  & リウマチ熱 & TBL-diseases-667 \\
 & 血管炎症候群 &  & 巨細胞性動脈炎 & TBL-diseases-668 \\
 &  &  & 高安動脈炎(大動脈炎症候群) & TBL-diseases-669 \\
 &  &  & 結節性多発動脈炎 & TBL-diseases-670 \\
 &  &  & 顕微鏡的多発血管炎 & TBL-diseases-671 \\
 &  &  & 多発血管炎性肉芽腫症 & TBL-diseases-672 \\
 &  &  & 好酸球性多発血管炎性肉芽腫症 & TBL-diseases-673 \\
 &  &  & IgA血管炎 & TBL-diseases-674 \\
 &  &  & 川崎病 & TBL-diseases-675 \\
 &  &  & 悪性関節リウマチ & TBL-diseases-676 \\
 &  &  & 抗GBM病 & TBL-diseases-677 \\
感染症 & 市中感染症 &  & 髄膜脳炎 & TBL-diseases-678 \\
 &  &  & 咽頭炎 & TBL-diseases-679 \\
 &  &  & 中耳炎 & TBL-diseases-680 \\
 &  &  & 血流感染・感染性心内膜炎 & TBL-diseases-681 \\
 &  &  & 肺炎 & TBL-diseases-682 \\
 &  &  & 腹腔内感染 & TBL-diseases-683 \\
 &  &  & 膀胱炎・腎盂腎炎 & TBL-diseases-684 \\
 &  &  & 皮膚軟部組織感染 & TBL-diseases-685 \\
 &  &  & 関節炎 & TBL-diseases-686 \\
 & 医療関連感染 &  & 血管内留置カテーテル関連感染 & TBL-diseases-687 \\
 &  &  & 尿路カテーテル感染 & TBL-diseases-688 \\
 &  &  & 医療関連肺炎・人工呼吸器関連肺炎 & TBL-diseases-689 \\
 &  &  & 手術部位感染 & TBL-diseases-690 \\
 &  &  & クロストリディオイデス・ディフィシル感染 & TBL-diseases-691 \\
 & 免疫不全 &  & 糖尿病 & TBL-diseases-692 \\
 &  &  & 腎臓病 & TBL-diseases-693 \\
 &  &  & 肝臓病 & TBL-diseases-694 \\
 &  &  & がん・血液疾患 & TBL-diseases-695 \\
 &  &  & 好中球減少 & TBL-diseases-696 \\
 &  &  & 免疫抑制薬使用中 & TBL-diseases-697 \\
 &  &  & HIV・AIDS & TBL-diseases-698 \\
 &  &  & 臓器移植 & TBL-diseases-699 \\
 & ワクチン予防可能な疾患 &  & 麻疹 & TBL-diseases-700 \\
 &  &  & 風疹 & TBL-diseases-701 \\
 &  &  & ムンプス & TBL-diseases-702 \\
 &  &  & 水痘 & TBL-diseases-703 \\
 &  &  & B型肝炎 & TBL-diseases-704 \\
 &  &  & インフルエンザ菌 & TBL-diseases-705 \\
 &  &  & 肺炎球菌 & TBL-diseases-706 \\
 &  &  & 破傷風ジフテリア & TBL-diseases-707 \\
 &  &  & インフルエンザ & TBL-diseases-708 \\
 &  &  & 新型コロナウイルス & TBL-diseases-709 \\
腫瘍 & 血液・造血器・リンパ系 &  & 急性白血病 & TBL-diseases-710 \\
 &  &  & 慢性骨髄性白血病 & TBL-diseases-711 \\
 &  &  & 骨髄異形成症候群 & TBL-diseases-712 \\
 &  &  & 成人T細胞白血病 & TBL-diseases-713 \\
 &  &  & 真性赤血球増加症 & TBL-diseases-714 \\
 &  &  & 本態性血小板血症 & TBL-diseases-715 \\
 &  &  & 骨髄線維症 & TBL-diseases-716 \\
 &  &  & 悪性リンパ腫 & TBL-diseases-717 \\
 &  &  & 多発性骨髄腫 & TBL-diseases-718 \\
 & 神経系 &  & 膠芽腫 & TBL-diseases-719 \\
 &  &  & 髄膜腫 & TBL-diseases-720 \\
 &  &  & 神経鞘腫 & TBL-diseases-721 \\
 &  &  & 転移性脳腫瘍 & TBL-diseases-722 \\
 & 皮膚系 &  & 基底細胞癌 & TBL-diseases-723 \\
 &  &  & 有棘細胞癌 & TBL-diseases-724 \\
 &  &  & 悪性黒色腫 & TBL-diseases-725 \\
 &  &  & リンパ腫 & TBL-diseases-726 \\
 & 運動器(筋骨格)系 & 骨軟部腫瘍 & 骨肉腫 & TBL-diseases-727 \\
 &  &  & 軟骨肉腫 & TBL-diseases-728 \\
 &  &  & Ewing肉腫 & TBL-diseases-729 \\
 &  &  & 転移性脊椎腫瘍 & TBL-diseases-730 \\
 & 循環器系 &  & 粘液腫 & TBL-diseases-731 \\
 & 呼吸器系 &  & 肺癌 & TBL-diseases-732 \\
 &  &  & 胸膜中皮腫 & TBL-diseases-733 \\
 &  &  & 転移性肺腫瘍 & TBL-diseases-734 \\
 &  &  & 縦隔腫瘍 & TBL-diseases-735 \\
 & 消化器系 &  & 食道癌 & TBL-diseases-736 \\
 &  &  & 胃癌 & TBL-diseases-737 \\
 &  &  & 大腸ポリープ & TBL-diseases-738 \\
 &  &  & 大腸癌 & TBL-diseases-739 \\
 &  &  & 胆嚢・胆管癌 & TBL-diseases-740 \\
 &  &  & 原発性肝癌 & TBL-diseases-741 \\
 &  &  & 膵神経内分泌腫瘍 & TBL-diseases-742 \\
 &  &  & 嚢胞性膵腫瘍 & TBL-diseases-743 \\
 &  &  & 膵癌 & TBL-diseases-744 \\
 & 腎・尿路系 &  & 腎癌 & TBL-diseases-745 \\
 &  &  & 腎盂尿管癌・膀胱癌 & TBL-diseases-746 \\
 & 生殖機能 &  & 前立腺癌 & TBL-diseases-747 \\
 &  &  & 精巣腫瘍 & TBL-diseases-748 \\
 &  &  & 子宮頸癌 & TBL-diseases-749 \\
 &  &  & 子宮体癌(子宮内膜癌) & TBL-diseases-750 \\
 &  &  & 卵巣腫瘍 & TBL-diseases-751 \\
 &  &  & 胞状奇胎 & TBL-diseases-752 \\
 &  & 絨毛性疾患 & 絨毛癌 & TBL-diseases-753 \\
 & 乳房 &  & 原発性乳癌 & TBL-diseases-754 \\
 &  &  & 線維腺腫 & TBL-diseases-755 \\
 &  &  & 乳腺症 & TBL-diseases-756 \\
 & 内分泌・栄養・代謝系 &  & 下垂体腫瘍 & TBL-diseases-757 \\
 &  &  & 腺腫様甲状腺腫 & TBL-diseases-758 \\
 &  &  & 甲状腺癌 & TBL-diseases-759 \\
 &  &  & 褐色細胞腫 & TBL-diseases-760 \\
 &  & 甲状腺腫瘍 & 多発性内分泌腫瘍症 & TBL-diseases-761 \\
 & 眼・視覚系 &  & 網膜芽細胞腫 & TBL-diseases-762 \\
 & 耳鼻・咽喉・口腔系(頭頸部) &  & 舌癌 & TBL-diseases-763 \\
 &  &  & 咽頭癌 & TBL-diseases-764 \\
 &  &  & 喉頭癌 & TBL-diseases-765 \\
 & 小児 &  & 脳腫瘍 & TBL-diseases-766 \\
 &  &  & 血液腫瘍 & TBL-diseases-767 \\
 &  &  & 網膜芽細胞腫 & TBL-diseases-768 \\
 &  &  & 神経芽腫 & TBL-diseases-769 \\
 &  &  & 腎芽腫 & TBL-diseases-770 \\
 &  &  & 肝芽腫 & TBL-diseases-771 \\
 &  &  & 奇形腫を含む胚細胞腫瘍 & TBL-diseases-772 \\
 & 遺伝性腫瘍 &  & 家族性大腸腺腫症 & TBL-diseases-773 \\
 &  &  & 遺伝性乳がん卵巣がん症候群 & TBL-diseases-774 \\
 &  &  & 遺伝性非ポリポーシス性大腸癌 <Lynch症候群> & TBL-diseases-775 \\
 &  &  & Li-Fraumeni症候群 & TBL-diseases-776 \\
 & オンコロジーエマージェンシー &  & 脊髄圧迫 & TBL-diseases-777 \\
 &  &  & 腫瘍崩壊 & TBL-diseases-778 \\
 &  &  & 上大静脈症候群 & TBL-diseases-779 \\
 &  &  & 代謝障害 & TBL-diseases-780 \\
 &  &  & 治療の有害事象等 & TBL-diseases-781 \\
\bottomrule
\end{xltabular}

\begin{xltabular}{\linewidth}{XXXX}
\caption{\label{tbl:knowledge}知識} \\
\toprule
臓器 & 分類 & 項目名 & id \\
\midrule
\endhead
血液・造血器・リンパ系 & 構造と機能 & 骨髄の構造 & TBL-knowledge-001 \\
 &  & 造血幹細胞から各血球への分化と成熟の過程 & TBL-knowledge-002 \\
 &  & 主な造血因子(エリスロポエチン、顆粒球コロニー刺激因子(G-CSF)、トロンボポエチン) & TBL-knowledge-003 \\
 &  & 脾臓、胸腺、リンパ節、扁桃とPeyer板の構造と機能 & TBL-knowledge-004 \\
 &  & 血漿タンパク質の種類と機能 & TBL-knowledge-005 \\
 &  & 赤血球とヘモグロビンの構造と機能 & TBL-knowledge-006 \\
 &  & 白血球の種類と機能 & TBL-knowledge-007 \\
 &  & 血小板の機能と止血や凝固・線溶の機序 & TBL-knowledge-008 \\
 & 症候 & 発熱 & TBL-knowledge-009 \\
 &  & 全身倦怠感 & TBL-knowledge-010 \\
 &  & 黄疸 & TBL-knowledge-011 \\
 &  & リンパ節腫脹 & TBL-knowledge-012 \\
 &  & 貧血 & TBL-knowledge-013 \\
 &  & 出血傾向 & TBL-knowledge-014 \\
 &  & 血栓傾向 & TBL-knowledge-015 \\
 & 検査方法 & 末梢血塗抹 & TBL-knowledge-016 \\
 &  & 凝固・線溶・血小板機能検査 & TBL-knowledge-017 \\
 &  & 骨髄検査(骨髄穿刺、骨髄生検) & TBL-knowledge-018 \\
 &  & 輸血関連検査 & TBL-knowledge-019 \\
 &  & タンパク分画、免疫電気泳動 & TBL-knowledge-020 \\
 &  & 遺伝子・染色体検査 & TBL-knowledge-021 \\
 & 特異的治療法 & 輸血 & TBL-knowledge-022 \\
 &  & 造血幹細胞移植 & TBL-knowledge-023 \\
神経系 & 構造と機能 & 中枢神経系と末梢神経系の構成 & TBL-knowledge-024 \\
 &  & 脳の血管支配と血液脳関門 & TBL-knowledge-025 \\
 &  & 脳のエネルギー代謝の特徴 & TBL-knowledge-026 \\
 &  & 主な脳内神経伝達物質(アセチルコリン・ドパミン・ノルアドレナリン)とその作用 & TBL-knowledge-027 \\
 &  & 髄膜・脳室系の構造と脳脊髄液の産生と循環 & TBL-knowledge-028 \\
 &  & 脊髄の構造、機能局在と伝導路 & TBL-knowledge-029 \\
 &  & 脊髄反射(伸張反射、屈筋反射)と筋の相反神経支配 & TBL-knowledge-030 \\
 &  & 脊髄神経と神経叢(頸・腕・腰仙骨)の構成および主な骨格筋支配と皮膚分布(デルマトーム) & TBL-knowledge-031 \\
 &  & 脳幹の構造と機能、および伝導路 & TBL-knowledge-032 \\
 &  & 脳神経の名称、核の局在、走行・分布と機能 & TBL-knowledge-033 \\
 &  & 大脳の構造と大脳皮質の機能局在(運動野・感覚野・言語野) & TBL-knowledge-034 \\
 &  & 辺縁系の構造と記憶・学習の機序との関連 & TBL-knowledge-035 \\
 &  & 錐体路を中心とした随意運動の発現機構 & TBL-knowledge-036 \\
 &  & 小脳の構造と機能 & TBL-knowledge-037 \\
 &  & 大脳基底核(線条体・淡蒼球・黒質)の線維結合と機能 & TBL-knowledge-038 \\
 &  & 痛覚、温度覚、触覚と深部感覚の受容機序と伝導路 & TBL-knowledge-039 \\
 &  & 視覚、聴覚・平衡覚、嗅覚、味覚の受容機序と伝導路 & TBL-knowledge-040 \\
 &  & 交感神経系と副交感神経系の中枢内局在、末梢分布、機能と伝達物質 & TBL-knowledge-041 \\
 &  & 内分泌および自律機能と関連づけた視床下部の構造と機能 & TBL-knowledge-042 \\
 &  & ストレス反応と本能・情動行動の発現機序 & TBL-knowledge-043 \\
 & 症候 & 頭痛 & TBL-knowledge-044 \\
 &  & めまい & TBL-knowledge-045 \\
 &  & けいれん & TBL-knowledge-046 \\
 &  & 意識障害 & TBL-knowledge-047 \\
 &  & 運動麻痺・筋力低下 & TBL-knowledge-048 \\
 &  & 歩行障害 & TBL-knowledge-049 \\
 &  & 感覚障害 & TBL-knowledge-050 \\
 &  & 認知障害 & TBL-knowledge-051 \\
 &  & 失語症・構音障害 & TBL-knowledge-052 \\
 &  & 振戦 & TBL-knowledge-053 \\
 &  & 小脳性・前庭性・感覚性運動失調障害 & TBL-knowledge-054 \\
 &  & 不随意運動(ミオクローヌス・舞踏運動・ジストニア・固定姿勢保持困難・アテトーシス・チック) & TBL-knowledge-055 \\
 &  & 頭蓋内圧亢進(急性・慢性) & TBL-knowledge-056 \\
 &  & 脳ヘルニア & TBL-knowledge-057 \\
 & 検査方法 & 脳・脊髄の画像検査(CT・MRI) & TBL-knowledge-058 \\
 &  & 神経系の電気生理学的検査(脳波検査、針筋電図検査、末梢神経伝導検査) & TBL-knowledge-059 \\
 & 特異的治療法 & 脳血管障害の急性期治療とリハビリテーション医療 & TBL-knowledge-060 \\
皮膚系 & 構造と機能 & 皮膚の組織構造 & TBL-knowledge-061 \\
 &  & 皮膚の細胞動態と角化の機構 & TBL-knowledge-062 \\
 &  & 皮膚の免疫防御能 & TBL-knowledge-063 \\
 & 症候 & 皮疹(紅斑・紫斑・色素斑・丘疹・結節・腫瘤・水疱・膿疱・嚢腫・びらん・潰瘍・毛細血管拡張・硬化・瘢痕・萎縮・鱗屑・痂皮・苔癬化・壊疽) & TBL-knowledge-064 \\
 &  & そう痒 & TBL-knowledge-065 \\
 &  & 粘膜疹 & TBL-knowledge-066 \\
 &  & 脱毛 & TBL-knowledge-067 \\
 & 検査方法 & 皮膚検査法(硝子圧法・皮膚描記法(Darier 徴候)・Nikolsky 現象・Tzanck 試験・光線テスト) & TBL-knowledge-068 \\
 &  & 皮膚アレルギー検査法(プリックテスト・皮内テスト・パッチテスト) & TBL-knowledge-069 \\
 &  & 微生物検査法(検体採取法・苛性カリ(KOH)直接検鏡法) & TBL-knowledge-070 \\
 &  & ダーモスコピー & TBL-knowledge-071 \\
 & 特異的治療法 & 外用療法 & TBL-knowledge-072 \\
 &  & 凍結療法 & TBL-knowledge-073 \\
 &  & 光線療法(PUVA療法) & TBL-knowledge-074 \\
運動器(筋骨格)系 & 構造と機能 & 骨・軟骨・関節・靱帯の構成と機能 & TBL-knowledge-075 \\
 &  & 頭頸部の構成 & TBL-knowledge-076 \\
 &  & 脊柱の構成と機能 & TBL-knowledge-077 \\
 &  & 四肢の骨格、主要筋群の運動と神経支配 & TBL-knowledge-078 \\
 &  & 骨盤の構成と性差 & TBL-knowledge-079 \\
 &  & 骨の成長と骨形成・吸収の機序 & TBL-knowledge-080 \\
 &  & 姿勢と体幹の運動にかかわる筋群 & TBL-knowledge-081 \\
 &  & 抗重力筋 & TBL-knowledge-082 \\
 & 症候 & 運動麻痺・筋力低下 & TBL-knowledge-083 \\
 &  & 関節痛・関節腫脹 & TBL-knowledge-084 \\
 &  & 腰背部痛 & TBL-knowledge-085 \\
 &  & 歩行障害 & TBL-knowledge-086 \\
 &  & 感覚障害 & TBL-knowledge-087 \\
 & 検査方法 & 筋骨格系の病態に即した徒手検査(四肢と脊柱の可動域検査・神経学的検査等) & TBL-knowledge-088 \\
 &  & 筋骨格系画像診断(単純エックス線撮影・CT・MRI・超音波検査・骨塩定量) & TBL-knowledge-089 \\
 &  & 関節液検査 & TBL-knowledge-090 \\
 & 特異的治療法 & 運動器疾患のリハビリテーション & TBL-knowledge-091 \\
 &  & 捻挫・骨折・脱臼の治療・処置 & TBL-knowledge-092 \\
循環器系 & 構造と機能 & 心臓の構造と分布する血管・神経、冠動脈の特長とその分布域 & TBL-knowledge-093 \\
 &  & 心筋細胞の微細構造と機能 & TBL-knowledge-094 \\
 &  & 心筋細胞の電気現象と心臓の興奮(刺激)伝導系 & TBL-knowledge-095 \\
 &  & 興奮収縮連関 & TBL-knowledge-096 \\
 &  & 体循環、肺循環と胎児・胎盤循環 & TBL-knowledge-097 \\
 &  & 大動脈と主な分枝(頭頸部、上肢、胸部、腹部、下肢)を図示し、分布域 & TBL-knowledge-098 \\
 &  & 主な静脈、門脈系と上・下大静脈系 & TBL-knowledge-099 \\
 &  & 毛細血管における物質・水分交換 & TBL-knowledge-100 \\
 &  & 胸管を経由するリンパの流れ & TBL-knowledge-101 \\
 &  & 心周期にともなう血行動態 & TBL-knowledge-102 \\
 &  & 心機能曲線と心拍出量の調節機序 & TBL-knowledge-103 \\
 &  & 主な臓器(脳、心臓、肺、腎臓)の循環調節 & TBL-knowledge-104 \\
 &  & 血圧調節の機序 & TBL-knowledge-105 \\
 &  & 体位や運動に伴う循環反応とその機序 & TBL-knowledge-106 \\
 & 症候 & 胸痛 & TBL-knowledge-107 \\
 &  & 腰背部痛 & TBL-knowledge-108 \\
 &  & 動悸 & TBL-knowledge-109 \\
 &  & 呼吸困難 & TBL-knowledge-110 \\
 &  & 咳・痰 & TBL-knowledge-111 \\
 &  & 浮腫 & TBL-knowledge-112 \\
 &  & 体重増加 & TBL-knowledge-113 \\
 &  & 意識障害 & TBL-knowledge-114 \\
 &  & 失神 & TBL-knowledge-115 \\
 &  & 胸水 & TBL-knowledge-116 \\
 & 検査方法 & 胸部単純エックス線撮影 & TBL-knowledge-117 \\
 &  & 心電図(安静時・運動負荷心電図・ホルタ―心電図) & TBL-knowledge-118 \\
 &  & 心臓超音波検査 & TBL-knowledge-119 \\
 &  & 心臓シンチグラフィ- & TBL-knowledge-120 \\
 &  & 冠動脈造影、冠動脈CT、MRI & TBL-knowledge-121 \\
 &  & 心カテーテル検査(心内圧・心機能・シャント率の測定) & TBL-knowledge-122 \\
 & 特異的治療法 & 虚血性心疾患に対する血行再建術(経皮的冠動脈形成術・ステント留置術・冠動脈バイパス術) & TBL-knowledge-123 \\
 &  & 不整脈に対する非薬物療法(カテーテルアブレーション・電気的除細動・ペースメーカー植え込み・植え込み型除細動器) & TBL-knowledge-124 \\
 &  & 心臓リハビリテーションなどの疾病管理プログラム & TBL-knowledge-125 \\
呼吸器系 & 構造と機能 & 気道の構造、肺葉・肺区域と肺門の構造 & TBL-knowledge-126 \\
 &  & 肺循環と体循環の違い & TBL-knowledge-127 \\
 &  & 縦隔と胸膜腔の構造 & TBL-knowledge-128 \\
 &  & 呼吸筋と呼吸運動の機序 & TBL-knowledge-129 \\
 &  & 肺気量分画、換気、死腔(換気力学(胸腔内圧、肺コンプライアンス、抵抗、クロージングボリューム(closing volume))) & TBL-knowledge-130 \\
 &  & 肺胞におけるガス交換と血流の関係 & TBL-knowledge-131 \\
 &  & 肺の換気と血流(換気血流比)が動脈血ガスにおよぼす影響(肺胞気-動脈血酸素分圧較差(alveolar-arterial oxygen difference (A-aDO2))) & TBL-knowledge-132 \\
 &  & 呼吸中枢を介する呼吸調節の機序 & TBL-knowledge-133 \\
 &  & 血液による酸素 と二酸化炭素 の運搬の仕組み & TBL-knowledge-134 \\
 &  & 気道と肺の防御機構(免疫学的・非免疫学的)と代謝機能 & TBL-knowledge-135 \\
 & 症候 & 胸痛 & TBL-knowledge-136 \\
 &  & 呼吸困難 & TBL-knowledge-137 \\
 &  & 咳・痰 & TBL-knowledge-138 \\
 &  & 血痰・喀血 & TBL-knowledge-139 \\
 &  & 喘鳴 & TBL-knowledge-140 \\
 &  & 胸部圧迫感 & TBL-knowledge-141 \\
 &  & 呼吸数・リズムの異常 & TBL-knowledge-142 \\
 &  & 胸水 & TBL-knowledge-143 \\
 & 検査方法 & 喀痰検査(喀痰細胞診・喀痰培養) & TBL-knowledge-144 \\
 &  & 胸水検査、胸膜生検 & TBL-knowledge-145 \\
 &  & 呼吸機能検査(スパイロメトリ・肺拡散能力・flow-volume曲線)、動脈血ガス分析、ポリソムノグラフィ、ピークフローメトリ & TBL-knowledge-146 \\
 &  & 画像検査(単純エックス線撮影・CT・MRI)、核医学検査(ポジトロン断層法(positron emission tomography (PET)) & TBL-knowledge-147 \\
 &  & 気管支内視鏡検査 & TBL-knowledge-148 \\
 & 特異的治療法 & 呼吸器理学療法・リハビリテーション & TBL-knowledge-149 \\
 &  & 酸素療法 & TBL-knowledge-150 \\
 &  & 人工換気 & TBL-knowledge-151 \\
消化器系 & 構造と機能 & 各消化器官の位置、形態と関係する血管 & TBL-knowledge-152 \\
 &  & 腹膜と臓器の関係 & TBL-knowledge-153 \\
 &  & 食道・胃・小腸・大腸の基本構造と部位による違い & TBL-knowledge-154 \\
 &  & 消化管運動の仕組み & TBL-knowledge-155 \\
 &  & 消化器官に対する自律神経の作用 & TBL-knowledge-156 \\
 &  & 肝の構造と機能 & TBL-knowledge-157 \\
 &  & 胃液の作用と分泌機序 & TBL-knowledge-158 \\
 &  & 胆汁の作用と胆嚢収縮の調節機序 & TBL-knowledge-159 \\
 &  & 膵外分泌系の構造と膵液の作用 & TBL-knowledge-160 \\
 &  & 小腸における消化・吸収の仕組み & TBL-knowledge-161 \\
 &  & 大腸における糞便形成と排便の仕組み & TBL-knowledge-162 \\
 &  & 主な消化管ホルモンの作用 & TBL-knowledge-163 \\
 &  & 歯、舌、唾液腺の構造と機能 & TBL-knowledge-164 \\
 &  & 咀しゃくと嚥下の機構 & TBL-knowledge-165 \\
 &  & 消化管の正常細菌叢(腸内細菌叢)の役割 & TBL-knowledge-166 \\
 & 症候 & 腹痛 & TBL-knowledge-167 \\
 &  & 悪心・嘔吐 & TBL-knowledge-168 \\
 &  & 食思(欲)不振 & TBL-knowledge-169 \\
 &  & 便秘 & TBL-knowledge-170 \\
 &  & 下痢 & TBL-knowledge-171 \\
 &  & 吐血 & TBL-knowledge-172 \\
 &  & 下血 & TBL-knowledge-173 \\
 &  & 腹部膨隆・腫瘤 & TBL-knowledge-174 \\
 &  & 黄疸 & TBL-knowledge-175 \\
 &  & 胸やけ & TBL-knowledge-176 \\
 &  & 肝腫大 & TBL-knowledge-177 \\
 & 検査方法 & 肝炎ウイルス検査 & TBL-knowledge-178 \\
 &  & 腫瘍マーカー(AFP・ CEA・ CA 19-9・ PIVKA-Ⅱなど) & TBL-knowledge-179 \\
 &  & 画像検査(単純エックス線撮影・CT・MRI) & TBL-knowledge-180 \\
 &  & 内視鏡検査 & TBL-knowledge-181 \\
 &  & 生検、細胞診 & TBL-knowledge-182 \\
 & 特異的治療法 & 経管・経腸栄養 & TBL-knowledge-183 \\
 &  & 内視鏡治療(止血・凝固・クリッピング・硬化療法など) & TBL-knowledge-184 \\
 &  & 血管内治療(動脈塞栓術など) & TBL-knowledge-185 \\
腎・尿路系(体液・電解質バランスを含む) & 構造と機能 & 体液の量と組成・浸透圧(小児と成人の違いを含めて) & TBL-knowledge-186 \\
 &  & 腎・尿路系の位置・形態と血管分布・神経支配 & TBL-knowledge-187 \\
 &  & 腎の機能の全体像やネフロン各部の構造と機能 & TBL-knowledge-188 \\
 &  & 腎糸球体における濾過の機序 & TBL-knowledge-189 \\
 &  & 尿細管各部における再吸収・分泌機構と尿の濃縮機序 & TBL-knowledge-190 \\
 &  & 水電解質、酸・塩基平衡の調節機構 & TBL-knowledge-191 \\
 &  & 腎で産生される又は腎に作用するホルモン・血管作動性物質(エリスロポエチン・ビタミンD、レニン・アンギオテンシンII、アルドステロン)の作用 & TBL-knowledge-192 \\
 &  & 蓄排尿の機序 & TBL-knowledge-193 \\
 & 症候 & 血尿 & TBL-knowledge-194 \\
 &  & タンパク尿 & TBL-knowledge-195 \\
 &  & 浮腫 & TBL-knowledge-196 \\
 &  & 脱水 & TBL-knowledge-197 \\
 &  & 尿量・排尿の異常 & TBL-knowledge-198 \\
 &  & 臨床症候の分類(急性腎炎症候群・慢性腎炎症候群・ネフローゼ症候群・急速進行性腎炎症候群・反復性または持続性血尿症候群) & TBL-knowledge-199 \\
 & 検査方法 & 糸球体濾過量(実測・推算)を含む腎機能検査法 & TBL-knowledge-200 \\
 &  & 腎・尿路系の画像診断(単純エックス線撮影・尿路造影・CT・MRI) & TBL-knowledge-201 \\
 &  & 腎生検の適応と禁忌 & TBL-knowledge-202 \\
 &  & 尿流動態検査 & TBL-knowledge-203 \\
 & 特異的治療法 & 腎代替療法(血液透析・腹膜透析・腎移植) & TBL-knowledge-204 \\
生殖機能 & 構造と機能 & 生殖腺の発生と性分化の過程 & TBL-knowledge-205 \\
 &  & 男性生殖器の発育の過程 & TBL-knowledge-206 \\
 &  & 男性生殖器の形態と機能 & TBL-knowledge-207 \\
 &  & 精巣の組織構造と精子形成の過程 & TBL-knowledge-208 \\
 &  & 陰茎の組織構造と勃起・射精の機序 & TBL-knowledge-209 \\
 &  & 女性生殖器の発育の過程 & TBL-knowledge-210 \\
 &  & 女性生殖器の形態と機能 & TBL-knowledge-211 \\
 &  & 性周期発現と排卵の機序 & TBL-knowledge-212 \\
 &  & 閉経の過程と疾病リスクの変化 & TBL-knowledge-213 \\
 & 症候 & 腹痛 & TBL-knowledge-214 \\
 &  & 腹部膨隆・腫瘤 & TBL-knowledge-215 \\
 &  &  & TBL-knowledge-216 \\
 &  &  & TBL-knowledge-217 \\
 &  &  & TBL-knowledge-218 \\
 &  &  & TBL-knowledge-219 \\
 &  & 月経異常 & TBL-knowledge-220 \\
 &  & 勃起不全 & TBL-knowledge-221 \\
 &  & 射精障害 & TBL-knowledge-222 \\
 &  & 精巣機能障害 & TBL-knowledge-223 \\
 &  & 不正性器出血 & TBL-knowledge-224 \\
 &  & 乳汁漏出症 & TBL-knowledge-225 \\
 &  & 腟分泌物(帯下)の増量 & TBL-knowledge-226 \\
 &  & 腟乾燥感 & TBL-knowledge-227 \\
 &  & 性交痛 & TBL-knowledge-228 \\
 & 検査方法 & 精巣と前立腺の画像検査法(尿路造影・CT・MRI)、超音波検査 & TBL-knowledge-229 \\
 &  & 血中ホルモン(卵胞刺激ホルモン(Follicle-Stimulating Hormone (FSH))、黄体形成ホルモン(luteinizing hormone (LH))、プロラクチン、ヒト絨毛性ゴナドトロピン(human chorionic gonadotropin (hCG))、エストロゲン、プロゲステロン)の測定 & TBL-knowledge-230 \\
 &  & 骨盤内臓器と腫瘍の画像診断(超音波断層法CT、MRI、子宮卵管造影(hysterosalpingography (HSG)) & TBL-knowledge-231 \\
 &  & 基礎体温測定 & TBL-knowledge-232 \\
 &  & 腟分泌物所見 & TBL-knowledge-233 \\
 & 特異的治療法 & 体外受精―胚移植(IVF-ET) & TBL-knowledge-234 \\
妊娠と分娩 & 構造と機能 & 妊娠・分娩・産褥での母体の解剖学的と生理学的変化 & TBL-knowledge-235 \\
 &  & 胎児・胎盤系の発達過程での機能・形態的変化 & TBL-knowledge-236 \\
 &  & 正常妊娠の経過(妊娠に伴う身体的変化を含む) & TBL-knowledge-237 \\
 &  & 正常分娩の経過 & TBL-knowledge-238 \\
 &  & 産褥の過程 & TBL-knowledge-239 \\
 &  & 育児に伴う母体の変化、精神問題および母子保健 & TBL-knowledge-240 \\
 & 症候 & 腹痛 & TBL-knowledge-241 \\
 &  & 悪心・嘔吐 & TBL-knowledge-242 \\
 &  & 腹部膨隆・腫瘤 & TBL-knowledge-243 \\
 &  & 性器出血 & TBL-knowledge-244 \\
 &  & 月経異常 & TBL-knowledge-245 \\
 & 検査方法 & 妊娠の検査(妊娠反応、超音波検査) & TBL-knowledge-246 \\
 &  & 妊娠中の検査(血液検査・出生前遺伝学的検査・羊水検査・分泌物検査・ノンストレステスト・超音波検査・超音波ドプラ法・羊水量) & TBL-knowledge-247 \\
 &  & 分娩の検査(超音波検査・胎児心拍数陣痛図) & TBL-knowledge-248 \\
 & 特異的治療法 & 妊娠時の薬物療法の注意点 & TBL-knowledge-249 \\
 &  & 人工妊娠中絶、鉗子・吸引分娩、帝王切開術の適応 & TBL-knowledge-250 \\
小児 & 症候 & 発熱 & TBL-knowledge-251 \\
 &  & 意識障害 & TBL-knowledge-252 \\
 &  & けいれん & TBL-knowledge-253 \\
 &  & 浮腫 & TBL-knowledge-254 \\
 &  & 発疹 & TBL-knowledge-255 \\
 &  & 咳・痰 & TBL-knowledge-256 \\
 &  & 呼吸困難 & TBL-knowledge-257 \\
 &  & 嚥下困難 & TBL-knowledge-258 \\
 &  & 腹痛 & TBL-knowledge-259 \\
 &  & 悪心・嘔吐 & TBL-knowledge-260 \\
 &  & 下血 & TBL-knowledge-261 \\
 &  & 便秘 & TBL-knowledge-262 \\
 &  & 下痢 & TBL-knowledge-263 \\
 &  & 黄疸 & TBL-knowledge-264 \\
 &  & 腹部膨隆・腫瘤 & TBL-knowledge-265 \\
 &  & リンパ節腫脹 & TBL-knowledge-266 \\
 &  & 尿量・排尿の異常 & TBL-knowledge-267 \\
 &  & 哺乳力低下 & TBL-knowledge-268 \\
 &  & 体重増加不良 & TBL-knowledge-269 \\
 &  & 活動性低下 & TBL-knowledge-270 \\
 & 検査方法 & 新生児マススクリーニング & TBL-knowledge-271 \\
 &  & 新生児聴覚スクリーニング & TBL-knowledge-272 \\
 &  & 乳幼児健康診査 & TBL-knowledge-273 \\
 & 特異的治療法 & 小児輸液療法 & TBL-knowledge-274 \\
 &  & 予防接種 & TBL-knowledge-275 \\
乳房 & 構造と機能 & 乳房の構造と機能 & TBL-knowledge-276 \\
 &  & 成長発達に伴う乳房の変化 & TBL-knowledge-277 \\
 &  & 乳汁分泌に関するホルモンの作用を説明できる。 & TBL-knowledge-278 \\
 & 症候 & 乳房腫瘤 & TBL-knowledge-279 \\
 &  & 異常乳汁分泌(血性乳頭分泌) & TBL-knowledge-280 \\
 &  & 乳房の腫脹・疼痛・変形 & TBL-knowledge-281 \\
 &  & 女性化乳房 & TBL-knowledge-282 \\
 & 検査方法 & 乳房腫瘤に対する画像診断(超音波検査・マンモグラフィ・MRI) & TBL-knowledge-283 \\
 &  & 乳房腫瘤に対する細胞・組織診断法 & TBL-knowledge-284 \\
 & 特異的治療法 & ※現時点で該当項目なし & TBL-knowledge-285 \\
内分泌・栄養・代謝系 & 構造と機能 & ホルモンの構造的分類、作用機序および分泌調節機能 & TBL-knowledge-286 \\
 &  & 視床下部ホルモン・下垂体ホルモンの名称、作用と相互関係 & TBL-knowledge-287 \\
 &  & 甲状腺と副甲状腺(上皮小体)から分泌されるホルモンの作用と分泌調節機構 & TBL-knowledge-288 \\
 &  & 副腎の構造と分泌されるホルモンの作用と分泌調節機構 & TBL-knowledge-289 \\
 &  & 膵島から分泌されるホルモンの作用 & TBL-knowledge-290 \\
 &  & 男性ホルモン・女性ホルモンの合成・代謝経路と作用 & TBL-knowledge-291 \\
 &  & 三大栄養素、ビタミン、微量元素の消化吸収と栄養素の生物学的利用効率 & TBL-knowledge-292 \\
 &  & 糖質・タンパク質・脂質の代謝経路と相互作用 & TBL-knowledge-293 \\
 &  & 血中ホルモン濃度に影響を与える因子およびホルモンの日内変動 & TBL-knowledge-294 \\
 & 症候 & 体重減少 & TBL-knowledge-295 \\
 &  & 体重増加 & TBL-knowledge-296 \\
 &  & 月経異常 & TBL-knowledge-297 \\
 &  & 低身長 & TBL-knowledge-298 \\
 &  & 甲状腺腫 & TBL-knowledge-299 \\
 &  & ホルモンの過剰または欠乏がもたらす身体症状 & TBL-knowledge-300 \\
 &  & エネルギー摂取の過剰または欠乏がもたらす身体症状 & TBL-knowledge-301 \\
 & 検査方法 & 血中・尿中ホルモン測定 & TBL-knowledge-302 \\
 &  & 内分泌機能検査、負荷試験 & TBL-knowledge-303 \\
 & 特異的治療法 & ※現時点でなし & TBL-knowledge-304 \\
眼・視覚系 & 構造と機能 & 眼球と付属器の構造 & TBL-knowledge-305 \\
 &  & 視覚情報の受容の仕組みと伝導路 & TBL-knowledge-306 \\
 &  & 眼球運動の仕組み & TBL-knowledge-307 \\
 &  & 対光反射、輻輳反射、角膜反射の機能 & TBL-knowledge-308 \\
 & 症候 & めまい & TBL-knowledge-309 \\
 &  & 頭痛 & TBL-knowledge-310 \\
 &  & 悪心・嘔吐 & TBL-knowledge-311 \\
 &  & 視力障害 & TBL-knowledge-312 \\
 &  & 視野異常 & TBL-knowledge-313 \\
 &  & 眼球運動障害 & TBL-knowledge-314 \\
 &  & 眼脂・眼の充血 & TBL-knowledge-315 \\
 &  & 飛蚊症 & TBL-knowledge-316 \\
 &  & 眼痛 & TBL-knowledge-317 \\
 & 検査方法 & 視力検査 & TBL-knowledge-318 \\
 &  & 視野検査 & TBL-knowledge-319 \\
 &  & 細隙灯顕微鏡検査 & TBL-knowledge-320 \\
 &  & 眼圧検査 & TBL-knowledge-321 \\
 &  & 眼底検査 & TBL-knowledge-322 \\
 & 特異的治療法 & レーザー治療 & TBL-knowledge-323 \\
耳鼻・咽喉・口腔系 & 構造と機能 & 外耳・中耳・内耳の構造 & TBL-knowledge-324 \\
 &  & 聴覚・平衡覚の受容のしくみと伝導路 & TBL-knowledge-325 \\
 &  & 口腔・鼻腔・咽頭・喉頭の構造 & TBL-knowledge-326 \\
 &  & 喉頭の機能と神経支配 & TBL-knowledge-327 \\
 &  & 眼球運動、姿勢制御と関連させた平衡感覚機構 & TBL-knowledge-328 \\
 &  & 味覚と嗅覚の受容のしくみと伝導路 & TBL-knowledge-329 \\
 & 症候 & めまい & TBL-knowledge-330 \\
 &  & 嚥下困難 & TBL-knowledge-331 \\
 &  & 気道狭窄 & TBL-knowledge-332 \\
 &  & 難聴 & TBL-knowledge-333 \\
 &  & 鼻出血 & TBL-knowledge-334 \\
 &  & 咽頭痛 & TBL-knowledge-335 \\
 &  & 開口障害 & TBL-knowledge-336 \\
 &  & 反回神経麻痺(嗄声) & TBL-knowledge-337 \\
 &  & 耳鳴 & TBL-knowledge-338 \\
 &  & 鼻閉 & TBL-knowledge-339 \\
 &  & 鼻漏 & TBL-knowledge-340 \\
 &  & 嗅覚障害 & TBL-knowledge-341 \\
 &  & いびき & TBL-knowledge-342 \\
 &  & 味覚障害 & TBL-knowledge-343 \\
 &  & 唾液分泌異常 & TBL-knowledge-344 \\
 &  & 口腔内異常 & TBL-knowledge-345 \\
 & 検査方法 & 聴力検査と平衡機能検査 & TBL-knowledge-346 \\
 &  & 味覚検査と嗅覚検査 & TBL-knowledge-347 \\
 &  & 耳鏡、鼻鏡、喉頭鏡、鼻咽腔・喉頭内視鏡 & TBL-knowledge-348 \\
 & 特異的治療法 & 補聴器・人工聴覚器 & TBL-knowledge-349 \\
 &  & 気管切開 & TBL-knowledge-350 \\
精神系 & 構造と機能 & ※神経系の項目を参照 & TBL-knowledge-351 \\
 & 症候 & 不安・抑うつ & TBL-knowledge-352 \\
 &  & 認知障害 & TBL-knowledge-353 \\
 &  & 意識障害 & TBL-knowledge-354 \\
 &  & 不眠 & TBL-knowledge-355 \\
 &  & 幻覚・妄想 & TBL-knowledge-356 \\
 &  & 心気症 & TBL-knowledge-357 \\
 & 検査方法 & 質問紙法 & TBL-knowledge-358 \\
 &  & Rorschachテスト & TBL-knowledge-359 \\
 &  & 簡易精神症状評価尺度(Brief Psychiatric Rating Scale (BPRS)) & TBL-knowledge-360 \\
 &  & Hamiltonうつ病評価尺度 & TBL-knowledge-361 \\
 &  & Beckのうつ病自己評価尺度 & TBL-knowledge-362 \\
 &  & 状態特性不安検査(State-Trait Anxiety Inventory  (STAI)) & TBL-knowledge-363 \\
 &  & Mini-Mental State Examination (MMSE) & TBL-knowledge-364 \\
 &  & 改訂長谷川式簡易知能評価スケール & TBL-knowledge-365 \\
 &  & 精神科診断分類法 & TBL-knowledge-366 \\
 & 特異的治療法 & 精神科面接 & TBL-knowledge-367 \\
 &  & 精神保健及び精神障害者福祉に関する法律、心神喪失者等医療観察法の適用場面 & TBL-knowledge-368 \\
 &  & コンサルテーション・リエゾン精神医学 & TBL-knowledge-369 \\
救急系(中毒・環境因子による疾患を含む) & 症候 & 地域の救急医療体制について病院前救護体制、メディカルコントロール、初期・二次・三次救急医療の概念を用いて説明できる & TBL-knowledge-370 \\
 &  & 意識障害 & TBL-knowledge-371 \\
 &  & 失神 & TBL-knowledge-372 \\
 &  & けいれん & TBL-knowledge-373 \\
 &  & 呼吸困難 & TBL-knowledge-374 \\
 &  & 胸痛 & TBL-knowledge-375 \\
 &  & 運動麻痺・筋力低下 & TBL-knowledge-376 \\
 &  & 腹痛 & TBL-knowledge-377 \\
 &  & 悪心・嘔吐 & TBL-knowledge-378 \\
 &  & 吐血 & TBL-knowledge-379 \\
 &  & ショック & TBL-knowledge-380 \\
 &  & 発熱 & TBL-knowledge-381 \\
 &  & 全身倦怠感 & TBL-knowledge-382 \\
 &  & 皮疹 & TBL-knowledge-383 \\
 &  & リンパ節腫脹 & TBL-knowledge-384 \\
 &  & 浮腫 & TBL-knowledge-385 \\
 &  & 呼吸困難 & TBL-knowledge-386 \\
 &  & 咳・痰 & TBL-knowledge-387 \\
 &  & 血尿 & TBL-knowledge-388 \\
 &  & 関節痛・関節腫脹 & TBL-knowledge-389 \\
 &  & 自己抗体の種類と臨床的意義を説明できる。 & TBL-knowledge-390 \\
 &  & 免疫抑制薬による治療 & TBL-knowledge-391 \\
 &  & リウマチ性疾患へのリハビリテーション & TBL-knowledge-392 \\
 &  & ショック & TBL-knowledge-393 \\
 &  & 発熱・高体温 & TBL-knowledge-394 \\
 &  & けいれん & TBL-knowledge-395 \\
 &  & 意識障害・失神 & TBL-knowledge-396 \\
 &  &  & TBL-knowledge-397 \\
 &  &  & TBL-knowledge-398 \\
 &  & 脱水 & TBL-knowledge-399 \\
 &  & 全身倦怠感 & TBL-knowledge-400 \\
 &  & 黄疸 & TBL-knowledge-401 \\
 &  & 発疹 & TBL-knowledge-402 \\
 &  & リンパ節腫脹 & TBL-knowledge-403 \\
 &  & 浮腫 & TBL-knowledge-404 \\
 &  & 胸水 & TBL-knowledge-405 \\
 &  & 胸痛・胸部圧迫感 & TBL-knowledge-406 \\
 &  & 呼吸困難・息切れ & TBL-knowledge-407 \\
 &  & 咳・痰 & TBL-knowledge-408 \\
 &  & 血痰・喀血 & TBL-knowledge-409 \\
 &  & 頭痛・頭重感 & TBL-knowledge-410 \\
 &  & 腹痛 & TBL-knowledge-411 \\
 &  & 悪心・嘔吐 & TBL-knowledge-412 \\
 &  & 便秘・下痢・血便 & TBL-knowledge-413 \\
 &  & 吐血・下血 & TBL-knowledge-414 \\
 &  & 血尿・タンパク尿 & TBL-knowledge-415 \\
 &  & 関節痛・関節腫脹 & TBL-knowledge-416 \\
 &  & 腰背部痛 & TBL-knowledge-417 \\
免疫・アレルギー & 症候 & 咽頭痛 & TBL-knowledge-418 \\
 &  & 発熱 & TBL-knowledge-419 \\
 &  & 食欲低下 & TBL-knowledge-420 \\
 &  & 体重減少 & TBL-knowledge-421 \\
 &  & 貧血 & TBL-knowledge-422 \\
 &  & リンパ節腫脹 & TBL-knowledge-423 \\
 &  & 心停止:心血管原性:急性心筋梗塞 & TBL-knowledge-424 \\
 &  & 心停止:心血管原性:急性大動脈解離 & TBL-knowledge-425 \\
 &  & 心停止:心血管原性:大動脈瘤破裂 & TBL-knowledge-426 \\
 & 検査方法 & 心停止:心血管原性:肺塞栓 & TBL-knowledge-427 \\
 & 特異的治療法 & 心停止:呼吸原性:気道閉塞 & TBL-knowledge-428 \\
 &  & 心停止:呼吸原性:緊張性気胸 & TBL-knowledge-429 \\
感染症 & 症候 & 心停止:呼吸原性:肺実質病変による低酸素血症 & TBL-knowledge-430 \\
 &  & 心停止:神経原性:重症頭部・脊髄外傷 & TBL-knowledge-431 \\
 &  & 心停止:神経原性:急性くも膜下出血 & TBL-knowledge-432 \\
 &  & 心停止:中毒・環境要因:中毒 & TBL-knowledge-433 \\
 &  & 心停止:中毒・環境要因:熱中症 & TBL-knowledge-434 \\
 &  & 心停止:中毒・環境要因:低体温症 & TBL-knowledge-435 \\
 &  & 心停止:電解質・酸塩基平衡異常:低・高カリウム血症 & TBL-knowledge-436 \\
 &  & 心停止:電解質・酸塩基平衡異常:アシドーシス & TBL-knowledge-437 \\
 &  & 心停止:電解質・酸塩基平衡異常:低血糖 & TBL-knowledge-438 \\
 &  & 中毒:食中毒 & TBL-knowledge-439 \\
 &  & 中毒:ガス中毒:一酸化炭素中毒 & TBL-knowledge-440 \\
 &  & 中毒:ガス中毒:硫化水素 & TBL-knowledge-441 \\
 &  & 中毒:ガス中毒:青酸ガス & TBL-knowledge-442 \\
 &  & 中毒:農薬:有機リン & TBL-knowledge-443 \\
 &  & 中毒:農薬:有機塩素 & TBL-knowledge-444 \\
 &  & 中毒:アルコール & TBL-knowledge-445 \\
 &  & 中毒:薬物:睡眠薬 & TBL-knowledge-446 \\
 &  & 中毒:薬物:向精神薬 & TBL-knowledge-447 \\
 &  & 中毒:薬物:解熱鎮痛薬 & TBL-knowledge-448 \\
 &  & 中毒:薬物:麻薬 & TBL-knowledge-449 \\
 &  & 中毒:薬物:覚醒剤 & TBL-knowledge-450 \\
 &  & 中毒:水銀 & TBL-knowledge-451 \\
 &  & 中毒:鉛 & TBL-knowledge-452 \\
 &  & 中毒:青酸 & TBL-knowledge-453 \\
 &  & 中毒:ヒ素 & TBL-knowledge-454 \\
 &  & 中毒:パラコート & TBL-knowledge-455 \\
腫瘍 & 症候 & 中毒:自然毒 & TBL-knowledge-456 \\
 &  & 中毒:腐食剤:酸 & TBL-knowledge-457 \\
 &  & 中毒:腐食剤:アルカリ & TBL-knowledge-458 \\
 &  & 中毒:腐食剤:フッ化水素 & TBL-knowledge-459 \\
 &  & 中毒:ボタン電池誤飲  & TBL-knowledge-460 \\
主要症候 & 症候 & 発熱 & TBL-knowledge-461 \\
 &  & 全身倦怠感 & TBL-knowledge-462 \\
 &  & 食思(欲)不振 & TBL-knowledge-463 \\
 &  & 体重減少 & TBL-knowledge-464 \\
 &  & 体重増加 & TBL-knowledge-465 \\
 &  & 意識障害 & TBL-knowledge-466 \\
 &  & 失神 & TBL-knowledge-467 \\
 &  & けいれん & TBL-knowledge-468 \\
 &  & めまい & TBL-knowledge-469 \\
 &  & 浮腫 & TBL-knowledge-470 \\
 &  & 発疹 & TBL-knowledge-471 \\
 &  & 咳・痰 & TBL-knowledge-472 \\
 &  & 血痰・喀血 & TBL-knowledge-473 \\
 &  & 呼吸困難 & TBL-knowledge-474 \\
 &  & 胸痛 & TBL-knowledge-475 \\
 &  & 動悸 & TBL-knowledge-476 \\
 &  & 嚥下困難 & TBL-knowledge-477 \\
 &  & 腹痛 & TBL-knowledge-478 \\
 &  & 悪心・嘔吐 & TBL-knowledge-479 \\
 &  & 吐血 & TBL-knowledge-480 \\
 &  & 下血 & TBL-knowledge-481 \\
 &  & 便秘 & TBL-knowledge-482 \\
 &  & 下痢 & TBL-knowledge-483 \\
 &  & 黄疸 & TBL-knowledge-484 \\
 &  & 腹部膨隆・腫瘤 & TBL-knowledge-485 \\
 &  & リンパ節腫脹 & TBL-knowledge-486 \\
 &  & 尿量・排尿の異常 & TBL-knowledge-487 \\
 &  & 血尿 & TBL-knowledge-488 \\
 &  & 月経異常 & TBL-knowledge-489 \\
 &  & 不安・抑うつ & TBL-knowledge-490 \\
 &  & 認知障害 & TBL-knowledge-491 \\
 &  & 頭痛 & TBL-knowledge-492 \\
 &  & 運動麻痺・筋力低下 & TBL-knowledge-493 \\
 &  & 歩行障害 & TBL-knowledge-494 \\
 &  & 感覚障害 & TBL-knowledge-495 \\
 &  & 腰背部痛 & TBL-knowledge-496 \\
 &  & 関節痛・関節腫脹 & TBL-knowledge-497 \\
\bottomrule
\end{xltabular}

\begin{xltabular}{\linewidth}{XXX}
\caption{\label{tbl:knowledge_blood}血液・造血器・リンパ系} \\
\toprule
分類 & 項目名 & id \\
\midrule
\endhead
構造と機能 & 骨髄の構造 & TBL-knowledge_blood-001 \\
 & 造血幹細胞から各血球への分化と成熟の過程 & TBL-knowledge_blood-002 \\
 & 主な造血因子(エリスロポエチン、顆粒球コロニー刺激因子(G-CSF)、トロンボポエチン) & TBL-knowledge_blood-003 \\
 & 脾臓、胸腺、リンパ節、扁桃とPeyer板の構造と機能 & TBL-knowledge_blood-004 \\
 & 血漿タンパク質の種類と機能 & TBL-knowledge_blood-005 \\
 & 赤血球とヘモグロビンの構造と機能 & TBL-knowledge_blood-006 \\
 & 白血球の種類と機能 & TBL-knowledge_blood-007 \\
 & 血小板の機能と止血や凝固・線溶の機序 & TBL-knowledge_blood-008 \\
症候 & 発熱 & TBL-knowledge_blood-009 \\
 & 全身倦怠感 & TBL-knowledge_blood-010 \\
 & 黄疸 & TBL-knowledge_blood-011 \\
 & リンパ節腫脹 & TBL-knowledge_blood-012 \\
 & 貧血 & TBL-knowledge_blood-013 \\
 & 出血傾向 & TBL-knowledge_blood-014 \\
 & 血栓傾向 & TBL-knowledge_blood-015 \\
検査方法 & 末梢血塗抹 & TBL-knowledge_blood-016 \\
 & 凝固・線溶・血小板機能検査 & TBL-knowledge_blood-017 \\
 & 骨髄検査(骨髄穿刺、骨髄生検) & TBL-knowledge_blood-018 \\
 & 輸血関連検査 & TBL-knowledge_blood-019 \\
 & タンパク分画、免疫電気泳動 & TBL-knowledge_blood-020 \\
 & 遺伝子・染色体検査 & TBL-knowledge_blood-021 \\
特異的治療法 & 輸血 & TBL-knowledge_blood-022 \\
 & 造血幹細胞移植 & TBL-knowledge_blood-023 \\
\bottomrule
\end{xltabular}

\begin{xltabular}{\linewidth}{XXX}
\caption{\label{tbl:knowledge_neuro}神経系} \\
\toprule
分類 & 項目名 & id \\
\midrule
\endhead
構造と機能 & 中枢神経系と末梢神経系の構成 & TBL-knowledge_neuro-001 \\
 & 脳の血管支配と血液脳関門 & TBL-knowledge_neuro-002 \\
 & 脳のエネルギー代謝の特徴 & TBL-knowledge_neuro-003 \\
 & 主な脳内神経伝達物質(アセチルコリン・ドパミン・ノルアドレナリン)とその作用 & TBL-knowledge_neuro-004 \\
 & 髄膜・脳室系の構造と脳脊髄液の産生と循環 & TBL-knowledge_neuro-005 \\
 & 脊髄の構造、機能局在と伝導路 & TBL-knowledge_neuro-006 \\
 & 脊髄反射(伸張反射、屈筋反射)と筋の相反神経支配 & TBL-knowledge_neuro-007 \\
 & 脊髄神経と神経叢(頸・腕・腰仙骨)の構成および主な骨格筋支配と皮膚分布(デルマトーム) & TBL-knowledge_neuro-008 \\
 & 脳幹の構造と機能、および伝導路 & TBL-knowledge_neuro-009 \\
 & 脳神経の名称、核の局在、走行・分布と機能 & TBL-knowledge_neuro-010 \\
 & 大脳の構造と大脳皮質の機能局在(運動野・感覚野・言語野) & TBL-knowledge_neuro-011 \\
 & 辺縁系の構造と記憶・学習の機序との関連 & TBL-knowledge_neuro-012 \\
 & 錐体路を中心とした随意運動の発現機構 & TBL-knowledge_neuro-013 \\
 & 小脳の構造と機能 & TBL-knowledge_neuro-014 \\
 & 大脳基底核(線条体・淡蒼球・黒質)の線維結合と機能 & TBL-knowledge_neuro-015 \\
 & 痛覚、温度覚、触覚と深部感覚の受容機序と伝導路 & TBL-knowledge_neuro-016 \\
 & 視覚、聴覚・平衡覚、嗅覚、味覚の受容機序と伝導路 & TBL-knowledge_neuro-017 \\
 & 交感神経系と副交感神経系の中枢内局在、末梢分布、機能と伝達物質 & TBL-knowledge_neuro-018 \\
 & 内分泌および自律機能と関連づけた視床下部の構造と機能 & TBL-knowledge_neuro-019 \\
 & ストレス反応と本能・情動行動の発現機序 & TBL-knowledge_neuro-020 \\
症候 & 頭痛 & TBL-knowledge_neuro-021 \\
 & めまい & TBL-knowledge_neuro-022 \\
 & けいれん & TBL-knowledge_neuro-023 \\
 & 意識障害 & TBL-knowledge_neuro-024 \\
 & 運動麻痺・筋力低下 & TBL-knowledge_neuro-025 \\
 & 歩行障害 & TBL-knowledge_neuro-026 \\
 & 感覚障害 & TBL-knowledge_neuro-027 \\
 & 認知障害 & TBL-knowledge_neuro-028 \\
 & 失語症・構音障害 & TBL-knowledge_neuro-029 \\
 & 振戦 & TBL-knowledge_neuro-030 \\
 & 小脳性・前庭性・感覚性運動失調障害 & TBL-knowledge_neuro-031 \\
 & 不随意運動(ミオクローヌス・舞踏運動・ジストニア・固定姿勢保持困難・アテトーシス・チック) & TBL-knowledge_neuro-032 \\
 & 頭蓋内圧亢進(急性・慢性) & TBL-knowledge_neuro-033 \\
 & 脳ヘルニア & TBL-knowledge_neuro-034 \\
検査方法 & 脳・脊髄の画像検査(CT・MRI) & TBL-knowledge_neuro-035 \\
 & 神経系の電気生理学的検査(脳波検査、針筋電図検査、末梢神経伝導検査) & TBL-knowledge_neuro-036 \\
特異的治療法 & 脳血管障害の急性期治療とリハビリテーション医療 & TBL-knowledge_neuro-037 \\
\bottomrule
\end{xltabular}

\begin{xltabular}{\linewidth}{XXX}
\caption{\label{tbl:knowledge_skin}皮膚系} \\
\toprule
分類 & 項目名 & id \\
\midrule
\endhead
構造と機能 & 皮膚の組織構造 & TBL-knowledge_skin-001 \\
 & 皮膚の細胞動態と角化の機構 & TBL-knowledge_skin-002 \\
 & 皮膚の免疫防御能 & TBL-knowledge_skin-003 \\
症候 & 皮疹(紅斑・紫斑・色素斑・丘疹・結節・腫瘤・水疱・膿疱・嚢腫・びらん・潰瘍・毛細血管拡張・硬化・瘢痕・萎縮・鱗屑・痂皮・苔癬化・壊疽) & TBL-knowledge_skin-004 \\
 & そう痒 & TBL-knowledge_skin-005 \\
 & 粘膜疹 & TBL-knowledge_skin-006 \\
 & 脱毛 & TBL-knowledge_skin-007 \\
検査方法 & 皮膚検査法(硝子圧法・皮膚描記法(Darier 徴候)・Nikolsky 現象・Tzanck 試験・光線テスト) & TBL-knowledge_skin-008 \\
 & 皮膚アレルギー検査法(プリックテスト・皮内テスト・パッチテスト) & TBL-knowledge_skin-009 \\
 & 微生物検査法(検体採取法・苛性カリ(KOH)直接検鏡法) & TBL-knowledge_skin-010 \\
 & ダーモスコピー & TBL-knowledge_skin-011 \\
特異的治療法 & 外用療法 & TBL-knowledge_skin-012 \\
 & 凍結療法 & TBL-knowledge_skin-013 \\
 & 光線療法(PUVA療法) & TBL-knowledge_skin-014 \\
\bottomrule
\end{xltabular}

\begin{xltabular}{\linewidth}{XXX}
\caption{\label{tbl:knowledge_cardio}循環器系} \\
\toprule
分類 & 項目名 & id \\
\midrule
\endhead
構造と機能 & 心臓の構造と分布する血管・神経、冠動脈の特長とその分布域 & TBL-knowledge_cardio-001 \\
 & 心筋細胞の微細構造と機能 & TBL-knowledge_cardio-002 \\
 & 心筋細胞の電気現象と心臓の興奮(刺激)伝導系 & TBL-knowledge_cardio-003 \\
 & 興奮収縮連関 & TBL-knowledge_cardio-004 \\
 & 体循環、肺循環と胎児・胎盤循環 & TBL-knowledge_cardio-005 \\
 & 大動脈と主な分枝(頭頸部、上肢、胸部、腹部、下肢)を図示し、分布域 & TBL-knowledge_cardio-006 \\
 & 主な静脈、門脈系と上・下大静脈系 & TBL-knowledge_cardio-007 \\
 & 毛細血管における物質・水分交換 & TBL-knowledge_cardio-008 \\
 & 胸管を経由するリンパの流れ & TBL-knowledge_cardio-009 \\
 & 心周期にともなう血行動態 & TBL-knowledge_cardio-010 \\
 & 心機能曲線と心拍出量の調節機序 & TBL-knowledge_cardio-011 \\
 & 主な臓器(脳、心臓、肺、腎臓)の循環調節 & TBL-knowledge_cardio-012 \\
 & 血圧調節の機序 & TBL-knowledge_cardio-013 \\
 & 体位や運動に伴う循環反応とその機序 & TBL-knowledge_cardio-014 \\
症候 & 胸痛 & TBL-knowledge_cardio-015 \\
 & 腰背部痛 & TBL-knowledge_cardio-016 \\
 & 動悸 & TBL-knowledge_cardio-017 \\
 & 呼吸困難 & TBL-knowledge_cardio-018 \\
 & 咳・痰 & TBL-knowledge_cardio-019 \\
 & 浮腫 & TBL-knowledge_cardio-020 \\
 & 体重増加 & TBL-knowledge_cardio-021 \\
 & 意識障害 & TBL-knowledge_cardio-022 \\
 & 失神 & TBL-knowledge_cardio-023 \\
 & 胸水 & TBL-knowledge_cardio-024 \\
検査方法 & 胸部単純エックス線撮影 & TBL-knowledge_cardio-025 \\
 & 心電図(安静時・運動負荷心電図・ホルタ―心電図) & TBL-knowledge_cardio-026 \\
 & 心臓超音波検査 & TBL-knowledge_cardio-027 \\
 & 心臓シンチグラフィ- & TBL-knowledge_cardio-028 \\
 & 冠動脈造影、冠動脈CT、MRI & TBL-knowledge_cardio-029 \\
 & 心カテーテル検査(心内圧・心機能・シャント率の測定) & TBL-knowledge_cardio-030 \\
特異的治療法 & 虚血性心疾患に対する血行再建術(経皮的冠動脈形成術・ステント留置術・冠動脈バイパス術) & TBL-knowledge_cardio-031 \\
 & 不整脈に対する非薬物療法(カテーテルアブレーション・電気的除細動・ペースメーカー植え込み・植え込み型除細動器) & TBL-knowledge_cardio-032 \\
 & 心臓リハビリテーションなどの疾病管理プログラム & TBL-knowledge_cardio-033 \\
\bottomrule
\end{xltabular}

\begin{xltabular}{\linewidth}{XXX}
\caption{\label{tbl:knowledge_lung}呼吸器系} \\
\toprule
分類 & 項目名 & id \\
\midrule
\endhead
構造と機能 & 気道の構造、肺葉・肺区域と肺門の構造 & TBL-knowledge_lung-001 \\
 & 肺循環と体循環の違い & TBL-knowledge_lung-002 \\
 & 縦隔と胸膜腔の構造 & TBL-knowledge_lung-003 \\
 & 呼吸筋と呼吸運動の機序 & TBL-knowledge_lung-004 \\
 & 肺気量分画、換気、死腔(換気力学(胸腔内圧、肺コンプライアンス、抵抗、クロージングボリューム(closing volume))) & TBL-knowledge_lung-005 \\
 & 肺胞におけるガス交換と血流の関係 & TBL-knowledge_lung-006 \\
 & 肺の換気と血流(換気血流比)が動脈血ガスにおよぼす影響(肺胞気-動脈血酸素分圧較差(alveolar-arterial oxygen difference (A-aDO2))) & TBL-knowledge_lung-007 \\
 & 呼吸中枢を介する呼吸調節の機序 & TBL-knowledge_lung-008 \\
 & 血液による酸素 と二酸化炭素 の運搬の仕組み & TBL-knowledge_lung-009 \\
 & 気道と肺の防御機構(免疫学的・非免疫学的)と代謝機能 & TBL-knowledge_lung-010 \\
症候 & 胸痛 & TBL-knowledge_lung-011 \\
 & 呼吸困難 & TBL-knowledge_lung-012 \\
 & 咳・痰 & TBL-knowledge_lung-013 \\
 & 血痰・喀血 & TBL-knowledge_lung-014 \\
 & 喘鳴 & TBL-knowledge_lung-015 \\
 & 胸部圧迫感 & TBL-knowledge_lung-016 \\
 & 呼吸数・リズムの異常 & TBL-knowledge_lung-017 \\
 & 胸水 & TBL-knowledge_lung-018 \\
検査方法 & 喀痰検査(喀痰細胞診・喀痰培養) & TBL-knowledge_lung-019 \\
 & 胸水検査、胸膜生検 & TBL-knowledge_lung-020 \\
 & 呼吸機能検査(スパイロメトリ・肺拡散能力・flow-volume曲線)、動脈血ガス分析、ポリソムノグラフィ、ピークフローメトリ & TBL-knowledge_lung-021 \\
 & 画像検査(単純エックス線撮影・CT・MRI)、核医学検査(ポジトロン断層法(positron emission tomography (PET)) & TBL-knowledge_lung-022 \\
 & 気管支内視鏡検査 & TBL-knowledge_lung-023 \\
特異的治療法 & 呼吸器理学療法・リハビリテーション & TBL-knowledge_lung-024 \\
 & 酸素療法 & TBL-knowledge_lung-025 \\
 & 人工換気 & TBL-knowledge_lung-026 \\
\bottomrule
\end{xltabular}

\begin{xltabular}{\linewidth}{XXX}
\caption{\label{tbl:knowledge_gastroenterology}消化器系} \\
\toprule
分類 & 項目名 & id \\
\midrule
\endhead
構造と機能 & 各消化器官の位置、形態と関係する血管 & TBL-knowledge_gastroenterology-001 \\
 & 腹膜と臓器の関係 & TBL-knowledge_gastroenterology-002 \\
 & 食道・胃・小腸・大腸の基本構造と部位による違い & TBL-knowledge_gastroenterology-003 \\
 & 消化管運動の仕組み & TBL-knowledge_gastroenterology-004 \\
 & 消化器官に対する自律神経の作用 & TBL-knowledge_gastroenterology-005 \\
 & 肝の構造と機能 & TBL-knowledge_gastroenterology-006 \\
 & 胃液の作用と分泌機序 & TBL-knowledge_gastroenterology-007 \\
 & 胆汁の作用と胆嚢収縮の調節機序 & TBL-knowledge_gastroenterology-008 \\
 & 膵外分泌系の構造と膵液の作用 & TBL-knowledge_gastroenterology-009 \\
 & 小腸における消化・吸収の仕組み & TBL-knowledge_gastroenterology-010 \\
 & 大腸における糞便形成と排便の仕組み & TBL-knowledge_gastroenterology-011 \\
 & 主な消化管ホルモンの作用 & TBL-knowledge_gastroenterology-012 \\
 & 歯、舌、唾液腺の構造と機能 & TBL-knowledge_gastroenterology-013 \\
 & 咀しゃくと嚥下の機構 & TBL-knowledge_gastroenterology-014 \\
 & 消化管の正常細菌叢(腸内細菌叢)の役割 & TBL-knowledge_gastroenterology-015 \\
症候 & 腹痛 & TBL-knowledge_gastroenterology-016 \\
 & 悪心・嘔吐 & TBL-knowledge_gastroenterology-017 \\
 & 食思(欲)不振 & TBL-knowledge_gastroenterology-018 \\
 & 便秘 & TBL-knowledge_gastroenterology-019 \\
 & 下痢 & TBL-knowledge_gastroenterology-020 \\
 & 吐血 & TBL-knowledge_gastroenterology-021 \\
 & 下血 & TBL-knowledge_gastroenterology-022 \\
 & 腹部膨隆・腫瘤 & TBL-knowledge_gastroenterology-023 \\
 & 黄疸 & TBL-knowledge_gastroenterology-024 \\
 & 胸やけ & TBL-knowledge_gastroenterology-025 \\
 & 肝腫大 & TBL-knowledge_gastroenterology-026 \\
検査方法 & 肝炎ウイルス検査 & TBL-knowledge_gastroenterology-027 \\
 & 腫瘍マーカー(AFP・ CEA・ CA 19-9・ PIVKA-Ⅱなど) & TBL-knowledge_gastroenterology-028 \\
 & 画像検査(単純エックス線撮影・CT・MRI) & TBL-knowledge_gastroenterology-029 \\
 & 内視鏡検査 & TBL-knowledge_gastroenterology-030 \\
 & 生検、細胞診 & TBL-knowledge_gastroenterology-031 \\
特異的治療法 & 経管・経腸栄養 & TBL-knowledge_gastroenterology-032 \\
 & 内視鏡治療(止血・凝固・クリッピング・硬化療法など) & TBL-knowledge_gastroenterology-033 \\
 & 血管内治療(動脈塞栓術など) & TBL-knowledge_gastroenterology-034 \\
\bottomrule
\end{xltabular}

\begin{xltabular}{\linewidth}{XXX}
\caption{\label{tbl:knowledge_renal}腎・尿路系(体液・電解質バランスを含む)} \\
\toprule
分類 & 項目名 & id \\
\midrule
\endhead
構造と機能 & 体液の量と組成・浸透圧(小児と成人の違いを含めて) & TBL-knowledge_renal-001 \\
 & 腎・尿路系の位置・形態と血管分布・神経支配 & TBL-knowledge_renal-002 \\
 & 腎の機能の全体像やネフロン各部の構造と機能 & TBL-knowledge_renal-003 \\
 & 腎糸球体における濾過の機序 & TBL-knowledge_renal-004 \\
 & 尿細管各部における再吸収・分泌機構と尿の濃縮機序 & TBL-knowledge_renal-005 \\
 & 水電解質、酸・塩基平衡の調節機構 & TBL-knowledge_renal-006 \\
 & 腎で産生される又は腎に作用するホルモン・血管作動性物質(エリスロポエチン・ビタミンD、レニン・アンギオテンシンII、アルドステロン)の作用 & TBL-knowledge_renal-007 \\
 & 蓄排尿の機序 & TBL-knowledge_renal-008 \\
症候 & 血尿 & TBL-knowledge_renal-009 \\
 & タンパク尿 & TBL-knowledge_renal-010 \\
 & 浮腫 & TBL-knowledge_renal-011 \\
 & 脱水 & TBL-knowledge_renal-012 \\
 & 尿量・排尿の異常 & TBL-knowledge_renal-013 \\
 & 臨床症候の分類(急性腎炎症候群・慢性腎炎症候群・ネフローゼ症候群・急速進行性腎炎症候群・反復性または持続性血尿症候群) & TBL-knowledge_renal-014 \\
検査方法 & 糸球体濾過量(実測・推算)を含む腎機能検査法 & TBL-knowledge_renal-015 \\
 & 腎・尿路系の画像診断(単純エックス線撮影・尿路造影・CT・MRI) & TBL-knowledge_renal-016 \\
 & 腎生検の適応と禁忌 & TBL-knowledge_renal-017 \\
 & 尿流動態検査 & TBL-knowledge_renal-018 \\
特異的治療法 & 腎代替療法(血液透析・腹膜透析・腎移植) & TBL-knowledge_renal-019 \\
\bottomrule
\end{xltabular}

\begin{xltabular}{\linewidth}{XXX}
\caption{\label{tbl:knowledge_repro}生殖機能} \\
\toprule
分類 & 項目名 & id \\
\midrule
\endhead
構造と機能 & 生殖腺の発生と性分化の過程 & TBL-knowledge_repro-001 \\
 & 男性生殖器の発育の過程 & TBL-knowledge_repro-002 \\
 & 男性生殖器の形態と機能 & TBL-knowledge_repro-003 \\
 & 精巣の組織構造と精子形成の過程 & TBL-knowledge_repro-004 \\
 & 陰茎の組織構造と勃起・射精の機序 & TBL-knowledge_repro-005 \\
 & 女性生殖器の発育の過程 & TBL-knowledge_repro-006 \\
 & 女性生殖器の形態と機能 & TBL-knowledge_repro-007 \\
 & 性周期発現と排卵の機序 & TBL-knowledge_repro-008 \\
 & 閉経の過程と疾病リスクの変化 & TBL-knowledge_repro-009 \\
症候 & 腹痛 & TBL-knowledge_repro-010 \\
 & 腹部膨隆・腫瘤 & TBL-knowledge_repro-011 \\
 &  & TBL-knowledge_repro-012 \\
 &  & TBL-knowledge_repro-013 \\
 &  & TBL-knowledge_repro-014 \\
 &  & TBL-knowledge_repro-015 \\
 & 月経異常 & TBL-knowledge_repro-016 \\
 & 勃起不全 & TBL-knowledge_repro-017 \\
 & 射精障害 & TBL-knowledge_repro-018 \\
 & 精巣機能障害 & TBL-knowledge_repro-019 \\
 & 不正性器出血 & TBL-knowledge_repro-020 \\
 & 乳汁漏出症 & TBL-knowledge_repro-021 \\
 & 腟分泌物(帯下)の増量 & TBL-knowledge_repro-022 \\
 & 腟乾燥感 & TBL-knowledge_repro-023 \\
 & 性交痛 & TBL-knowledge_repro-024 \\
検査方法 & 精巣と前立腺の画像検査法(尿路造影・CT・MRI)、超音波検査 & TBL-knowledge_repro-025 \\
 & 血中ホルモン(卵胞刺激ホルモン(Follicle-Stimulating Hormone (FSH))、黄体形成ホルモン(luteinizing hormone (LH))、プロラクチン、ヒト絨毛性ゴナドトロピン(human chorionic gonadotropin (hCG))、エストロゲン、プロゲステロン)の測定 & TBL-knowledge_repro-026 \\
 & 骨盤内臓器と腫瘍の画像診断(超音波断層法CT、MRI、子宮卵管造影(hysterosalpingography (HSG)) & TBL-knowledge_repro-027 \\
 & 基礎体温測定 & TBL-knowledge_repro-028 \\
 & 腟分泌物所見 & TBL-knowledge_repro-029 \\
特異的治療法 & 体外受精―胚移植(IVF-ET) & TBL-knowledge_repro-030 \\
\bottomrule
\end{xltabular}

\begin{xltabular}{\linewidth}{XXX}
\caption{\label{tbl:knowledge_preg}妊娠と分娩} \\
\toprule
分類 & 項目名 & id \\
\midrule
\endhead
構造と機能 & 妊娠・分娩・産褥での母体の解剖学的と生理学的変化 & TBL-knowledge_preg-001 \\
 & 胎児・胎盤系の発達過程での機能・形態的変化 & TBL-knowledge_preg-002 \\
 & 正常妊娠の経過(妊娠に伴う身体的変化を含む) & TBL-knowledge_preg-003 \\
 & 正常分娩の経過 & TBL-knowledge_preg-004 \\
 & 産褥の過程 & TBL-knowledge_preg-005 \\
 & 育児に伴う母体の変化、精神問題および母子保健 & TBL-knowledge_preg-006 \\
症候 & 腹痛 & TBL-knowledge_preg-007 \\
 & 悪心・嘔吐 & TBL-knowledge_preg-008 \\
 & 腹部膨隆・腫瘤 & TBL-knowledge_preg-009 \\
 & 性器出血 & TBL-knowledge_preg-010 \\
 & 月経異常 & TBL-knowledge_preg-011 \\
検査方法 & 妊娠の検査(妊娠反応、超音波検査) & TBL-knowledge_preg-012 \\
 & 妊娠中の検査(血液検査・出生前遺伝学的検査・羊水検査・分泌物検査・ノンストレステスト・超音波検査・超音波ドプラ法・羊水量) & TBL-knowledge_preg-013 \\
 & 分娩の検査(超音波検査・胎児心拍数陣痛図) & TBL-knowledge_preg-014 \\
特異的治療法 & 妊娠時の薬物療法の注意点 & TBL-knowledge_preg-015 \\
 & 人工妊娠中絶、鉗子・吸引分娩、帝王切開術の適応 & TBL-knowledge_preg-016 \\
\bottomrule
\end{xltabular}

\begin{xltabular}{\linewidth}{XXX}
\caption{\label{tbl:knowledge_pedi}小児} \\
\toprule
分類 & 項目名 & id \\
\midrule
\endhead
症候 & 発熱 & TBL-knowledge_pedi-001 \\
 & 意識障害 & TBL-knowledge_pedi-002 \\
 & けいれん & TBL-knowledge_pedi-003 \\
 & 浮腫 & TBL-knowledge_pedi-004 \\
 & 発疹 & TBL-knowledge_pedi-005 \\
 & 咳・痰 & TBL-knowledge_pedi-006 \\
 & 呼吸困難 & TBL-knowledge_pedi-007 \\
 & 嚥下困難 & TBL-knowledge_pedi-008 \\
 & 腹痛 & TBL-knowledge_pedi-009 \\
 & 悪心・嘔吐 & TBL-knowledge_pedi-010 \\
 & 下血 & TBL-knowledge_pedi-011 \\
 & 便秘 & TBL-knowledge_pedi-012 \\
 & 下痢 & TBL-knowledge_pedi-013 \\
 & 黄疸 & TBL-knowledge_pedi-014 \\
 & 腹部膨隆・腫瘤 & TBL-knowledge_pedi-015 \\
 & リンパ節腫脹 & TBL-knowledge_pedi-016 \\
 & 尿量・排尿の異常 & TBL-knowledge_pedi-017 \\
 & 哺乳力低下 & TBL-knowledge_pedi-018 \\
 & 体重増加不良 & TBL-knowledge_pedi-019 \\
 & 活動性低下 & TBL-knowledge_pedi-020 \\
検査方法 & 新生児マススクリーニング & TBL-knowledge_pedi-021 \\
 & 新生児聴覚スクリーニング & TBL-knowledge_pedi-022 \\
 & 乳幼児健康診査 & TBL-knowledge_pedi-023 \\
特異的治療法 & 小児輸液療法 & TBL-knowledge_pedi-024 \\
 & 予防接種 & TBL-knowledge_pedi-025 \\
\bottomrule
\end{xltabular}

\begin{xltabular}{\linewidth}{XXX}
\caption{\label{tbl:knowledge_breast}乳房} \\
\toprule
分類 & 項目名 & id \\
\midrule
\endhead
構造と機能 & 乳房の構造と機能 & TBL-knowledge_breast-001 \\
 & 成長発達に伴う乳房の変化 & TBL-knowledge_breast-002 \\
 & 乳汁分泌に関するホルモンの作用を説明できる。 & TBL-knowledge_breast-003 \\
症候 & 乳房腫瘤 & TBL-knowledge_breast-004 \\
 & 異常乳汁分泌(血性乳頭分泌) & TBL-knowledge_breast-005 \\
 & 乳房の腫脹・疼痛・変形 & TBL-knowledge_breast-006 \\
 & 女性化乳房 & TBL-knowledge_breast-007 \\
検査方法 & 乳房腫瘤に対する画像診断(超音波検査・マンモグラフィ・MRI) & TBL-knowledge_breast-008 \\
 & 乳房腫瘤に対する細胞・組織診断法 & TBL-knowledge_breast-009 \\
特異的治療法 & ※現時点で該当項目なし & TBL-knowledge_breast-010 \\
\bottomrule
\end{xltabular}

\begin{xltabular}{\linewidth}{XXX}
\caption{\label{tbl:knowledge_endo}内分泌・栄養・代謝系} \\
\toprule
分類 & 項目名 & id \\
\midrule
\endhead
構造と機能 & ホルモンの構造的分類、作用機序および分泌調節機能 & TBL-knowledge_endo-001 \\
 & 視床下部ホルモン・下垂体ホルモンの名称、作用と相互関係 & TBL-knowledge_endo-002 \\
 & 甲状腺と副甲状腺(上皮小体)から分泌されるホルモンの作用と分泌調節機構 & TBL-knowledge_endo-003 \\
 & 副腎の構造と分泌されるホルモンの作用と分泌調節機構 & TBL-knowledge_endo-004 \\
 & 膵島から分泌されるホルモンの作用 & TBL-knowledge_endo-005 \\
 & 男性ホルモン・女性ホルモンの合成・代謝経路と作用 & TBL-knowledge_endo-006 \\
 & 三大栄養素、ビタミン、微量元素の消化吸収と栄養素の生物学的利用効率 & TBL-knowledge_endo-007 \\
 & 糖質・タンパク質・脂質の代謝経路と相互作用 & TBL-knowledge_endo-008 \\
 & 血中ホルモン濃度に影響を与える因子およびホルモンの日内変動 & TBL-knowledge_endo-009 \\
症候 & 体重減少 & TBL-knowledge_endo-010 \\
 & 体重増加 & TBL-knowledge_endo-011 \\
 & 月経異常 & TBL-knowledge_endo-012 \\
 & 低身長 & TBL-knowledge_endo-013 \\
 & 甲状腺腫 & TBL-knowledge_endo-014 \\
 & ホルモンの過剰または欠乏がもたらす身体症状 & TBL-knowledge_endo-015 \\
 & エネルギー摂取の過剰または欠乏がもたらす身体症状 & TBL-knowledge_endo-016 \\
検査方法 & 血中・尿中ホルモン測定 & TBL-knowledge_endo-017 \\
 & 内分泌機能検査、負荷試験 & TBL-knowledge_endo-018 \\
特異的治療法 & ※現時点でなし & TBL-knowledge_endo-019 \\
\bottomrule
\end{xltabular}

\begin{xltabular}{\linewidth}{XXX}
\caption{\label{tbl:knowledge_eye}眼・視覚系} \\
\toprule
分類 & 項目名 & id \\
\midrule
\endhead
構造と機能 & 眼球と付属器の構造 & TBL-knowledge_eye-001 \\
 & 視覚情報の受容の仕組みと伝導路 & TBL-knowledge_eye-002 \\
 & 眼球運動の仕組み & TBL-knowledge_eye-003 \\
 & 対光反射、輻輳反射、角膜反射の機能 & TBL-knowledge_eye-004 \\
症候 & めまい & TBL-knowledge_eye-005 \\
 & 頭痛 & TBL-knowledge_eye-006 \\
 & 悪心・嘔吐 & TBL-knowledge_eye-007 \\
 & 視力障害 & TBL-knowledge_eye-008 \\
 & 視野異常 & TBL-knowledge_eye-009 \\
 & 眼球運動障害 & TBL-knowledge_eye-010 \\
 & 眼脂・眼の充血 & TBL-knowledge_eye-011 \\
 & 飛蚊症 & TBL-knowledge_eye-012 \\
 & 眼痛 & TBL-knowledge_eye-013 \\
検査方法 & 視力検査 & TBL-knowledge_eye-014 \\
 & 視野検査 & TBL-knowledge_eye-015 \\
 & 細隙灯顕微鏡検査 & TBL-knowledge_eye-016 \\
 & 眼圧検査 & TBL-knowledge_eye-017 \\
 & 眼底検査 & TBL-knowledge_eye-018 \\
特異的治療法 & レーザー治療 & TBL-knowledge_eye-019 \\
\bottomrule
\end{xltabular}

\begin{xltabular}{\linewidth}{XXX}
\caption{\label{tbl:knowledge_otorhinolaryngology}耳鼻・咽喉・口腔系} \\
\toprule
分類 & 項目名 & id \\
\midrule
\endhead
構造と機能 & 外耳・中耳・内耳の構造 & TBL-knowledge_otorhinolaryngology-001 \\
 & 聴覚・平衡覚の受容のしくみと伝導路 & TBL-knowledge_otorhinolaryngology-002 \\
 & 口腔・鼻腔・咽頭・喉頭の構造 & TBL-knowledge_otorhinolaryngology-003 \\
 & 喉頭の機能と神経支配 & TBL-knowledge_otorhinolaryngology-004 \\
 & 眼球運動、姿勢制御と関連させた平衡感覚機構 & TBL-knowledge_otorhinolaryngology-005 \\
 & 味覚と嗅覚の受容のしくみと伝導路 & TBL-knowledge_otorhinolaryngology-006 \\
症候 & めまい & TBL-knowledge_otorhinolaryngology-007 \\
 & 嚥下困難 & TBL-knowledge_otorhinolaryngology-008 \\
 & 気道狭窄 & TBL-knowledge_otorhinolaryngology-009 \\
 & 難聴 & TBL-knowledge_otorhinolaryngology-010 \\
 & 鼻出血 & TBL-knowledge_otorhinolaryngology-011 \\
 & 咽頭痛 & TBL-knowledge_otorhinolaryngology-012 \\
 & 開口障害 & TBL-knowledge_otorhinolaryngology-013 \\
 & 反回神経麻痺(嗄声) & TBL-knowledge_otorhinolaryngology-014 \\
 & 耳鳴 & TBL-knowledge_otorhinolaryngology-015 \\
 & 鼻閉 & TBL-knowledge_otorhinolaryngology-016 \\
 & 鼻漏 & TBL-knowledge_otorhinolaryngology-017 \\
 & 嗅覚障害 & TBL-knowledge_otorhinolaryngology-018 \\
 & いびき & TBL-knowledge_otorhinolaryngology-019 \\
 & 味覚障害 & TBL-knowledge_otorhinolaryngology-020 \\
 & 唾液分泌異常 & TBL-knowledge_otorhinolaryngology-021 \\
 & 口腔内異常 & TBL-knowledge_otorhinolaryngology-022 \\
検査方法 & 聴力検査と平衡機能検査 & TBL-knowledge_otorhinolaryngology-023 \\
 & 味覚検査と嗅覚検査 & TBL-knowledge_otorhinolaryngology-024 \\
 & 耳鏡、鼻鏡、喉頭鏡、鼻咽腔・喉頭内視鏡 & TBL-knowledge_otorhinolaryngology-025 \\
特異的治療法 & 補聴器・人工聴覚器 & TBL-knowledge_otorhinolaryngology-026 \\
 & 気管切開 & TBL-knowledge_otorhinolaryngology-027 \\
\bottomrule
\end{xltabular}

\begin{xltabular}{\linewidth}{XXX}
\caption{\label{tbl:knowledge_psycho}精神系} \\
\toprule
分類 & 項目名 & id \\
\midrule
\endhead
構造と機能 & ※神経系の項目を参照 & TBL-knowledge_psycho-001 \\
症候 & 不安・抑うつ & TBL-knowledge_psycho-002 \\
 & 認知障害 & TBL-knowledge_psycho-003 \\
 & 意識障害 & TBL-knowledge_psycho-004 \\
 & 不眠 & TBL-knowledge_psycho-005 \\
 & 幻覚・妄想 & TBL-knowledge_psycho-006 \\
 & 心気症 & TBL-knowledge_psycho-007 \\
検査方法 & 質問紙法 & TBL-knowledge_psycho-008 \\
 & Rorschachテスト & TBL-knowledge_psycho-009 \\
 & 簡易精神症状評価尺度(Brief Psychiatric Rating Scale (BPRS)) & TBL-knowledge_psycho-010 \\
 & Hamiltonうつ病評価尺度 & TBL-knowledge_psycho-011 \\
 & Beckのうつ病自己評価尺度 & TBL-knowledge_psycho-012 \\
 & 状態特性不安検査(State-Trait Anxiety Inventory  (STAI)) & TBL-knowledge_psycho-013 \\
 & Mini-Mental State Examination (MMSE) & TBL-knowledge_psycho-014 \\
 & 改訂長谷川式簡易知能評価スケール & TBL-knowledge_psycho-015 \\
 & 精神科診断分類法 & TBL-knowledge_psycho-016 \\
特異的治療法 & 精神科面接 & TBL-knowledge_psycho-017 \\
 & 精神保健及び精神障害者福祉に関する法律、心神喪失者等医療観察法の適用場面 & TBL-knowledge_psycho-018 \\
 & コンサルテーション・リエゾン精神医学 & TBL-knowledge_psycho-019 \\
\bottomrule
\end{xltabular}

\begin{xltabular}{\linewidth}{XXX}
\caption{\label{tbl:knowledge_er}救急系(中毒・環境因子による疾患を含む)} \\
\toprule
分類 & 項目名 & id \\
\midrule
\endhead
症候 & 地域の救急医療体制について病院前救護体制、メディカルコントロール、初期・二次・三次救急医療の概念を用いて説明できる & TBL-knowledge_er-001 \\
 & 意識障害 & TBL-knowledge_er-002 \\
 & 失神 & TBL-knowledge_er-003 \\
 & けいれん & TBL-knowledge_er-004 \\
 & 呼吸困難 & TBL-knowledge_er-005 \\
 & 胸痛 & TBL-knowledge_er-006 \\
 & 運動麻痺・筋力低下 & TBL-knowledge_er-007 \\
 & 腹痛 & TBL-knowledge_er-008 \\
 & 悪心・嘔吐 & TBL-knowledge_er-009 \\
 & 吐血 & TBL-knowledge_er-010 \\
 & ショック & TBL-knowledge_er-011 \\
 & 心停止:心血管原性:急性心筋梗塞 & TBL-knowledge_er-012 \\
 & 心停止:心血管原性:急性大動脈解離 & TBL-knowledge_er-013 \\
 & 心停止:心血管原性:大動脈瘤破裂 & TBL-knowledge_er-014 \\
 & 心停止:心血管原性:肺塞栓 & TBL-knowledge_er-015 \\
 & 心停止:呼吸原性:気道閉塞 & TBL-knowledge_er-016 \\
 & 心停止:呼吸原性:緊張性気胸 & TBL-knowledge_er-017 \\
 & 心停止:呼吸原性:肺実質病変による低酸素血症 & TBL-knowledge_er-018 \\
 & 心停止:神経原性:重症頭部・脊髄外傷 & TBL-knowledge_er-019 \\
 & 心停止:神経原性:急性くも膜下出血 & TBL-knowledge_er-020 \\
 & 心停止:中毒・環境要因:中毒 & TBL-knowledge_er-021 \\
 & 心停止:中毒・環境要因:熱中症 & TBL-knowledge_er-022 \\
 & 心停止:中毒・環境要因:低体温症 & TBL-knowledge_er-023 \\
 & 心停止:電解質・酸塩基平衡異常:低・高カリウム血症 & TBL-knowledge_er-024 \\
 & 心停止:電解質・酸塩基平衡異常:アシドーシス & TBL-knowledge_er-025 \\
 & 心停止:電解質・酸塩基平衡異常:低血糖 & TBL-knowledge_er-026 \\
 & 中毒:食中毒 & TBL-knowledge_er-027 \\
 & 中毒:ガス中毒:一酸化炭素中毒 & TBL-knowledge_er-028 \\
 & 中毒:ガス中毒:硫化水素 & TBL-knowledge_er-029 \\
 & 中毒:ガス中毒:青酸ガス & TBL-knowledge_er-030 \\
 & 中毒:農薬:有機リン & TBL-knowledge_er-031 \\
 & 中毒:農薬:有機塩素 & TBL-knowledge_er-032 \\
 & 中毒:アルコール & TBL-knowledge_er-033 \\
 & 中毒:薬物:睡眠薬 & TBL-knowledge_er-034 \\
 & 中毒:薬物:向精神薬 & TBL-knowledge_er-035 \\
 & 中毒:薬物:解熱鎮痛薬 & TBL-knowledge_er-036 \\
 & 中毒:薬物:麻薬 & TBL-knowledge_er-037 \\
 & 中毒:薬物:覚醒剤 & TBL-knowledge_er-038 \\
 & 中毒:水銀 & TBL-knowledge_er-039 \\
 & 中毒:鉛 & TBL-knowledge_er-040 \\
 & 中毒:青酸 & TBL-knowledge_er-041 \\
 & 中毒:ヒ素 & TBL-knowledge_er-042 \\
 & 中毒:パラコート & TBL-knowledge_er-043 \\
 & 中毒:自然毒 & TBL-knowledge_er-044 \\
 & 中毒:腐食剤:酸 & TBL-knowledge_er-045 \\
 & 中毒:腐食剤:アルカリ & TBL-knowledge_er-046 \\
 & 中毒:腐食剤:フッ化水素 & TBL-knowledge_er-047 \\
 & 中毒:ボタン電池誤飲  & TBL-knowledge_er-048 \\
\bottomrule
\end{xltabular}

\begin{xltabular}{\linewidth}{XXX}
\caption{\label{tbl:knowledge_autoimmune}免疫・アレルギー} \\
\toprule
分類 & 項目名 & id \\
\midrule
\endhead
症候 & 発熱 & TBL-knowledge_autoimmune-001 \\
 & 全身倦怠感 & TBL-knowledge_autoimmune-002 \\
 & 皮疹 & TBL-knowledge_autoimmune-003 \\
 & リンパ節腫脹 & TBL-knowledge_autoimmune-004 \\
 & 浮腫 & TBL-knowledge_autoimmune-005 \\
 & 呼吸困難 & TBL-knowledge_autoimmune-006 \\
 & 咳・痰 & TBL-knowledge_autoimmune-007 \\
 & 血尿 & TBL-knowledge_autoimmune-008 \\
 & 関節痛・関節腫脹 & TBL-knowledge_autoimmune-009 \\
検査方法 & 自己抗体の種類と臨床的意義を説明できる。 & TBL-knowledge_autoimmune-010 \\
特異的治療法 & 免疫抑制薬による治療 & TBL-knowledge_autoimmune-011 \\
 & リウマチ性疾患へのリハビリテーション & TBL-knowledge_autoimmune-012 \\
\bottomrule
\end{xltabular}

\begin{xltabular}{\linewidth}{XXX}
\caption{\label{tbl:knowledge_infection}感染症} \\
\toprule
分類 & 項目名 & id \\
\midrule
\endhead
症候 & ショック & TBL-knowledge_infection-001 \\
 & 発熱・高体温 & TBL-knowledge_infection-002 \\
 & けいれん & TBL-knowledge_infection-003 \\
 & 意識障害・失神 & TBL-knowledge_infection-004 \\
 & 脱水 & TBL-knowledge_infection-005 \\
 & 全身倦怠感 & TBL-knowledge_infection-006 \\
 & 黄疸 & TBL-knowledge_infection-007 \\
 & 発疹 & TBL-knowledge_infection-008 \\
 & リンパ節腫脹 & TBL-knowledge_infection-009 \\
 & 浮腫 & TBL-knowledge_infection-010 \\
 & 胸水 & TBL-knowledge_infection-011 \\
 & 胸痛・胸部圧迫感 & TBL-knowledge_infection-012 \\
 & 呼吸困難・息切れ & TBL-knowledge_infection-013 \\
 & 咳・痰 & TBL-knowledge_infection-014 \\
 & 血痰・喀血 & TBL-knowledge_infection-015 \\
 & 頭痛・頭重感 & TBL-knowledge_infection-016 \\
 & 腹痛 & TBL-knowledge_infection-017 \\
 & 悪心・嘔吐 & TBL-knowledge_infection-018 \\
 & 便秘・下痢・血便 & TBL-knowledge_infection-019 \\
 & 吐血・下血 & TBL-knowledge_infection-020 \\
 & 血尿・タンパク尿 & TBL-knowledge_infection-021 \\
 & 関節痛・関節腫脹 & TBL-knowledge_infection-022 \\
 & 腰背部痛 & TBL-knowledge_infection-023 \\
 & 咽頭痛 & TBL-knowledge_infection-024 \\
\bottomrule
\end{xltabular}

\begin{xltabular}{\linewidth}{XXX}
\caption{\label{tbl:knowledge_oncology}腫瘍} \\
\toprule
分類 & 項目名 & id \\
\midrule
\endhead
症候 & 発熱 & TBL-knowledge_oncology-001 \\
 & 食欲低下 & TBL-knowledge_oncology-002 \\
 & 体重減少 & TBL-knowledge_oncology-003 \\
 & 貧血 & TBL-knowledge_oncology-004 \\
 & リンパ節腫脹 & TBL-knowledge_oncology-005 \\
\bottomrule
\end{xltabular}
